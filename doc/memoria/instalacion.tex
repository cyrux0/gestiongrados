% -*-instalacion.tex-*-
% Este fichero es parte de la plantilla LaTeX para
% la realización de Proyectos Final de Carrera, protejido
% bajo los términos de la licencia GFDL.
% Para más información, la licencia completa viene incluida en el
% fichero fdl-1.3.tex

% Copyright (C) 2009 Pablo Recio Quijano 

Veamos que tenemos que hacer para instalar \LaTeX{} con todas sus
capacidades en un sistema basado en Debian, como Ubuntu.
Primero hay que tener en cuenta que \LaTeX{} es relativamente pesado
con respecto a otros compiladores. \\

Nosotros vamos a utilizar la distribución de \LaTeX{} incluida en los
repositorios de Ubuntu llamada \programa{texlive}. Si la buscas en
tu gestor de paquetes, encontrarás infinidad de paquetes aparte
del principal. Existen otras distribuciones como Te\TeX\\

Si instalas solo los básicos, es decir instalas \programa{texlive} y
los programas necesarios para él, no podrás compilar este documento,
ya que faltarian paquetes tales como \programa{supertabular} y
varios. Por eso, si no tienes problema de espacio en el disco duro te
recomiendo que instales el paquete \programa{texlive-full}, que
instala \negrita{todos} los paquetes de \programa{texlive}, incluyendo
documentación en todos los idiomas disponibles. Si buscas no tener
problemas de dependencias, este es tu método.\\

\begin{lstlisting}[style=consola]
  sudo apt-get install texlive-full
\end{lstlisting}

En caso de querer ser un poco más concreto, en principio puedes
trabajar con la más básica (\programa{texlive} y sus dependencias) y
en función de los paquetes que te vayan faltando, los instalas.

% TO-DO: paquetes concretos e instalación en otros sistemas