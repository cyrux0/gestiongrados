
\documentclass{book}

% Esto es para poder escribir acentos directamente:
\usepackage[utf8]{inputenc}
% Esto es para que el LaTeX sepa que el texto está en español:
\usepackage[spanish]{babel}
\usepackage{verbatim}
\usepackage{hyperref}
\hypersetup{colorlinks=true,linkcolor=black}
\title{Casos de uso de la aplicación para incluir en la memoria}

\author{Daniel Ignacio Salazar Recio}

\begin{document}
\maketitle
\chapter{DRS}
\section{Modelo de requisitos}
\subsection{Modelo de casos de uso}
\subsubsection*{Gestión de carga de trabajo}
Esta es la sección de gestión de carga de trabajo.


\subsubsection*{Caso de uso: Seleccionar titulación}
\label{select_titulacion}
\begin{itemize}
\item{\bf Descripción:} Caso de uso abstracto incluído en otros casos de uso para seleccionar una titulación de una lista de disponibles.
\item{\bf Actores:} Usuario
\item{\bf Precondiciones:} Ninguna.
\item{\bf Postcondiciones:} Se selecciona una titulación para su uso en otra finalidad.
\item{\bf Escenario principal:}
	\begin{enumerate}
	\item El sistema muestra un listado de las titulaciones disponibles.
	\item El usuario selecciona la titulación deseada.
	\end{enumerate}
\item{\bf Escenarios alternativos:}
	\begin{itemize}
		\item[1.a.] No hay ninguna titulación registrada.
		\begin{enumerate}
			\item El sistema indica el error y el caso de uso finaliza.
		\end{enumerate}
	\end{itemize}
\end{itemize}

\subsubsection*{Caso de uso: Registrar titulación}
\begin{itemize}
\item{\bf Descripción:} Registra una nueva titulación en el sistema.
\item{\bf Actores:} Administrador.
\item{\bf Precondiciones:} Ninguna.
\item{\bf Postcondiciones:} La titulación queda registrada.
\item{\bf Escenario principal:}
\begin{enumerate}
\item El administrador introduce el código, el nombre y todos los demás datos de la titulación.
\item El sistema comprueba que los datos cumplen el formato.
\item El sistema confirma el alta de la titulación mostrando un mensaje.
\end{enumerate}
\item{\bf Escenarios alternativos:}
	\begin{itemize}
	\item[2.a.] Alguno de los datos introducidos tiene un formato incorrecto.
		\begin{enumerate}
		\item El sistema lo indica mostrando un mensaje de error y se vuelve al paso anterior.
		\end{enumerate}
	\item[2.b.] Falta algún campo obligatorio.
		\begin{enumerate}
		\item El sistema lo indica mostrando un mensaje de error y se vuelve al paso anterior.
		\end{enumerate}
	\item[2.c.] Ya existe alguna titulación con ese código o nombre.
		\begin{enumerate}
		\item El sistema indica el error y se vuelve al paso anterior.
		\end{enumerate}
	\item[*a.] El administrador decide cancelar el registro en cualquier momento, el caso de uso termina. 
	\end{itemize}
\end{itemize}

\subsubsection*{Caso de uso: Editar titulación}
\begin{itemize}
\item{\bf Descripción:} Edita una titulación existente en el sistema modificando sus datos.
\item{\bf Actores:} Administrador.
\item{\bf Precondiciones:} Ninguna.
\item{\bf Postcondiciones:} La titulacion queda modificada en el sistema.
\item{\bf Escenario principal:}
\begin{enumerate}
	\item Se realiza el caso de uso {\em\hyperref[select_titulacion]{Seleccionar titulación}}.
	\item El sistema muestra sus datos actuales, permitiendo su edición.
	\item El administrador modifica los datos.
	\item El sistema comprueba que todos los datos son correctos.
	\item El sistema muestra un mensaje indicando que la edición se ha completado.
\end{enumerate}
\item{\bf Escenarios alternativos:}
	\begin{itemize}
	\item[4.a.] Alguno de los datos tiene un formato incorrecto.
		\begin{enumerate}
		\item El sistema muestra un mensaje de error indicándolo, a continuación se vuelve al paso anterior.
		\end{enumerate}
	\item[4.b.] Falta algún campo obligatorio por rellenar.
		\begin{enumerate}
		\item El sistema muestra un mensaje de error indicándolo, a continuación se vuelve al paso anterior.
		\end{enumerate}
	\item[4.c.] Ya existe alguna titulación con el nombre o código introducidos.
		\begin{enumerate}
		\item El sitema indica el error y se vuelve al paso anterior.
		\end{enumerate}
	\item[*a.] En cualquier momento el administrador decide cancelar la edición, el caso de uso se da por terminado.
	\end{itemize}
\end{itemize}

\subsubsection*{Caso de uso: Borrar titulación}
% Tomar este CU como ejemplo
\begin{itemize}
\item{\bf Descripción:} Borra una titulación del sistema.
\item{\bf Actores:} Administrador.
\item{\bf Precondiciones:} La titulación existe en el sistema.
\item{\bf Postcondiciones:} La titulación queda eliminada del sistema.
\item{\bf Escenario principal:}
	\begin{enumerate}
	\item Se realiza el caso de uso  {\em\hyperref[select_titulacion]{Seleccionar titulación}}.
	\item El sistema muestra un diálogo de confirmación.
	\item El administrador confirma que quiere eliminar la titulación del sistema.
	\item El sistema elimina la titulación.
	\item El sistema muestra un mensaje confirmando que se ha eliminado la titulación.
	\end{enumerate}
\item{\bf Escenarios alternativos:}
	\begin{itemize}
	\item[3.a.] El administrador selecciona que no desea eliminar la titulación.
		\begin{enumerate}
		\item El caso de uso se reinicia.
		\end{enumerate}
	\item[*a.] En cualquier momento el administrador decide cancelar la eliminación.
		\begin{enumerate}
		\item El caso de uso se termina.
		\end{enumerate}
	\end{itemize}
\end{itemize}

\subsubsection*{Caso de uso: Ver detalles de titulación}
\begin{itemize}
\item{\bf Descripción:} Muestra los datos de una titulación en detalle, así como sus asignaturas.
\item{\bf Actores:} Usuario.
\item{\bf Precondiciones:} La titulación existe en el sistema.
\item{\bf Postcondiciones:} Los datos de la titulación se muestran por pantalla.
\item{\bf Escenario principal:}
	\begin{enumerate}
	\item Se realiza el caso de uso {\em\hyperref[select_titulacion]{Seleccionar titulación}}.
	\item El sistema muestra los datos de la titulación y un listado de sus asignaturas si las tiene.
	\end{enumerate}
\item{\bf Escenarios alternativos:}
	\begin{itemize}
	\item[*a.] En cualquier momento el usuario decide cancelar el proceso, el caso de uso se termina.
	\end{itemize}
\end{itemize}

\subsubsection*{Caso de uso: Seleccionar asignatura}
\label{select_asignatura}
\begin{itemize}
\item{\bf Descripción:} Caso de uso abstracto incluído por otros casos de uso para seleccionar una asignatura del listado de las que tiene disponibles una titulación concreta.
\item{\bf Actores:} Usuario
\item{\bf Precondiciones:} Ninguna.
\item{\bf Postcondiciones:} Queda seleccionada una asignatura para algún fin concreto de otro caso de uso.
\item{\bf Escenario principal:}
	\begin{enumerate}
	\item Se realiza el caso de uso {\em\hyperref[select_titulacion]{Seleccionar titulación}}.
	\item El sistema muestra un listado de las asignaturas disponibles asociadas a la titulación seleccionada.
	\item El usuario selecciona una asignatura de la lista.
	\end{enumerate}
\item{\bf Escenarios alternativos:}
	\begin{itemize}
		\item[2.a.] No hay ninguna asignatura registrada en esa titulación.
		\begin{enumerate}
			\item El sistema indica el error y el caso de uso finaliza.
		\end{enumerate}
	\end{itemize}
\end{itemize}

% Posible caso de uso Copiar asignatura
\subsubsection*{Caso de uso: Registrar asignatura}
\begin{itemize}
\item{\bf Descripción:} Se da de alta una nueva asignatura en el sistema.
\item{\bf Actores:} Administrador.
\item{\bf Precondiciones:} Existe alguna titulación con la que asociar la asignatura.
\item{\bf Postcondiciones:} La asignatura queda registrada en el sistema.
\item{\bf Escenario principal:}
	\begin{enumerate}
 	\item Se realiza el caso de uso {\em\hyperref[select_titulacion]{Seleccionar titulación}}.
	\item El sistema muestra un formulario para introducir los datos.
	\item El administrador introduce el código, el nombre y todos los demás datos de la asignatura.
	\item El sistema comprueba que todos los datos cumplen el formato requerido.
	\item El sistema registra la asignatura y muestra un mensaje confirmándolo.
	\end{enumerate}
\item{\bf Escenarios alternativos:}
	\begin{itemize}
	\item[1.a.] No hay titulaciones registradas en el sistema.
		\begin{enumerate}
		\item El sistema indica el error y el caso de uso se cancela.
		\end{enumerate}
	\item[5.a.] Alguno de los datos no cumple el formato correcto.
		\begin{enumerate}
		\item El sistema indica el error y se vuelve al paso anterior.
		\end{enumerate}
	\item[5.b.] Falta por rellenar algún campo obligatorio.
		\begin{enumerate}
		\item El sistema indica el error y se vuelve al paso anterior.
		\end{enumerate}
	\item[5.c.] Ya existe alguna asignatura con ese código o nombre.
		\begin{enumerate}
		\item El sistema indica el error y se vuelve al paso anterior.
		\end{enumerate}
	\item[*a.] En cualquier momento el administrador decide cancelar el proceso.
		\begin{enumerate}
		\item El caso de uso se cancela.
		\end{enumerate}
	\end{itemize}
\end{itemize}

\subsubsection*{Caso de uso: Editar Asignatura}
\begin{itemize}
\item{\bf Descripción:} Se modifican los datos de una asignatura existente en el sistema.
\item{\bf Actores:} Administrador.
\item{\bf Precondiciones:} Ninguna.
\item{\bf Postcondiciones:} La asignatura queda modificada en el sistema.
\item{\bf Escenario principal:}
	\begin{enumerate}
	\item Se realiza el caso de uso {\em \hyperref[select_asignatura]{Seleccionar asignatura}}.
	\item El sistema muestra los datos de la asignatura en un formato editable.
	\item El administrador hace las modificaciones que considere necesarias.
	\item El sistema comprueba que los datos modificados cumplen el formato requerido.
	\item El sistema guarda la asignatura y muestra un mensaje confirmándolo.
	\end{enumerate}
\item{\bf Escenarios alternativos:}
	\begin{itemize}
	\item[4.a.] Alguno de los datos no cumple el formato correcto.
		\begin{enumerate}
		\item El sistema indica el error y se vuelve al paso anterior.
		\end{enumerate}
	\item[4.b.] Falta por rellenar algún campo obligatorio.
		\begin{enumerate}
		\item El sistema indica el error y se vuelve al paso anterior.
		\end{enumerate}
	\item[4.c.] Ya existe alguna asignatura con ese código o nombre.
		\begin{enumerate}
		\item El sistema indica el error y se vuelve al paso anterior.
		\end{enumerate}
	\item[*a.] En cualquier momento el administrador decide cancelar el proceso.
		\begin{enumerate}
		\item El caso de uso se cancela.
		\end{enumerate}
	\end{itemize}
\end{itemize}

\subsubsection*{Caso de uso: Borrar asignatura}
\begin{itemize}
\item{\bf Descripción:} Se borra una asingatura del sistema.
\item{\bf Actores:} Administrador.
\item{\bf Precondiciones:} Ninguna.
\item{\bf Postcondiciones:} La asignatura queda eliminada del sistema.
\item{\bf Escenario principal:}
	\begin{enumerate}
	\item Se realiza el caso de uso {\em \hyperref[select_asignatura]{Seleccionar asignatura}}.
	\item El sistema muestra un diálogo de confirmación.
	\item El administrador confirma que desea borrar la asignatura.
	\item El sistema borra la asignatura y muestra un mensaje confirmándolo.
	\end{enumerate}
\item{\bf Escenarios alternativos:}
	\begin{itemize}
	\item[3.a.] El administrador selecciona que no desea eliminar la asignatura.
		\begin{enumerate}
		\item El caso de uso se reinicia.
		\end{enumerate}
	\item[*a.] En cualquier momento el administrador decide cancelar la eliminación.
		\begin{enumerate}
		\item El caso de uso se termina.
		\end{enumerate}
	\end{itemize}
\end{itemize}

\subsubsection*{Caso de uso: Consultar asignatura}
\begin{itemize}
\item{\bf Descripción:} Muestra los datos en detalle de una asignatura.
\item{\bf Actores:} Usuario.
\item{\bf Precondiciones:} Ninguna.
\item{\bf Postcondiciones:} Se muestran los datos de la asignatura por pantalla.
\item{\bf Escenario principal:} 
	\begin{enumerate}
	\item Se realiza el caso de uso {\em \hyperref[select_asignatura]{Seleccionar asignatura}}.
	\item El sistema muestra la información relacionada con la asignatura.
	\end{enumerate}
\item{\bf Escenarios alternativos:}
	\begin{itemize}
	\item[*a.] En cualquier momento el administrador decide cancelar el proceso.
		\begin{enumerate}
		\item El caso de uso se termina.
		\end{enumerate}
	\end{itemize}
\end{itemize}

\subsubsection*{Caso de uso: Añadir carga de trabajo}
\begin{itemize}
\item{\bf Descripción:} Se añaden los detalles de la carga de trabajo para una asignatura en un curso determinado.
\item{\bf Actores:} Subdirector.
\item{\bf Precondiciones:} Ninguna.
\item{\bf Postcondiciones:} La carga de trabajo queda registrada en el sistema asociada a una asignatura y un curso determinado.
\item{\bf Escenario principal:}
	\begin{enumerate}
	\item Se realiza el caso de uso {\em \hyperref[select_asignatura]{Seleccionar asignatura}}.
	\item El sistema muestra un listado con los cursos disponibles.
	\item El usuario selecciona el curso para el que desea añadir la carga de trabajo.
	\item El sistema comprueba que no exista ya una carga de trabajo asociada a ese curso.
	\item El usuario introduce los datos de la carga de trabajo.
	\item El sistema comprueba que los datos cumplen el formato requerido.
	\item El sistema guarda la carga de trabajo y muestra un mensaje confirmándolo.
	\end{enumerate}
\item{\bf Escenarios alternativos:}
	\begin{itemize}
	\item[5.a.] Ya existe una carga de trabajo establecida para el curso seleccionado.
		\begin{enumerate}
		\item El sistema indica el error y el caso de uso vuelve al paso anterior.
		\end{enumerate}
	\item[7.a.] Alguno de los datos introducidos no cumple el formato correcto.
		\begin{enumerate}
		\item El sistema indica el error y se vuelve al paso anterior.
		\end{enumerate}
	\item[7.b.] Alguno de los campos obligatorios no ha sido rellenado.
		\begin{enumerate}
		\item El sistema indica el error y se vuelve al paso anterior.
		\end{enumerate}	
	\item[*a.] En cualquier momento el administrador decide cancelar el proceso.
		\begin{enumerate}
		\item El caso de uso se termina.
		\end{enumerate}
	\end{itemize}
\end{itemize}

\subsubsection*{Caso de uso: Editar carga de trabajo}
\begin{itemize}
\item{\bf Descripción:} Se edita una carga de trabajo existente para un curso determinado.
\item{\bf Actores:} Subdirector.
\item{\bf Precondiciones:} Ninguna.
\item{\bf Postcondiciones:} La carga de trabajo queda modificada en el sistema.
\item{\bf Pasos:}
	\begin{enumerate}
	\item El sistema muestra un listado de las asignaturas disponibles.
	\item El usuario selecciona la asignatura para la que desea editar la carga de trabajo.
	\item El sistema muestra un listado de los cursos disponibles.
	\item El usuario selecciona el curso para el que desea editar la carga de trabajo.
	\item El sistema comprueba que exista una carga asociada a ese curso.
	\item El sistema muestra los datos de la carga en un formato editable.
	\item El usuario modifica los datos.
	\item El sistema comprueba que los datos cumplen el formato requerido.
	\item El sistema guarda los cambios y muestra un mensaje confirmándolo.
	\end{enumerate}
\item{\bf Escenarios alternativos:}
	\begin{itemize}
	\item[1.a.] No hay asignaturas registradas en el sistema.
		\begin{enumerate}
		\item El sistema indica el error y el caso de uso termina.
		\end{enumerate}
	\item[5.a.] No existe una carga de trabajo establecida para el curso seleccionado.
		\begin{enumerate}
		\item El sistema indica el error y el caso de uso vuelve al paso anterior.
		\end{enumerate}
	\item[7.a.] Alguno de los datos introducidos no cumple el formato correcto.
		\begin{enumerate}
		\item El sistema indica el error y se vuelve al paso anterior.
		\end{enumerate}
	\item[7.b.] Alguno de los campos obligatorios no ha sido rellenado.
		\begin{enumerate}
		\item El sistema indica el error y se vuelve al paso anterior.
		\end{enumerate}	
	\item[*a.] En cualquier momento el administrador decide cancelar el proceso.
		\begin{enumerate}
		\item El caso de uso se termina.
		\end{enumerate}
	\end{itemize}
\end{itemize}

% Hacer casos de uso selección de titulación y asignatura para incluirlos

\subsubsection*{Caso de uso: Borrar carga de trabajo}
\begin{itemize}
\item{\bf Descripción:} Se borra una carga de trabajo existente en el sistema asociada a un curso.
\item{\bf Actores:} Subdirector.
\item{\bf Precondiciones:} Ninguna.
\item{\bf Postcondiciones:} La carga de trabajo asociada a la asignatura y curso seleccionados queda. eliminada del sistema.
\item{\bf Escenario principal:}
	\begin{enumerate}
	\item El sistema muestra un listado de las asignaturas disponibles.
	\item El usuario selecciona la asignatura para la que desea editar la carga de trabajo.
	\item El sistema muestra un listado de los cursos disponibles.
	\item El usuario selecciona el curso para el que desea borrar la carga de trabajo.
	\item El sistema comprueba que exista una carga asociada a ese curso.
	\item El sistema muestra un diálogo de confirmación.
	\item El usuario confirma que desea borrar la carga.
	\item El sistema muestra un mensaje confirmando la eliminación y borra la carga.
	\end{enumerate}
\item{\bf Escenarios alternativos:}
	\begin{itemize}
	\item[1.a.] No hay asignaturas registradas en el sistema.
		\begin{enumerate}
		\item El sistema indica el error y el caso de uso termina.
		\end{enumerate}
	\item[5.a.] No existe una carga de trabajo establecida para el curso seleccionado.
		\begin{enumerate}
		\item El sistema indica el error y el caso de uso vuelve al paso anterior.
		\end{enumerate}
	\item[7.a.] El administrador selecciona que no desea eliminar la carga.
		\begin{enumerate}
		\item El caso de uso se reinicia.
		\end{enumerate}
	\item[*a.] En cualquier momento el administrador decide cancelar la eliminación.
		\begin{enumerate}
		\item El caso de uso se termina.
		\end{enumerate}
	\end{itemize}
\end{itemize}

% Posible caso de uso Copiar carga de trabajo
%% De este subsistema quedaría Consultar carga de trabajo (podría ser un extend de consultar asignatura)
\subsubsection*{Caso de uso: Consultar carga de trabajo}
\begin{itemize}
\item{\bf Descripción:} Se consulta la carga de trabajo de una asignatura para un curso determinado.
\item{\bf Actores:} Usuario.
\item{\bf Precondiciones:} Ninguna.
\item{\bf Postcondiciones:} Se muestran los datos al usuario por pantalla.
\item{\bf Escenario principal:}
	\begin{enumerate}
	\item El sistema muestra una lista de asignaturas disponibles.
	\item El usuario selecciona la asignatura deseada.
	\item El sistema muestra una lista de los cursos disponibles.
	\item El usuario elige el curso para el que quiere consultar la carga de trabajo.
	\item El sistema comprueba que exista una carga asociada a ese curso.
	\item La carga existe y es mostrada al usuario.
	\end{enumerate}
\item{\bf Escenarios alternativos:}
	\begin{itemize}
	\item[1.a.]No hay ninguna asignatura registrada en el sistema.
		\begin{enumerate}
		\item El sistema indica el error y el caso de uso se cancela.
		\end{enumerate}
	\item[5.a.]No existe ninguna carga asociada a ese curso.
		\begin{enumerate}
		\item El sistema muestra un mensaje informando del error y vuelve al paso anterior.
		\end{enumerate}
	\item[*a.]En cualquier momento el usuario decide cancelar el proceso.
		\begin{enumerate}
		\item El caso de uso se cancela.
		\end{enumerate}		
	\end{itemize}
\end{itemize}


% Plantilla para CUs
\begin{comment}
\begin{itemize}
\item{\bf Descripción:}
\item{\bf Actores:}
\item{\bf Precondiciones:}
\item{\bf Postcondiciones:}
\item{\bf Escenario principal:}
	\begin{enumerate}
	\item
	\end{enumerate}
\item{\bf Escenarios alternativos:}
	\begin{itemize}
		\item[2.a.]
		\begin{enumerate}
			\item
		\end{enumerate}
	\end{itemize}
\end{itemize}
\end{comment}

\end{document}
