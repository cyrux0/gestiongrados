
\documentclass{book}

% Esto es para poder escribir acentos directamente:
\usepackage[utf8]{inputenc}
% Esto es para que el LaTeX sepa que el texto está en español:
\usepackage[spanish]{babel}


\title{Casos de uso de la aplicación para incluir en la memoria
}

\author{Daniel Ignacio Salazar Recio}

\begin{document}
\maketitle
\chapter{DRS}
\section{Modelo de requisitos}
\subsection{Modelo de casos de uso}
\subsubsection*{Gestión de carga de trabajo}
Esta es la sección de gestión de carga de trabajo.

\subsubsection*{Caso de uso: Registrar titulación}
\begin{itemize}
\item{\bf Descripción:} Registra una nueva titulación en el sistema
\item{\bf Actores:} Administrador
\item{\bf Precondiciones:}
\item{\bf Pasos:}
\begin{enumerate}
\item El administrador introduce todos los datos de la titulación.
\item El sistema comprueba que los datos cumplen el formato
	\begin{enumerate}
	\item Alguno de los datos introducidos tiene un formato incorrecto
		\begin{enumerate}
		\item El sistema lo indica mostrando un mensaje de error
		\end{enumerate}
	\item Falta algún campo obligatorio
		\begin{enumerate}
		\item El sistema lo indica mostrando un mensaje de error
		\end{enumerate}
	\end{enumerate}
\item El sistema registra la titulación
\end{enumerate}
\end{itemize}

\subsubsection*{Caso de uso: Editar titulación}
\begin{itemize}
\item{\bf Descripción:} Edita una titulación existente en el sistema modificando sus datos
\item{\bf Actores:} Administrador
\item{\bf Precondiciones:} La titulación existe en el sistema
\item{\bf Pasos:}
\begin{enumerate}
\item El administrador selecciona una titulación para editarla
\item El sistema muestra sus datos actuales, permitiendo su edición
\item El administrador modifica los datos
\item El sistema comprueba que todos los datos son correctos
	\begin{enumerate}
	\item Alguno de los datos tiene un formato incorrecto
		\begin{enumerate}
		\item El sistema muestra un mensaje de error
		\end{enumerate}
	\item Falta algún campo obligatorio
		\begin{enumerate}
		\item El sistema muestra un mensaje de error
		\end{enumerate}
	\end{enumerate}
\item El sistema guarda los datos modificados
\end{enumerate}
\end{itemize}

\end{document}
