% -*-memoria.tex-*-
% Este fichero es parte de la plantilla LaTeX para
% la realización de Proyectos Final de Carrera, protejido
% bajo los términos de la licencia GFDL.
% Para más información, la licencia completa viene incluida en el
% fichero fdl-1.3.tex

% Copyright (C) 2009 Pablo Recio Quijano 

%-------------------------------------------------------
% ---- Plantilla para libros / memorias PFC -----

% Realizada por Pablo Recio Quijano y Noelia Sales Montes 
% Formato de portada y primera página tomado del PFC de
% Francisco Javier Vázquez Púa, en su proyecto 'libgann'
% -------------------------------------------------------

\documentclass[a4paper,11pt]{book}

\usepackage{./estilos/estiloBase} % Basicamente son todas las
                                  % librerias usadas. En caso de que
                                  % falten librerias se van añadiendo
                                  % al fichero.
\usepackage{./estilos/colores}  % Algunos colores ya generados, para
                                % los algunos estilos más avanzados.
\usepackage{./estilos/comandos} % Algunos comandos personalizados
\usepackage{longtable}
\graphicspath{{./imagenes/}} % Indicamos la ruta donde se encuentran
                             % las imagenes, para ahorrarnos la ruta
                             % completa, y solo modificar aquí si en
                             % un momento dado lo movemos

\begin{document}

% Renombramos las figuras y las tablas
\renewcommand{\figurename}{Figura}
\renewcommand{\listfigurename}{Indice de figuras}
\renewcommand{\tablename}{Tabla}
\renewcommand{\listtablename}{Indice de tablas}

\pagestyle{empty}
\input{portada.tex}
\cleardoublepage

\input{primerahoja.tex}
\cleardoublepage
\pagestyle{plain}

\frontmatter % Introducción, índices ...

% -*-previo.tex-*-
% Este fichero es parte de la plantilla LaTeX para
% la realización de Proyectos Final de Carrera, protejido
% bajo los términos de la licencia GFDL.
% Para más información, la licencia completa viene incluida en el
% fichero fdl-1.3.tex

% Copyright (C) 2009 Pablo Recio Quijano 

\section*{Agradecimientos}

Me gustaria dar las gracias a mi familia por haberme apoyado durante el desarrollo del proyecto en especial mis padres y mi hermano, así como a los compañeros de la carrera que me han ayudado con consejos y ánimos. También dar las gracias a mi tutor de proyecto Juan José Domínguez por los consejos y supervisión.

\cleardoublepage

\section*{Licencia} % Por ejemplo GFDL, aunque puede ser cualquiera

Este documento ha sido liberado bajo Licencia GFDL 1.3 (GNU Free
Documentation License). Se incluyen los términos de la licencia en
inglés al final del mismo.\\

Copyright (c) 2011 Daniel Ignacio Salazar Recio.\\

Permission is granted to copy, distribute and/or modify this document under the
terms of the GNU Free Documentation License, Version 1.3 or any later version
published by the Free Software Foundation; with no Invariant Sections, no
Front-Cover Texts, and no Back-Cover Texts. A copy of the license is included in
the section entitled "GNU Free Documentation License".\\

\cleardoublepage

\section*{Notación y formato}

Cuando nos refiramos a un programa en concreto, utilizaremos la
notación: \\ \programa{NetBeans}.\\

Cuando nos refiramos a un comando, o función de un lenguaje, usaremos
la notación: \\ \comando{save}.\\

Cuando nos refiramos a un comando o expresión en consola, usaremos la notación:
\begin{lstlisting}[style=consola]
	mysql > FLUSH PRIVILEGES
\end{lstlisting}

Cuando nos refiramos a fragmentos de código, usaremos la notación:
\begin{lstlisting}[style=PHP]
class Titulaciones extends MY_Controller 
{

    function __construct() {
        parent::__construct();

        $this->layout = '';
        $this->notices = '';
        $this->alerts = '';
        $this->_filter(array('add', 'create', 
			'delete', 'edit', 'update',
			'show_informes', 'show', 
			'exportar_planificacion'), 
			array($this, 'authenticate'), 1); 
	}
}
\end{lstlisting}
\cleardoublepage

\tableofcontents
\listoffigures
\listoftables

\mainmatter % Contenido en si ...

\chapter{Introducción}
% -*-cap1.tex-*-
% Este fichero es parte de la plantilla LaTeX para
% la realización de Proyectos Final de Carrera, protejido
% bajo los términos de la licencia GFDL.
% Para más información, la licencia completa viene incluida en el
% fichero fdl-1.3.tex

% Copyright (C) 2009 Pablo Recio Quijano 

Con este Proyecto de Fin de Carrera se pretende la consecución de dos objetivos fundamentales: poner en práctica los conocimientos adquiridos en la titulación de Ingeniería Técnica en Informática de Sistemas y buscar un incremento de los conocimientos en la rama del desarrollo web, al no haber estudiado nada de este tema durante la carrera. \\

\section{Objetivos y alcance}

El proyecto consiste en la creación de un software que ayude a los coordinadores de las titulaciones de grado a programar la planificación docente. En principio está pensado únicamente para el contexto de la Escuela Superior de Ingeniería, aunque debería ser fácilmente adaptable a otras facultades. \\

Actualmente para hacer esta planificación se utilizan hojas de cálculo de {\em Microsoft Excel} o {\em OpenOffice}, haciendo que el trabajo sea algo tedioso al tener que comprobar multitud de factores manualmente. La aplicación pretende facilitar esta labor, realizando esas comprobaciones automáticamente. Por ejemplo la tarea de realizar un horario y comprobar que un aula no esté ya ocupada por otra asignatura. Para ello el objetivo es crear una aplicación web de código abierto.\\

Otro objetivo que se pretende con este proyecto es hacerlo escalable, para que en un futuro se le puedan realizar las ampliaciones necesarias sin necesidad de cambiar demasiado lo que está ya hecho.

\section{Estructura del documento}

El documento se compone de los siguientes capítulos:

\begin{itemize}
\item {\bf Introducción:} descripción del proyecto, objectivos y alcance del mismo y estructura básica del documento.
\item {\bf Planificación:} descripción del desarrollo de la planificación temporal y plazos de realización.
\item {\bf Descripción general:} descripción detallada sobre el proyecto, especificando tecnologías y herramientas usadas para su desarrollo.
\item {\bf Análisis:} fase de análisis del sistema, empleando la metodología seleccionada. Definición de requisitos funcionales del sistema, modelo conceptual y modelo de comportamiento.
\item {\bf Diseño:} fase de diseño del sistema, diseño de la base de datos y diagramas de clase aplicadas al diseño.
\item {\bf Implementación:} aspectos más relevantes de la fase de implementación del sistema y explicación de los problemas encontrados durante el desarrollo.
\item {\bf Pruebas y validaciones:} pruebas realizadas al software para verificar que todo funciona correctamente y según lo esperado.
\item {\bf Conclusiones:} valoración y conclusiones personales obtenidas tras la realización del proyecto.
\item {\bf Apéndices:}
\begin{itemize}
\item {\bf Manual de instalación:} manual para instalar correctamente la aplicación.
\item {\bf Manual de usuario:} manual para ayudar al usuario en el uso de la aplicación.
\item {\bf Manual de importación:} ejemplos de los archivos admitidos por la aplicación para importar y exportar datos.
\item {\bf Exportación:} ejemplos de los archivos exportados por la aplicación.
\end{itemize}
\item {\bf Bibliografía:} libros y referencias consultadas durante la realización del proyecto.
\item {\bf Licencia GPL 3:} texto completo sobre la licencia GPL 3, por la cual se rige el proyecto.
\end{itemize}

\section{Definiciones y acrónimos}

A continuación se detallan las abreviaturas y acrónimos utilizados a lo largo de todo el documento.

\begin{itemize}
\item {\bf PHP:} PHP: Hypertext Preprocessor. Es un lenguaje de scripting del lado del servidor.
\item {\bf XHTML:} eXtensible Hypertext Markup Language. Lenguaje de marcado para estructurar las vistas de un documento web.
\item {\bf IDE:} Entorno de desarrollo integrado. Es una aplicación con herramientas para facilitar el trabajo de un desarrollador.
\item {\bf SQL:} Lenguaje de consulta estructurado. Lenguaje para realizar operaciones sobre una base de datos.
\item {\bf MySQL:} Sistema de gestión de base de datos relacional.
\item {\bf CSS:} Cascading Style Sheets, hojas de estilo en cascada. Utilizadas para definir el estilo de un documento web.
\item {\bf ER:} Entidad-relación. Diagrama utilizado para mostrar la especificación de una base de datos.
\item {\bf ESI:} Escuela Superior de Ingenieria.
\item {\bf MVC:} Modelo Vista Controlador. Patrón de diseño arquitectónico utilizado a la hora de hacer el diseño de un sistema.
\end{itemize}

\chapter{Planificación}

La planificación se divide en varias fases, a continuación se explicará en detalle cada una de ellas.

\section{Fase inicial}

La primera fase consistió en el planteamiento de la idea del proyecto, que en principio era desarrollar una aplicación web. El tutor finalmente decide proponer este proyecto, dejando al alumno la libre elección de las tecnologías utilizadas para su desarrollo.

\section{Fase de análisis}

Se realizan diversas reuniones con el tutor para hacer el planteamiento y una especificación informal de los requisitos. Debido a la complejidad de la aplicación fue una labor compleja que necesitó de varias reuniones y en las que hubo cambios en los requisitos debido a la forma en la que se realiza la planificación docente de las titulaciones de grado, que ha cambiado durante los últimos años.

\section{Fase de aprendizaje}

Para la realización del proyecto usaron tecnologías de las que no se tenían conocimiento, por lo tanto fue necesaria una amplia fase de aprendizaje. Esta fase se puede dividir en tres partes, el aprendizaje de PHP como lenguaje, aprendizaje del framework {\em CodeIgniter}\footnote{Framework de desarrollo en PHP, basado en el patrón MVC} y del ORM\footnote{Object Relational Mapper. Es una técnica software para convertir datos entre sistemas incompatibles en lenguajes de programación orientados a objetos. Por ejemplo datos de una base de datos en objetos del lenguaje PHP} {\em Doctrine}\footnote{Implementación para PHP de un ORM}, y finalmente de los lenguajes de la parte del cliente, es decir, HTML y JavaScript\footnote{Lenguaje de scripting del lado del cliente. Basado en objetos}.

\section{Fase de diseño}

Fase en la que se realiza el diseño de la aplicación. Es importante hacer un buen diseño aquí para que no surjan problemas más adelante y se haga necesario hacer cambios muy costosos en el diseño.

\section{Implementación}

Fase más extensa del desarrollo del proyecto. Consiste en implementar los requisitos especificados en la fase de análisis siguiendo para ello el diseño realizado en la fase anterior, procurando que la aplicación final satisfaga las necesidades.

\section{Pruebas}

Etapa importante en la que se comprueba una por una las funcionalidades del sistema verificando que no hay errores y que todo funciona como debe.

\section{Redacción de la memoria}
Esta fase se ha ido solapando con las demás ya que se ha realizado conjuntamente a las otras a medida que se iba desarrollando el proyecto. 

\section{Diagrama de Gantt}
A continuación se muestra el diagrama de Gantt\footnote{Herramienta gráfica para mostrar el tiempo previsto para la realización de tareas dentro de un proyecto.} realizado con la herramienta {\em Planner}\footnote{Herramienta software libre para realizar diagramas de Gantt}, en el que se puede comprobar los plazos utilizados para las fases del desarrollo del proyecto.
\newpage

\figura{gantt1.PNG}{scale=0.6, angle=90}{Diagrama de Gannt. Desarrollo del
  proyecto 1/2}{gannt}{H}

\figura{gantt2.PNG}{scale=0.5, angle=90}{Diagrama de Gannt. Desarrollo del
  proyecto 1/2}{gannt}{H}



\chapter{Descripción general del proyecto}
Este proyecto tiene la condición de {\em Software Libre}, por lo que en caso de necesitar ser ampliado, cualquier persona podria hacerlo. El proyecto es una aplicación nueva, no es continuación de otro proyecto.

\section{Descripción}

El proyecto consiste en una aplicación web, con distintos perfiles de usuario, en la que se llevará a cabo la configuración de la planificación docente de las distintas titulaciones de grado de la ESI. Como se ha dicho existirán varios perfiles de usuario, cada uno tendrá un cometido.

\section{Perfiles de usuario}

A continuación se expondrán los diferentes perfiles detallando a que funcionalidad tendrá acceso cada uno.

\subsection{Perfil Administrador}

El administrador solo tendrá acceso a la gestión de usuarios y a la creación de copias de seguridad de la base de datos. Por tanto el administrador podra crear nuevos usuarios con los perfiles que considere necesarios.

\subsection{Perfil Planificador}
Es el perfil que tiene acceso a más funcionalidades del sistema, llevará a cabo la gestión de titulaciones, asignaturas, planificación docente, calendario, aulas y horarios, realizando la configuración de todo. Es el perfil principal de la aplicación.

\subsection{Perfil Profesor}
Únicamente tendrá acceso a la visualización de la planificación docente de una titulación concreta.

\subsection{Perfil Alumno}
El alumno solo tendrá acceso a la visualización de los horarios. Podrá configurar un horario con las asignaturas y grupos pertenecientes a su titulación, generando un horario únicamente con las asignaturas que el quiera consultar.

\section{Interfaz de usuario}
La interfaz será algo simple, visualizada en un navegador web, con un menú principal en el que se tendrá acceso a las diferentes funcionalidades, estando ocultas las que no pertenezcan al perfil del usuario.

\section{Software}
Al ser una aplicación web, ésta será multiplataforma, pudiendo funcionar sobre cualquier navegador actual, ya que cumple los estándares de la W3C\footnote{World Wide Web Consortium. Es una comunidad internacional dedicada a desarrollar estándares web.}.\\

Como lenguaje de servidor la aplicación utiliza PHP, se toma la decisión de utilizarlo por la amplia documentación que hay disponible, además de la multitud de librerías que existen para simplificar su utilización. Además se ha utilizado el framework MVC {\em CodeIgniter}, que simplifica muchas tareas que de implementarlas únicamente con PHP sin la ayuda de ninguna librería se harían muy tediosas.\\

Para las vistas se ha utilizado XHTML y CSS, por su facilidad para estructurar los documentos y darles un estilo adecuado.\\

En la parte de los datos se ha usado {\em MySQL} como SGBD, utilizando {\em Doctrine} como un ORM para abstraer el uso de la base de datos dentro de la aplicación.

\chapter{Análisis}

\section{Metodología de desarrollo}
Para la realización del proyecto y su documentación se ha utilizado el {\em Rational Unified Process (RUP)}, junto con el {\em Lenguaje Unificado de Modelado (UML)}. Se ha elegido este sistema ya que es la metodología estándar más utilizada, además de ser un grupo de metodologías que se adaptan muy bien a las necesidades de un producto.

\section{Especificación de requisitos del sistema}
A continuación se enumeran los requisitos funcionales que se consideran fundamentales para el sistema. Éstos serán detallados utilizando casos de uso, describiendo tanto su escenario principal como sus posibles flujos alternativos. Además se detallará cada caso de uso con su diagrama de secuencia correspondiente.

\subsection{Gestión de usuarios}
\begin{figure}[H]
\label{gestion-usuarios}
\begin{center}
\includegraphics[scale=0.5]{./gestionusuarios.png}
\end{center}
\caption{Diagrama de casos de uso de la gestión de usuarios}
\end{figure}

\subsubsection*{Caso de uso: Añadir usuario}
\label{add_usuario}
\begin{itemize}
\item{\bf Descripción:} Caso de uso para la creación de un usuario.
\item{\bf Actores:} Administrador
\item{\bf Precondiciones:} El administrador se ha identificado correctamente en el sistema.
\item{\bf Postcondiciones:} Se crea un usuario con el perfil correspondiente.
\item{\bf Escenario principal:}
\begin{enumerate}
\item El administrador introduce los datos del usuario y el nivel de privilegios.
\item El sistema valida que los datos son correctos y no hay ningún usuario con el mismo email y DNI
\item El sistema crea el usuario y envía por correo el password al usuario.
\end{enumerate}
\item{\bf Escenarios alternativos:}
\begin{itemize}
\item[2.a.] Alguno de los datos no es correcto.
\begin{enumerate}
\item El sistema indica el error y el caso de uso vuelve al paso anterior.
\end{enumerate}
\item[2.b.] Ya existe algún usuario con el mismo email o DNI.
\begin{enumerate}
\item El sistema indica el error y el caso de uso vuelve al paso anterior.
\end{enumerate}
\item[*.a.] En cualquier momento el administrador decide cancelar el proceso.
\begin{enumerate}
\item El caso de uso finaliza.
\end{enumerate}
\end{itemize}
\end{itemize}

\subsubsection*{Caso de uso: Registro de alumno}
\label{add_alumno}
\begin{itemize}
\item{\bf Descripción:} Caso de uso para la creación de un alumno.
\item{\bf Actores:} Alumno.
\item{\bf Precondiciones:} No hay ningún usuario identificado en el sistema.
\item{\bf Postcondiciones:} Se crea un usuario con el perfil de alumno.
\item{\bf Escenario principal:}
\begin{enumerate}
\item El alumno introduce sus datos, DNI, titulación y email.
\item El sistema valida que los datos son correctos y no hay ningún usuario con el mismo email y DNI.
\item El sistema crea el usuario y envía por correo el password al usuario
\end{enumerate}
\item{\bf Escenarios alternativos:}
\begin{itemize}
\item[2.a.] Alguno de los datos no es correcto.
\begin{enumerate}
\item El sistema indica el error y el caso de uso vuelve al paso anterior.
\end{enumerate}
\item[2.b.] Ya existe algún usuario con el mismo email o DNI.
\begin{enumerate}
\item El sistema indica el error y el caso de uso vuelve al paso anterior.
\end{enumerate}
\item[*.a.] En cualquier momento el alumno decide cancelar el proceso.
\begin{enumerate}
\item El caso de uso finaliza.
\end{enumerate}
\end{itemize}
\end{itemize}

\subsubsection*{Caso de uso: Editar usuario}
\label{editar_usuario}
\begin{itemize}
\item{\bf Descripción:} Caso de uso para la edición de un usuario.
\item{\bf Actores:} Usuario.
\item{\bf Precondiciones:} El usuario que se intenta editar coincide con el identificado en el sistema o bien el usuario identificado es un administrador.
\item{\bf Postcondiciones:} Se actualizan los datos del usuario.
\item{\bf Escenario principal:}
\begin{enumerate}
\item El sistema muestra los datos actuales del usuario.
\item El usuario modifica sus datos, DNI, titulación y email.
\item El sistema valida que los datos son correctos y no hay ningún usuario con el mismo email y DNI.
\item El sistema modifica el usuario.
\end{enumerate}
\item{\bf Escenarios alternativos:}
\begin{itemize}
\item[2.a.] Alguno de los datos no es correcto.
\begin{enumerate}
\item El sistema indica el error y el caso de uso vuelve al paso anterior.
\end{enumerate}
\item[2.b.] Ya existe algún usuario con el mismo email o DNI.
\begin{enumerate}
\item El sistema indica el error y el caso de uso vuelve al paso anterior.
\end{enumerate}
\item[*.a.] En cualquier momento el usuario decide cancelar el proceso.
\begin{enumerate}
\item El caso de uso finaliza.
\end{enumerate}
\end{itemize}
\end{itemize}


\subsection{Gestión de titulaciones}

\begin{figure}[H] 
  \label{gestion-titulaciones} 
	\begin{center}
    \includegraphics[scale=0.5]{./gestiontitulaciones.png}
  \end{center}
\caption{Diagrama de casos de uso de la gestión de titulaciones}
\end{figure}
\subsubsection*{Caso de uso: Seleccionar titulación}
\label{select_titulacion}
\begin{itemize}
\item{\bf Descripción:} Caso de uso abstracto incluído en otros casos de uso para seleccionar una titulación de una lista de disponibles.
\item{\bf Actores:} Planificador.
\item{\bf Precondiciones:} El usuario identificado en el sistema es un planificador.
\item{\bf Postcondiciones:} Se selecciona una titulación para su uso en otra finalidad.
\item{\bf Escenario principal:}
\begin{enumerate}
\item El sistema muestra un listado de las titulaciones disponibles.
\item El usuario selecciona la titulación deseada.
\end{enumerate}
\item{\bf Escenarios alternativos:}
\begin{itemize}
\item[1.a.] No hay ninguna titulación registrada.
\begin{enumerate}
\item El sistema indica el error y el caso de uso finaliza.
\end{enumerate}
\end{itemize}
\end{itemize}



\subsubsection*{Caso de uso: Registrar titulación}
\begin{itemize}
\item{\bf Descripción:} Registra una nueva titulación en el sistema.
\item{\bf Actores:} Planificador.
\item{\bf Precondiciones:} El usuario identificado en el sistema es un planificador.
\item{\bf Postcondiciones:} La titulación queda registrada.
\item{\bf Escenario principal:}
\begin{enumerate}
\item El planificador introduce el código, el nombre y todos los demás datos de la titulación.
\item El sistema comprueba que los datos cumplen el formato.
\item El sistema confirma el alta de la titulación mostrando un mensaje.
\end{enumerate}
\item{\bf Escenarios alternativos:}
	\begin{itemize}
	\item[2.a.] Alguno de los datos introducidos tiene un formato incorrecto.
		\begin{enumerate}
		\item El sistema lo indica mostrando un mensaje de error y se vuelve al paso anterior.
		\end{enumerate}
	\item[2.b.] Falta algún campo obligatorio.
		\begin{enumerate}
		\item El sistema lo indica mostrando un mensaje de error y se vuelve al paso anterior.
		\end{enumerate}
	\item[2.c.] Ya existe alguna titulación con ese código o nombre.
		\begin{enumerate}
		\item El sistema indica el error y se vuelve al paso anterior.
		\end{enumerate}
	\item[*a.] El planificador decide cancelar el registro en cualquier momento, el caso de uso termina. 
	\end{itemize}
\end{itemize}



\subsubsection*{Caso de uso: Editar titulación}
\begin{itemize}
\item{\bf Descripción:} Edita una titulación existente en el sistema modificando sus datos.
\item{\bf Actores:} Planificador.
\item{\bf Precondiciones:} El usuario identificado en el sistema es un planificador.
\item{\bf Postcondiciones:} La titulacion queda modificada en el sistema.
\item{\bf Escenario principal:}
\begin{enumerate}
	\item Se realiza el caso de uso {\em\hyperref[select_titulacion]{Seleccionar titulación}}.
	\item El sistema muestra sus datos actuales, permitiendo su edición.
	\item El planificador modifica los datos.
	\item El sistema comprueba que todos los datos son correctos.
	\item El sistema muestra un mensaje indicando que la edición se ha completado.
\end{enumerate}
\item{\bf Escenarios alternativos:}
	\begin{itemize}
	\item[4.a.] Alguno de los datos tiene un formato incorrecto.
		\begin{enumerate}
		\item El sistema muestra un mensaje de error indicándolo, a continuación se vuelve al paso anterior.
		\end{enumerate}
	\item[4.b.] Falta algún campo obligatorio por rellenar.
		\begin{enumerate}
		\item El sistema muestra un mensaje de error indicándolo, a continuación se vuelve al paso anterior.
		\end{enumerate}
	\item[4.c.] Ya existe alguna titulación con el nombre o código introducidos.
		\begin{enumerate}
		\item El sitema indica el error y se vuelve al paso anterior.
		\end{enumerate}
	\item[*a.] En cualquier momento el planificador decide cancelar la edición, el caso de uso se da por terminado.
	\end{itemize}
\end{itemize}



\subsubsection*{Caso de uso: Borrar titulación}
% Tomar este CU como ejemplo
\begin{itemize}
\item{\bf Descripción:} Borra una titulación del sistema.
\item{\bf Actores:} Planificador.
\item{\bf Precondiciones:} La titulación existe en el sistema. El usuario identificado en el sistema es un planificador.
\item{\bf Postcondiciones:} La titulación queda eliminada del sistema.
\item{\bf Escenario principal:}
	\begin{enumerate}
	\item Se realiza el caso de uso  {\em\hyperref[select_titulacion]{Seleccionar titulación}}.
	\item El sistema muestra un diálogo de confirmación.
	\item El planificador confirma que quiere eliminar la titulación del sistema.
	\item El sistema elimina la titulación.
	\item El sistema muestra un mensaje confirmando que se ha eliminado la titulación.
	\end{enumerate}
\item{\bf Escenarios alternativos:}
	\begin{itemize}
	\item[3.a.] El planificador selecciona que no desea eliminar la titulación.
		\begin{enumerate}
		\item El caso de uso se reinicia.
		\end{enumerate}
	\item[*a.] En cualquier momento el planificador decide cancelar la eliminación.
		\begin{enumerate}
		\item El caso de uso se termina.
		\end{enumerate}
	\end{itemize}
\end{itemize}



\subsubsection*{Caso de uso: Ver detalles de titulación}
\begin{itemize}
\item{\bf Descripción:} Muestra los datos de una titulación en detalle, así como sus asignaturas.
\item{\bf Actores:} Planificador o profesor.
\item{\bf Precondiciones:} La titulación existe en el sistema. El usuario identificado en el sistema es un planificador o un profesor.
\item{\bf Postcondiciones:} Los datos de la titulación se muestran por pantalla.
\item{\bf Escenario principal:}
	\begin{enumerate}
	\item Se realiza el caso de uso {\em\hyperref[select_titulacion]{Seleccionar titulación}}.
	\item El sistema muestra los datos de la titulación y un listado de sus asignaturas si las tiene, junto con su planificación docente.
	\end{enumerate}
\item{\bf Escenarios alternativos:}
	\begin{itemize}
	\item[*a.] En cualquier momento el usuario decide cancelar el proceso, el caso de uso se termina.
	\end{itemize}
\end{itemize}





\subsubsection*{Caso de uso: Importar titulación}
\begin{itemize}
\item{\bf Descripción:} Caso de uso para importar titulaciones de forma masiva desde un archivo csv con un formato concreto.
\item{\bf Actores:} Planificador.
\item{\bf Precondiciones:} El archivo tiene el formato correcto. El usuario identificado en el sistema es un planificador.
\item{\bf Postcondiciones:} Se crean las titulaciones indicadas en el archivo.
\item{\bf Escenario principal:}
	\begin{enumerate}
	\item El planificador selecciona el archivo.
	\item El sistema comprueba que cada línea tenga el formato correcto.
	\item El sistema crea una titulación por cada línea con los datos indicados en el archivo.
	\end{enumerate}
\item{\bf Escenarios alternativos:}
	\begin{itemize}
		\item[2.a.] Alguna línea no cumple el formato
		\begin{enumerate}
			\item El sistema indica el error y el caso de uso finaliza.
		\end{enumerate}
		\item[2.b.] Ya existe una titulación creada con el mismo identificador.
		\begin{enumerate}
			\item El sistema lo indica y el caso de uso finaliza.
		\end{enumerate}
	\end{itemize}
\end{itemize}



\subsubsection{Gestión de asignaturas}
% Aquí va el diagrama de la gestión de asignaturas
\begin{figure}[H] 
  \label{gestion-asignaturas} 
	\begin{center}
    \includegraphics[scale=0.5]{./gestionasignaturas.png}
  \end{center}
\caption{Diagrama de casos de uso de la gestión de asignaturas}
\end{figure}
\subsubsection*{Caso de uso: Seleccionar asignatura}
\label{select_asignatura}
\begin{itemize}
\item{\bf Descripción:} Caso de uso abstracto incluído por otros casos de uso para seleccionar una asignatura del listado de las que tiene disponibles una titulación concreta.
\item{\bf Actores:} Usuario.
\item{\bf Precondiciones:} Ninguna.
\item{\bf Postcondiciones:} Queda seleccionada una asignatura para algún fin concreto de otro caso de uso.
\item{\bf Escenario principal:}
	\begin{enumerate}
	\item Se realiza el caso de uso {\em\hyperref[select_titulacion]{Seleccionar titulación}}.
	\item El sistema muestra un listado de las asignaturas disponibles asociadas a la titulación seleccionada.
	\item El usuario selecciona una asignatura de la lista.
	\end{enumerate}
\item{\bf Escenarios alternativos:}
	\begin{itemize}
		\item[2.a.] No hay ninguna asignatura registrada en esa titulación.
		\begin{enumerate}
			\item El sistema indica el error y el caso de uso finaliza.
		\end{enumerate}
	\end{itemize}
\end{itemize}



\subsubsection*{Caso de uso: Registrar asignatura}
\begin{itemize}
\item{\bf Descripción:} Se da de alta una nueva asignatura en el sistema.
\item{\bf Actores:} Planificador.
\item{\bf Precondiciones:} Existe alguna titulación con la que asociar la asignatura. El usuario identificado en el sistema es un planificador.
\item{\bf Postcondiciones:} La asignatura queda registrada en el sistema.
\item{\bf Escenario principal:}
	\begin{enumerate}
 	\item Se realiza el caso de uso {\em\hyperref[select_titulacion]{Seleccionar titulación}}.
	\item El sistema muestra un formulario para introducir los datos.
	\item El planificador introduce el código, el nombre y todos los demás datos de la asignatura.
	\item El sistema comprueba que todos los datos cumplen el formato requerido.
	\item El sistema registra la asignatura y muestra un mensaje confirmándolo.
	\end{enumerate}
\item{\bf Escenarios alternativos:}
	\begin{itemize}
	\item[3.a.] El planificador selecciona que desea tomar los datos de otra asignatura.
		\begin{enumerate}
		\item Se realiza el caso de uso {\em \hyperref[duplicar_asignatura]{Duplicar asignatura}}.
		\end{enumerate}
	\item[4.a.] Alguno de los datos no cumple el formato correcto.
		\begin{enumerate}
		\item El sistema indica el error y se vuelve al paso anterior.
		\end{enumerate}
	\item[4.b.] Falta por rellenar algún campo obligatorio.
		\begin{enumerate}
		\item El sistema indica el error y se vuelve al paso anterior.
		\end{enumerate}
	\item[4.c.] Ya existe alguna asignatura con ese código o nombre.
		\begin{enumerate}
		\item El sistema indica el error y se vuelve al paso anterior.
		\end{enumerate}
	\item[*a.] En cualquier momento el planificador decide cancelar el proceso.
		\begin{enumerate}
		\item El caso de uso se cancela.
		\end{enumerate}
	\end{itemize}
\end{itemize}



\subsubsection*{Caso de uso: Editar Asignatura}
\begin{itemize}
\item{\bf Descripción:} Se modifican los datos de una asignatura existente en el sistema.
\item{\bf Actores:} Planificador.
\item{\bf Precondiciones:} El usuario identificado en el sistema es un planificador.
\item{\bf Postcondiciones:} La asignatura queda modificada en el sistema.
\item{\bf Escenario principal:}
	\begin{enumerate}
	\item Se realiza el caso de uso {\em \hyperref[select_asignatura]{Seleccionar asignatura}}.
	\item El sistema muestra los datos de la asignatura en un formato editable.
	\item El planificador hace las modificaciones que considere necesarias.
	\item El sistema comprueba que los datos modificados cumplen el formato requerido.
	\item El sistema guarda la asignatura y muestra un mensaje confirmándolo.
	\end{enumerate}
\item{\bf Escenarios alternativos:}
	\begin{itemize}
	\item[4.a.] Alguno de los datos no cumple el formato correcto.
		\begin{enumerate}
		\item El sistema indica el error y se vuelve al paso anterior.
		\end{enumerate}
	\item[4.b.] Falta por rellenar algún campo obligatorio.
		\begin{enumerate}
		\item El sistema indica el error y se vuelve al paso anterior.
		\end{enumerate}
	\item[4.c.] Ya existe alguna asignatura con ese código o nombre.
		\begin{enumerate}
		\item El sistema indica el error y se vuelve al paso anterior.
		\end{enumerate}
	\item[*a.] En cualquier momento el planificador decide cancelar el proceso.
		\begin{enumerate}
		\item El caso de uso se cancela.
		\end{enumerate}
	\end{itemize}
\end{itemize}



\subsubsection*{Caso de uso: Borrar asignatura}
\begin{itemize}
\item{\bf Descripción:} Se borra una asingatura del sistema.
\item{\bf Actores:} Planificador.
\item{\bf Precondiciones:} El usuario identificado en el sistema es un planificador.
\item{\bf Postcondiciones:} La asignatura queda eliminada del sistema.
\item{\bf Escenario principal:}
	\begin{enumerate}
	\item Se realiza el caso de uso {\em \hyperref[select_asignatura]{Seleccionar asignatura}}.
	\item El sistema muestra un diálogo de confirmación.
	\item El planificador confirma que desea borrar la asignatura.
	\item El sistema borra la asignatura y muestra un mensaje confirmándolo.
	\end{enumerate}
\item{\bf Escenarios alternativos:}
	\begin{itemize}
	\item[3.a.] El planificador selecciona que no desea eliminar la asignatura.
		\begin{enumerate}
		\item El caso de uso se reinicia.
		\end{enumerate}
	\item[*a.] En cualquier momento el planificador decide cancelar la eliminación.
		\begin{enumerate}
		\item El caso de uso se termina.
		\end{enumerate}
	\end{itemize}
\end{itemize}



\subsubsection*{Caso de uso: Consultar asignatura}
\begin{itemize}
\item{\bf Descripción:} Muestra los datos en detalle de una asignatura.
\item{\bf Actores:} Usuario.
\item{\bf Precondiciones:} El usuario identificado en el sistema es un planificador.
\item{\bf Postcondiciones:} Se muestran los datos de la asignatura por pantalla.
\item{\bf Escenario principal:} 
	\begin{enumerate}
	\item Se realiza el caso de uso {\em \hyperref[select_asignatura]{Seleccionar asignatura}}.
	\item El sistema muestra la información relacionada con la asignatura.
	\end{enumerate}
\item{\bf Escenarios alternativos:}
	\begin{itemize}
	\item[*a.] En cualquier momento el planificador decide cancelar el proceso.
		\begin{enumerate}
		\item El caso de uso se termina.
		\end{enumerate}
	\end{itemize}
\end{itemize}


\subsubsection*{Caso de uso: Importar asignatura desde archivo}
\begin{itemize}
\item{\bf Descripción:} Caso de uso para importar asignaturas de forma masiva desde un archivo csv con un formato concreto.
\item{\bf Actores:} planificador
\item{\bf Precondiciones:} El archivo tiene el formato correcto.
\item{\bf Postcondiciones:} Se crean las asignaturas indicadas en el archivo.
\item{\bf Escenario principal:}
	\begin{enumerate}
	\item El planificador selecciona el archivo.
	\item El sistema comprueba que cada línea tenga el formato correcto.
	\item El sistema crea una asignatura por cada línea con los datos indicados en el archivo.
	\end{enumerate}
\item{\bf Escenarios alternativos:}
	\begin{itemize}
		\item[2.a.] Alguna línea no cumple el formato
		\begin{enumerate}
			\item El sistema indica el error y el caso de uso finaliza.
		\end{enumerate}
		\item[2.b.] Ya existe una asignatura creada con el mismo identificador.
		\begin{enumerate}
			\item El sistema lo indica y el caso de uso finaliza.
		\end{enumerate}
	\end{itemize}
\end{itemize}


\subsubsection{Gestión de cursos}
\begin{figure}[H] 
  \label{gestion-cursos} 
	\begin{center}
    \includegraphics[scale=0.5]{./gestioncursos.png}
  \end{center}
\caption{Diagrama de casos de uso de la gestión de cursos}
\end{figure}

\subsubsection*{Caso de uso: Seleccionar curso}
\begin{itemize}
\item{\bf Descripción:} Caso de uso abstracto que es incluído en otros casos de uso.
\item{\bf Actores:} Usuario.
\item{\bf Precondiciones:} Ninguna.
\item{\bf Postcondiciones:} Queda seleccionado un curso para su uso con algún fin.
\item{\bf Escenario principal:}
	\begin{enumerate}
	\item El sistema muestra un listado con los cursos disponibles.
        \item El usuario selecciona un curso.
	\end{enumerate}
\item{\bf Escenarios alternativos:}
	\begin{itemize}
	\item[1.a.]No hay ningún curso registrado en el sistema.
	  \begin{enumerate}
	  \item El sistema indica el error y el caso de uso finaliza.
	  \end{enumerate}
	\end{itemize}
\end{itemize}




\subsubsection*{Caso de uso: Registrar curso}
\begin{itemize}
\item{\bf Descripción:} Se da de alta en el sistema la configuración de un nuevo curso.
\item{\bf Actores:} Planificador.
\item{\bf Precondiciones:} El usuario identificado en el sistema es un planificador.
\item{\bf Postcondiciones:} Queda registrado en el sistema el curso.
\item{\bf Escenario principal:}
  \begin{enumerate}
  \item El planificador introduce los datos de configuración del curso, incluyendo fecha de inicio y final de curso.
  \item El sistema comprueba que los datos sean correctos.
  \item El sistema informa de que el curso ha sido registrado con éxito.
  \end{enumerate}
\item{\bf Escenarios alternativos:}
  \begin{itemize}
  \item[2.a.] Alguno de los datos introducidos no cumple el formato correcto.
    \begin{enumerate}
    \item El sistema indica el error y se vuelve al paso anterior.
    \end{enumerate}
  \item[2.b.] Ya existe un curso registrado que empieza o termina en el mismo año que el introducido.
    \begin{enumerate}
    \item El sistema indica el error y se vuelve al paso anterior.
    \end{enumerate}
  \item[2.c.] Los años introducidos de inicio y fin del curso no son consecutivos.
    \begin{enumerate}
    \item El sistema indica el error y se vuelve al paso anterior.
    \end{enumerate}
  \item[2.d.] Alguna de las fechas de exámenes introducidas no están comprendidas entre la duración del curso.
    \begin{enumerate}
    \item El sistema indica el error y se vuelve al paso anterior.
    \end{enumerate}
  \item[*a.] En cualquier momento el planificador decide cancelar el proceso.
    \begin{enumerate}
    \item El caso de uso finaliza.
    \end{enumerate}
  \end{itemize}
\end{itemize}



\subsubsection*{Caso de uso: Editar curso}
\begin{itemize}
\item{\bf Descripción:} Se edita en el sistema la configuración de un curso.
\item{\bf Actores:} Planificador.
\item{\bf Precondiciones:} El usuario identificado en el sistema es un planificador.
\item{\bf Postcondiciones:} Quedan registradas en el sistema las modificaciones realizadas.
\item{\bf Escenario principal:}
  \begin{enumerate}
  \item Se realiza el caso de uso {\em \hyperref[select_curso]{Seleccionar curso}}.
  \item El sistema muestra los datos en forma editable.
  \item El planificador modifica los datos de configuración del curso que considere necesarios.
  \item El sistema comprueba que los datos sean correctos.
  \item El sistema informa de que el curso ha sido registrado con éxito.
  \end{enumerate}
\item{\bf Escenarios alternativos:}
  \begin{itemize}
  \item[4.a.] Alguno de los datos introducidos no cumple el formato correcto.
    \begin{enumerate}
    \item El sistema indica el error y se vuelve al paso anterior.
    \end{enumerate}
  \item[4.b.] Ya existe un curso registrado que empieza o termina en el mismo año que el introducido.
    \begin{enumerate}
    \item El sistema indica el error y se vuelve al paso anterior.
    \end{enumerate}
  \item[4.c.] Los años introducidos de inicio y fin del curso no son consecutivos.
    \begin{enumerate}
    \item El sistema indica el error y se vuelve al paso anterior.
    \end{enumerate}
  \item[4.d.] Alguna de las fechas de exámenes introducidas no están comprendidas entre la duración del curso.
    \begin{enumerate}
    \item El sistema indica el error y se vuelve al paso anterior.
    \end{enumerate}
  \item[*a.] En cualquier momento el planificador decide cancelar el proceso.
    \begin{enumerate}
    \item El caso de uso finaliza.
    \end{enumerate}
  \end{itemize}
\end{itemize}



\subsubsection*{Caso de uso: Consultar curso}
\begin{itemize}
\item{\bf Descripción:} Se consulta la configuración de un curso mostrandola por pantalla
\item{\bf Actores:} Planificador.
\item{\bf Precondiciones:} El usuario identificado en el sistema es un planificador.
\item{\bf Postcondiciones:} Se muestra por pantalla la configuración del curso seleccionado.
\item{\bf Escenario principal:}
  \begin{enumerate}
  \item Se realiza el caso de uso {\em \hyperref[select_curso]{Seleccionar curso}}.
  \item El sistema muestra los datos del curso.
  \end{enumerate}
\item{\bf Escenarios alternativos:}
  \begin{itemize}
  \item[*a.] En cualquier momento el planificador decide cancelar el proceso.
    \begin{enumerate}
    \item El caso de uso finaliza.
    \end{enumerate}
  \end{itemize}
\end{itemize}



\subsubsection*{Caso de uso: Borrar curso}
\begin{itemize}
\item{\bf Descripción:} Se elimina un curso registrado en el sistema.
\item{\bf Actores:} Planificador
\item{\bf Precondiciones:} El usuario identificado en el sistema es un planificador.
\item{\bf Postcondiciones:} El curso seleccionado queda eliminado del sistema
\item{\bf Escenario principal:}
	\begin{enumerate}
	\item Se realiza el caso de uso {\em \hyperref[select_curso]{Seleccionar curso}}.
	\item El sistema muestra un diálogo pidiendo la confirmación del borrado.
	\item El planificador selecciona que desea confirmar el borrado.
	\item El sistema muestra un mensaje confirmando el éxito en la operación.
	\end{enumerate}
\item{\bf Escenarios alternativos:}
	\begin{itemize}
		\item[3.a.] El planificador selecciona que no desea confirmar el borrado.
		\begin{enumerate}
			\item El caso de uso se reinicia.
		\end{enumerate}
		\item[*a.] En cualquier momento el planificador cancela el proceso.
		\begin{enumerate}
			\item El caso de uso finaliza.
		\end{enumerate}
	\end{itemize}
\end{itemize}


\subsubsection{Gestión de planificación docente}
\begin{figure}[H] 
  \label{gestion-planificacion} 
	\begin{center}
    \includegraphics[scale=0.5]{./gestionplanificacion.png}
  \end{center}
\caption{Diagrama de casos de uso de la gestión de la planificación docente}
\end{figure}
\subsubsection*{Caso de uso: Añadir plan docente}
\begin{itemize}
\item{\bf Descripción:} Se añaden los detalles del plan docente para una asignatura en un curso determinado.
\item{\bf Actores:} Planificador.
\item{\bf Precondiciones:} El usuario identificado en el sistema es un planificador.
\item{\bf Postcondiciones:} El plan docente queda registrado en el sistema, asociado a una asignatura y un curso determinado.
\item{\bf Escenario principal:}
	\begin{enumerate}
	\item Se realiza el caso de uso {\em \hyperref[select_asignatura]{Seleccionar asignatura}}.
	\item Se realiza el caso de uso {\em \hyperref[select_curso]{Seleccionar curso}}.
	\item El sistema comprueba que no exista ya un plan docente asociado a ese curso.
	\item El usuario introduce los datos del plan docente.
	\item El sistema comprueba que los datos cumplen el formato requerido.
	\item El sistema guarda la carga de trabajo y muestra un mensaje confirmándolo.
	\end{enumerate}
\item{\bf Escenarios alternativos:}
	\begin{itemize}
	\item[3.a.] Ya existe una carga de trabajo establecida para el curso seleccionado.
		\begin{enumerate}
		\item El sistema indica el error y el caso de uso vuelve al paso anterior.
		\end{enumerate}
	\item[4.a.] El usuario indica que quiere tomar los datos de otra carga de un curso anterior.
		\begin{enumerate}
		\item Se realiza el caso de uso {\em \hyperref[duplicar_carga]{Duplicar carga de trabajo}}
		\end{enumerate}
	\item[5.a.] Alguno de los datos introducidos no cumple el formato correcto.
		\begin{enumerate}
		\item El sistema indica el error y se vuelve al paso anterior.
		\end{enumerate}
	\item[5.b.] Alguno de los campos obligatorios no ha sido rellenado.
		\begin{enumerate}
		\item El sistema indica el error y se vuelve al paso anterior.
		\end{enumerate}	
	\item[*a.] En cualquier momento el planificador decide cancelar el proceso.
		\begin{enumerate}
		\item El caso de uso se termina.
		\end{enumerate}
	\end{itemize}
\end{itemize}



\subsubsection*{Caso de uso: Editar plan docente}
\begin{itemize}
\item{\bf Descripción:} Se edita una carga de trabajo existente para un curso determinado.
\item{\bf Actores:} Planificador.
\item{\bf Precondiciones:} El usuario identificado en el sistema es un planificador.
\item{\bf Postcondiciones:} La carga de trabajo queda modificada en el sistema.
\item{\bf Pasos:}
	\begin{enumerate}
	\item Se realiza el caso de uso {\em \hyperref[select_asignatura]{Seleccionar asignatura}}.
	\item Se realiza el caso de uso {\em \hyperref[select_curso]{Seleccionar curso}}.
	\item El sistema comprueba que exista una carga asociada a ese curso.
	\item El sistema muestra los datos de la carga en un formato editable.
	\item El usuario modifica los datos.
	\item El sistema comprueba que los datos cumplen el formato requerido.
	\item El sistema guarda los cambios y muestra un mensaje confirmándolo.
	\end{enumerate}
\item{\bf Escenarios alternativos:}
	\begin{itemize}
	\item[3.a.] No existe una carga de trabajo establecida para el curso seleccionado.
		\begin{enumerate}
		\item El sistema indica el error y el caso de uso vuelve al paso anterior.
		\end{enumerate}
	\item[5.a.] Alguno de los datos introducidos no cumple el formato correcto.
		\begin{enumerate}
		\item El sistema indica el error y se vuelve al paso anterior.
		\end{enumerate}
	\item[5.b.] Alguno de los campos obligatorios no ha sido rellenado.
		\begin{enumerate}
		\item El sistema indica el error y se vuelve al paso anterior.
		\end{enumerate}	
	\item[*a.] En cualquier momento el planificador decide cancelar el proceso.
		\begin{enumerate}
		\item El caso de uso se termina.
		\end{enumerate}
	\end{itemize}
\end{itemize}



\subsubsection*{Caso de uso: Borrar plan docente}
\begin{itemize}
\item{\bf Descripción:} Se borra una carga de trabajo existente en el sistema asociada a un curso.
\item{\bf Actores:} Planificador.
\item{\bf Precondiciones:} El usuario identificado en el sistema es un planificador.
\item{\bf Postcondiciones:} La carga de trabajo asociada a la asignatura y curso seleccionados queda. eliminada del sistema.
\item{\bf Escenario principal:}
	\begin{enumerate}
	\item Se realiza el caso de uso {\em \hyperref[select_asignatura]{Seleccionar asignatura}}
	\item Se realiza el caso de uso {\em \hyperref[select_curso]{Seleccionar curso}}.
	\item El sistema comprueba que exista una carga asociada a ese curso.
	\item El sistema muestra un diálogo de confirmación.
	\item El usuario confirma que desea borrar la carga.
	\item El sistema muestra un mensaje confirmando la eliminación y borra la carga.
	\end{enumerate}
\item{\bf Escenarios alternativos:}
	\begin{itemize}
	\item[3.a.] No existe una carga de trabajo establecida para el curso seleccionado.
	  \begin{enumerate}
	  \item El sistema indica el error y el caso de uso vuelve al paso anterior.
	  \end{enumerate}
	\item[5.a.] El administrador selecciona que no desea eliminar la carga.
		\begin{enumerate}
		\item El caso de uso se reinicia.
		\end{enumerate}
	\item[*a.] En cualquier momento el administrador decide cancelar la eliminación.
		\begin{enumerate}
		\item El caso de uso se termina.
		\end{enumerate}
	\end{itemize}
\end{itemize}



\subsubsection*{Caso de uso: Consultar plan docente}
\begin{itemize}
\item{\bf Descripción:} Se consulta la carga de trabajo de una asignatura para un curso determinado.
\item{\bf Actores:} Planificador.
\item{\bf Precondiciones:} El usuario identificado en el sistema es un planificador.
\item{\bf Postcondiciones:} Se muestran los datos al usuario por pantalla.
\item{\bf Escenario principal:}
	\begin{enumerate}
	\item Se realiza el caso de uso {\em \hyperref[select_asignatura]{Seleccionar asignatura}}
	\item Se realiza el caso de uso {\em \hyperref[select_curso]{Seleccionar curso}}.
	\item El sistema comprueba que exista una carga asociada a ese curso.
	\item La carga existe y es mostrada al usuario.
	\end{enumerate}
\item{\bf Escenarios alternativos:}
	\begin{itemize}
	\item[3.a.]No existe ninguna carga asociada a ese curso.
		\begin{enumerate}
		\item El sistema muestra un mensaje informando del error y vuelve al paso anterior.
		\end{enumerate}
	\item[*a.]En cualquier momento el usuario decide cancelar el proceso.
		\begin{enumerate}
		\item El caso de uso se cancela.
		\end{enumerate}		
	\end{itemize}
\end{itemize}




\subsubsection*{Caso de uso: Importar plan docente desde archivo}
\begin{itemize}
\item{\bf Descripción:} Caso de uso para importar planes docentes de forma masiva desde un archivo csv con un formato concreto.
\item{\bf Actores:} Planificador.
\item{\bf Precondiciones:} El archivo tiene el formato correcto. El usuario identificado en el sistema es un planificador.
\item{\bf Postcondiciones:} Se crean los planes docentes indicados en el archivo.
\item{\bf Escenario principal:}
	\begin{enumerate}
	\item El administrador selecciona el archivo.
	\item El sistema comprueba que cada línea tenga el formato correcto.
	\item El sistema crea un plan docente por cada línea con los datos indicados en el archivo.
	\end{enumerate}
\item{\bf Escenarios alternativos:}
	\begin{itemize}
		\item[2.a.] Alguna línea no cumple el formato
		\begin{enumerate}
			\item El sistema indica el error y el caso de uso finaliza.
		\end{enumerate}
		\item[2.b.] Ya existe un plan docente creado para la asignatura indicada y para ese curso.
		\begin{enumerate}
			\item El sistema lo indica y el caso de uso finaliza.
		\end{enumerate}
	\end{itemize}
\end{itemize}


\subsubsection*{Caso de uso: Generación de informe de asignatura}
\begin{itemize}
\item{\bf Descripción:} Caso de uso para generar un informe de las horas asignadas a una asignatura.
\item{\bf Actores:} Planificador.
\item{\bf Precondiciones:} La asignatura tiene un plan docente creado y asignadas horas en los horarios. El usuario identificado en el sistema es un planificador.
\item{\bf Postcondiciones:} Se genera un informe en pdf permitiendo su descarga.
\item{\bf Escenario principal:}
	\begin{enumerate}
	\item Se realiza el caso de uso {\em \hyperref[select_curso]{Seleccionar curso}}.
	\item Se realiza el caso de uso {\em \hyperref[select_asignatura]{Seleccionar asignatura}}.
	\item El sistema comprueba el plan docente de la asignatura y las asignaciones en los horarios.
	\item El sistema muestra un desglose de las horas de cada actividad de la asignatura y de cada semana teniendo en cuenta los eventos del calendario, permitiendo la descarga del archivo.
	\end{enumerate}
\item{\bf Escenarios alternativos:}
	\begin{itemize}
		\item[*.a.] En cualquier momento se decide parar el proceso.
		\begin{enumerate}
			\item El caso de uso finaliza.
		\end{enumerate}
		\item[3.a] La asignatura no tiene un plan docente asignado.
		\begin{enumerate}
			\item El caso de uso finaliza.
		\end{enumerate}
	\end{itemize}
\end{itemize}

\subsubsection{Gestión del calendario}
\begin{figure}[H] 
  \label{gestion-calendario} 
	\begin{center}
    \includegraphics[scale=0.5]{./gestioncalendario.png}
  \end{center}
\caption{Diagrama de casos de uso de la gestión del calendario}
\end{figure}


\subsubsection*{Caso de uso: Añadir evento al calendario}
\begin{itemize}
\item{\bf Descripción:} Añade un evento de fechas al calendario, que puede ser una festividad o un período de vacaciones.
\item{\bf Actores:} Planificador.
\item{\bf Precondiciones:} El usuario identificado en el sistema es un planificador.
\item{\bf Postcondiciones:} El evento queda registrado para el calendario de un curso concreto.
\item{\bf Escenario principal:}
	\begin{enumerate}
	\item Se realiza el caso de uso {\em \hyperref[select_curso]{Seleccionar curso}}.
	\item El subdirector introduce los datos correspondientes al evento, que serían, el nombre del evento, el tipo de evento, si es una fecha concreta o un rango de ellas, y su fecha de inicio y finalización.
	\item El sistema comprueba que los datos son correctos y que no se solapen con otros eventos.
	\item El sistema indica que todo es correcto y muestra un mensaje confirmando el éxito de la operación.
	\end{enumerate}
\item{\bf Escenarios alternativos:}
	\begin{itemize}
		\item[3.a.] Alguno de los datos tiene un formato incorrecto o algún campo está en blanco.
		\begin{enumerate}
			\item El sistema lo indica y vuelve al paso anterior.
		\end{enumerate}
		\item[3.b.] Las fechas del evento se solapan con algún otro evento o fecha del curso.
		\begin{enumerate}
			\item El sistema lo indica y vuelve al paso anterior.
		\end{enumerate}
		\item[*a.] En cualquier momento el usuario decide cancelar el proceso.
		\begin{enumerate}
		\item El caso de uso finaliza.
		\end{enumerate}
	\end{itemize}
\end{itemize}



\subsubsection*{Caso de uso: Borrar evento del calendario}
\begin{itemize}
\item{\bf Descripción:} El usuario selecciona alguno de los eventos del calendario para eliminarlo y este queda eliminado del sistema.
\item{\bf Actores:} Planificador.
\item{\bf Precondiciones:} El usuario identificado en el sistema es un planificador.
\item{\bf Postcondiciones:} El evento queda eliminado del calendario del curso concreto.
\item{\bf Escenario principal:}
	\begin{enumerate}
	\item Se realiza el caso de uso {\em \hyperref[select_curso]{Seleccionar curso}}.
	\item El sistema muestra un listado de fechas de eventos.
	\item El subdirector selecciona la fecha que desea eliminar.
	\item El sistema muestra un mensaje confirmando que el evento ha sido eliminado.
	\end{enumerate}
\item{\bf Escenarios alternativos:}
	\begin{itemize}
		\item[2.a.] No hay ningún evento registrado en el calendario de ese curso.
		\begin{enumerate}
			\item El sistema lo indica y el caso de uso finaliza.
		\end{enumerate}
		\item[*a.] En cualquier momento el subdirector decide cancelar el proceso.
		\begin{enumerate}
		\item El caso de uso finaliza sin eliminar el evento.
		\end{enumerate}
	\end{itemize}
\end{itemize}



\subsubsection*{Caso de uso: Visualizar calendario completo}
\begin{itemize}
\item{\bf Descripción:} Se muestra un calendario completo con las fechas marcadas.
\item{\bf Actores:} Planificador.
\item{\bf Precondiciones:} El usuario identificado en el sistema es un planificador.
\item{\bf Postcondiciones:} Se muestra el calendario por pantalla.
\item{\bf Escenario principal:}
	\begin{enumerate}
	\item Se realiza el caso de uso {\em \hyperref[select_curso]{Seleccionar curso}}.
	\item El sistema muestra un calendario con las fechas de los eventos marcadas.
	\end{enumerate}
\item{\bf Escenarios alternativos:}
\end{itemize}


\subsubsection*{Caso de uso: Ver detalle de evento del calendario}
\begin{itemize}
\item{\bf Descripción:} Se muestran los datos detallados de un evento del calendario
\item{\bf Actores:} Planificador.
\item{\bf Precondiciones:} Hay eventos registrados en el sistema. El usuario identificado en el sistema es un planificador.
\item{\bf Postcondiciones:} Se muestran los datos.
\item{\bf Escenario principal:}
	\begin{enumerate}
	\item Se realiza el caso de uso {\em \hyperref[select_curso]{Seleccionar curso}}.
	\item El sistema muestra el calendario para el curso actual con los eventos creados sobre él.
	\item El usuario selecciona un evento
	\item El sistema muestra los datos del evento, título, razón, etc.
	\end{enumerate}
\item{\bf Escenarios alternativos:}
\end{itemize}




\subsubsection*{Caso de uso: Exportar calendario}
\begin{itemize}
\item{\bf Descripción:} Caso de uso para exportar el calendario de un curso a un formato externo como csv o una hoja de cálculo.
\item{\bf Actores:} Planificador.
\item{\bf Precondiciones:} El curso existe y tiene eventos creados. El usuario identificado en el sistema es un planificador.
\item{\bf Postcondiciones:} Se exporta un archivo con un formato adecuado.
\item{\bf Escenario principal:}
	\begin{enumerate}
	\item Se realiza el caso de uso {\em \hyperref[select_curso]{Seleccionar curso}}.
	\item Se muestra y permite descarga del archivo exportado.
	\end{enumerate}
\item{\bf Escenarios alternativos:}
	\begin{itemize}
		\item[*.a.] En cualquier momento el administrador decide cancelar el proceso.
		\begin{enumerate}
			\item Se finaliza el caso de uso.
		\end{enumerate}
	\end{itemize}
\end{itemize}


\subsubsection{Gestión de horarios}
\begin{figure}[H] 
  \label{gestion-horarios} 
	\begin{center}
    \includegraphics[scale=0.5]{./gestionhorarios.png}
  \end{center}
\caption{Diagrama de casos de uso de la gestión de horarios}
\end{figure}

\subsubsection*{Caso de uso: Seleccionar grupos de teoría}
\label{select_grupo}
\begin{itemize}
\item{\bf Descripción:} Se muestran los grupos de teoría de una titulación. Este caso de uso es incluido por más casos de uso.
\item{\bf Actores:} Planificador.
\item{\bf Precondiciones:} Hay titulaciones y cursos creados. El usuario identificado en el sistema es un planificador.
\item{\bf Postcondiciones:} Se muestra la información
\item{\bf Escenario principal:}
	\begin{enumerate}
	\item Se realiza el caso de uso {\em \hyperref[select_curso]{Seleccionar curso}}.
	\item Se realiza el caso de uso {\em \hyperref[select_titulacion]{Seleccionar titulación}}.
	\item El sistema muestra un listado con los cursos de la titulación y el número de grupos de cada curso.
	\item El administrador selecciona el grupo deseado.
	\end{enumerate}
\item{\bf Escenarios alternativos:}
\end{itemize}



\subsubsection*{Caso de uso: Ubicar slot de horario}
\label{guardar_slot}
\begin{itemize}
\item{\bf Descripción:} Caso de uso abstracto que es incluído por otros casos de uso. Se utiliza para ubicar en un horario un slot de una actividad de una asignatura.
\item{\bf Actores:} Planificador.
\item{\bf Precondiciones:} La asignatura existe y tiene asignadas horas en el plan docente para esa actividad. El usuario identificado en el sistema es un planificador.
\item{\bf Postcondiciones:} El slot de la actividad queda guardado en el horario.
\item{\bf Escenario principal:}
	\begin{enumerate}
	\item El administrador selecciona una asignatura y su actividad
	\item El sistema comprueba que tenga asignadas horas en el plan docente
	\item El administrador selecciona el aula en la que se guardará el slot
	\item El administrador selecciona el lugar que ocupará en el horario
	\item El sistema comprueba que el aula no esté ocupada en ese momento y que si el slot es de teoría que no se solape con otros slots.
	\item El sistema guarda el slot en el horario.
	\end{enumerate}
\item{\bf Escenarios alternativos:}
	\begin{itemize}
		\item[2.a.] La asignatura no tiene horas asignadas para esa actividad.
		\begin{enumerate}
			\item El sistema lo indica y se vuelve al paso anterior.
		\end{enumerate}
		\item[5.a.] El aula está ocupada en ese horario.
		\begin{enumerate}
			\item El sistema lo indica y se vuelve al paso 3.
		\end{enumerate}
		\item[5.b.] El slot es de teoría y se solapa con otros slots.
		\begin{enumerate}
			\item El sistema lo indica y se vuelve al paso anterior.
		\end{enumerate}
	\end{itemize}
\end{itemize}



\subsubsection*{Caso de uso: Editar horario tipo}
\begin{itemize}
\item{\bf Descripción:} Se muestra el horario de un grupo de teoría en un formato editable.
\item{\bf Actores:} Planificador.
\item{\bf Precondiciones:} Hay algún grupo creado para la titulación, semestre, curso y año académico. El usuario identificado en el sistema es un planificador.
\item{\bf Postcondiciones:} Se muestran las asignaturas disponibles, permitiendo su edición. Se muestran tanto los slots de teoría como los de las demás actividades
\item{\bf Escenario principal:}
	\begin{enumerate}
	\item Se realiza el caso de uso {\em \hyperref[select_grupo]{Seleccionar grupos de teoría}}.
	\item El administrador selecciona el curso y semestre deseado para editar su horario tipo.
	\item El sistema comprueba que exista el grupo y se que no exista un horario ya empezado.
	\item Se realiza el caso de uso {\em \hyperref[guardar_slot]{Ubicar slot de horario}}.
	\item El administrador realiza el paso anterior las veces que sean necesarias hasta completar el horario.
	\end{enumerate}
\item{\bf Escenarios alternativos:}
	\begin{itemize}
		\item[*.a.] En cualquier momento el administrador decide parar el proceso.
		\begin{enumerate}
			\item El sistema guarda los cambios hechos hasta ahora y el caso de uso finaliza.
		\end{enumerate}
		\item[2.a] No ha sido creado ningún grupo para ese curso.
		\begin{enumerate}
			\item El sistema indica el error y finaliza el caso de uso.
		\end{enumerate}
		\item[3.a] Ya hay un horario empezado para ese grupo y semestre
		\begin{enumerate}
			\item El sistema muestra el horario ya empezado ubicando los slots donde estaban en el antiguo horario y el caso de uso continua.
		\end{enumerate}
	\end{itemize}
\end{itemize}



\subsubsection*{Caso de uso: Editar horario de semana inicial (sólo teoría)}
\begin{itemize}
\item{\bf Descripción:} Se muestra para su edición el horario de un grupo de teoría en una semana en la que sólo se imparte teoría
\item{\bf Actores:} Planificador. 
\item{\bf Precondiciones:} Ya ha sido creado el grupo y editado el horario tipo correspondiente. El usuario identificado en el sistema es un planificador.
\item{\bf Postcondiciones:} Se muestra el horario con los slots de teoría del horario tipo ubicados en su sitio, salvo los que no sean posible por no ser día lectivo
\item{\bf Escenario principal:}
	\begin{enumerate}
	\item Se realiza el caso de uso {\em \hyperref[select_grupo]{Seleccionar grupo de teoría}}.
	\item El administrador selecciona el curso y semestre además de la semana que se editará.
	\item El sistema comprueba que exista el grupo y se que no exista un horario ya empezado.
	\item El sistema muestra los slots de teoría del horario tipo correspondiente, sacando del horario los que no sean posibles ubicar automáticamente por que el día sea no lectivo.
	\item Se realiza el caso de uso {\em \hyperref[guardar_slot]{Ubicar slot de horario}}.
	\item Se realiza el paso anterior las veces que se consideren necesarias hasta completar el horario.
	\end{enumerate}
\item{\bf Escenarios alternativos:}
	\begin{itemize}
		\item[*.a.] En cualquier momento el administrador decide parar el proceso.
		\begin{enumerate}
			\item El sistema guarda los cambios hechos hasta ahora y el caso de uso finaliza.
		\end{enumerate}
		\item[2.a] No ha sido creado ningún grupo para ese curso.
		\begin{enumerate}
			\item El sistema indica el error y finaliza el caso de uso.
		\end{enumerate}
		\item[3.a] Ya hay un horario empezado para ese grupo, semestre y semana
		\begin{enumerate}
			\item El sistema muestra el horario ya empezado ubicando los slots donde estaban en el antiguo horario y el caso de uso continua.
		\end{enumerate}
	\end{itemize}
\end{itemize}



\subsubsection*{Caso de uso: Añadir grupo de teoría}
\begin{itemize}
\item{\bf Descripción:} Se añade un grupo de teoría a un curso de una titulación para poder editar más adelante su horario
\item{\bf Actores:} Planificador. 
\item{\bf Precondiciones:} La titulación existe y tiene asignaturas con planes docentes creados para el curso actual. El usuario identificado en el sistema es un planificador.
\item{\bf Postcondiciones:} El grupo se añade al curso correspondiente.
\item{\bf Escenario principal:}
	\begin{enumerate}
	\item Se realiza el caso de uso {\em \hyperref[select_grupo]{Seleccionar grupo de teoría}}.
	\item El administrador selecciona el curso de la titulación al que se va a añadir el grupo.
	\item El sistema comprueba que no se haya sobrepasado el número de grupos indicado en el plan docente y guarda el grupo.
	\end{enumerate}
\item{\bf Escenarios alternativos:}
	\begin{itemize}
		\item[2.a.] Se han sobrepasado el número de grupos indicado en el plan docente.
		\begin{enumerate}
			\item El sistema advierte del problema pero el caso de uso continúa.
		\end{enumerate}
	\end{itemize}
\end{itemize}



\subsubsection*{Caso de uso: Eliminar grupo de teoría}
\begin{itemize}
\item{\bf Descripción:} Se elimina un grupo de teoría de un curso de una titulación
\item{\bf Actores:} Planificador.
\item{\bf Precondiciones:} Hay grupos creados para la titulación y curso. El usuario identificado en el sistema es un planificador.
\item{\bf Postcondiciones:} Se elimina el grupo así como toda su información asociada (horarios, informes, etc).
\item{\bf Escenario principal:}
	\begin{enumerate}
	\item Se realiza el caso de uso {\em \hyperref[select_grupo]{Seleccionar grupo de teoría}}.
	\item El administrador selecciona el curso de la titulación al que se va a eliminar el grupo.
	\item El sistema pide confirmación de la eliminación.
	\item El administrador confirma que desea eliminar el grupo.
	\item Queda borrado el grupo y toda la información asociada y el sistema lo indica.
	\end{enumerate}
\item{\bf Escenarios alternativos:}
	\begin{itemize}
		\item[4.a.] El administrador cancela el proceso.
		\begin{enumerate}
			\item El caso de uso finaliza.
		\end{enumerate}
	\end{itemize}
\end{itemize}



\subsubsection*{Caso de uso: Exportar horario}
\begin{itemize}
\item{\bf Descripción:} Caso de uso para exportar un horario creado a un formato externo como csv o una hoja de cálculo.
\item{\bf Actores:} Planificador.
\item{\bf Precondiciones:} Existe un grupo creado y está creado el horario. El usuario identificado en el sistema es un planificador.
\item{\bf Postcondiciones:} Se exporta un archivo con un formato adecuado.
\item{\bf Escenario principal:}
	\begin{enumerate}
	\item Se realiza el caso de uso {\em \hyperref[select_grupo]{Seleccionar grupo de teoría}}.
	\item El administrador selecciona la semana, semestre y curso.
	\item El sistema busca el horario elegido.
	\item Se muestra y permite descarga del archivo exportado.
	\end{enumerate}
\item{\bf Escenarios alternativos:}
	\begin{itemize}
		\item[*.a.] En cualquier momento el administrador decide cancelar el proceso.
		\begin{enumerate}
			\item El caso de uso finaliza.
		\end{enumerate}
		\item[3.a.] No existe un horario creado para los datos elegidos.
		\begin{enumerate}
			\item El sistema lo indica y se vuelve al paso anterior.
		\end{enumerate}
	\end{itemize}
\end{itemize}


\subsubsection*{Caso de uso: Comprobar grupo de teoría}
\begin{itemize}
\item{\bf Descripción:} Caso de uso para comprobar que el número de horas introducidas en los horarios de un grupo coincide con lo indicado en el plan docente de cada asignatura.
\item{\bf Actores:} Planificador.
\item{\bf Precondiciones:} El grupo existe y tiene horarios creados. El usuario identificado en el sistema es un planificador.
\item{\bf Postcondiciones:} Se muestra la comprobación indicando si faltan o sobran horas en las asignaturas y actividades.
\item{\bf Escenario principal:}
	\begin{enumerate}
	\item Se realiza el caso de uso {\em \hyperref[select_grupo]{Seleccionar grupo de teoría}}.
	\item El sistema muestra el cálculo de horas de un grupo mediante los horarios de esa semana, teniendo en cuenta los eventos asignados en el calendario e indica dónde faltan horas por asignar para que sea solucionado.
	\end{enumerate}
\item{\bf Escenarios alternativos:}
	\begin{itemize}
		\item[*.a.] En cualquier momento se decide parar el proceso.
		\begin{enumerate}
			\item El caso de uso finaliza.
		\end{enumerate}
	\end{itemize}
\end{itemize}



\subsubsection{Gestión de aulas}
\begin{figure}[H] 
  \label{gestion-aulas} 
	\begin{center}
    \includegraphics[scale=0.5]{./gestionaulas.png}
  \end{center}
\caption{Diagrama de casos de uso de la gestión de aulas}
\end{figure}
\subsubsection*{Caso de uso: Crear aula}
\begin{itemize}
\item{\bf Descripción:} Caso de uso para crear un aula a la que asignar slots de horario
\item{\bf Actores:} Planificador.
\item{\bf Precondiciones:} El usuario identificado en el sistema es un planificador.
\item{\bf Postcondiciones:} El aula queda creada
\item{\bf Escenario principal:}
	\begin{enumerate}
	\item El administrador introduce los datos del aula: nombre y actividades para las que sirve
	\item El sistema comprueba que no exista ningún aula con ese nombre y el aula queda creada.
	\end{enumerate}
\item{\bf Escenarios alternativos:}
	\begin{itemize}
		\item[*.a.] En cualquier momento el administrador decide cancelar el proceso.
		\begin{enumerate}
			\item El caso de uso finaliza.
		\end{enumerate}
		\item[2.a.] Ya existe un aula con ese nombre.
		\begin{enumerate}
			\item El sistema lo indica y se vuelve al paso anterior.
		\end{enumerate}
	\end{itemize}
\end{itemize}



\subsubsection*{Caso de uso: Editar aula}
\begin{itemize}
\item{\bf Descripción:} Caso de uso para editar los datos de un aula ya creada
\item{\bf Actores:} Planificador.
\item{\bf Precondiciones:} El aula existe. El usuario identificado en el sistema es un planificador.
\item{\bf Postcondiciones:} El aula queda guardada con los nuevos datos.
\item{\bf Escenario principal:}
	\begin{enumerate}
	\item El sistema muestra los datos actuales del aula permitiendo su edición
	\item El administrador edita el nombre y añade o elimina actividades.
	\item El sistema valida que el nuevo nombre no exista y guarda los datos
	\end{enumerate}
\item{\bf Escenarios alternativos:}
	\begin{itemize}
		\item[*.a.] En cualquier momento el administrador decide cancelar el proceso
		\begin{enumerate}
			\item El caso de uso finaliza.
		\end{enumerate}
		\item[3.a.] Ya existe un aula con ese nombre
		\begin{enumerate}
			\item El sistema lo indica y se vuelve al paso anterior
		\end{enumerate}
	\end{itemize}
\end{itemize}



\subsubsection*{Caso de uso: Eliminar aula}
\begin{itemize}
\item{\bf Descripción:} Caso de uso para eliminar un aula del sistema.
\item{\bf Actores:} Planificador.
\item{\bf Precondiciones:} El aula existe en el sistema. El usuario identificado en el sistema es un planificador.
\item{\bf Postcondiciones:} El aula queda eliminada del sistema y todas las referencias a ésta quedan borradas.
\item{\bf Escenario principal:}
	\begin{enumerate}
	\item Se selecciona el aula a eliminar
	\item El sistema pide confirmación
	\item El administrador confirma la eliminación
	\item El sistema elimina el aula y borra todas las referencias a ésta.
	\end{enumerate}
\item{\bf Escenarios alternativos:}
	\begin{itemize}
		\item[3.a.] El administrador decide que no desea eliminar el aula.
		\begin{enumerate}
			\item El caso de uso finaliza
		\end{enumerate}
	\end{itemize}
\end{itemize}



\subsubsection*{Caso de uso: Ver ocupación de aula}
\begin{itemize}
\item{\bf Descripción:} Caso de uso para ver la ocupación actual de un aula según los horarios creados para el curso actual.
\item{\bf Actores:} Planificador.
\item{\bf Precondiciones:} El aula existe en el sistema. El usuario identificado en el sistema es un planificador.
\item{\bf Postcondiciones:} Se muestran todos los slots que están asignados a ese aula junto con su horario.
\item{\bf Escenario principal:}
	\begin{enumerate}
	\item El administrador selecciona el aula para la que desea ver la ocupación.
	\item El administrador selecciona el semestre y la semana para el que desea ver la ocupación.
	\item El sistema muestra la ocupación actual.
	\end{enumerate}
\item{\bf Escenarios alternativos:}
	\begin{itemize}
		\item[*.a.] En cualquier momento el administrador decide cancelar el proceso.
		\begin{enumerate}
			\item El caso de uso finaliza.
		\end{enumerate}
	\end{itemize}
\end{itemize}

\subsubsection*{Caso de uso: Visualización de horario}
\begin{itemize}
\item{\bf Descripción:} Caso de uso para visualizar un horario por parte de un alumno personalizado con las asignaturas seleccionadas por él.
\item{\bf Actores:} Alumno.
\item{\bf Precondiciones:} El usuario identificado en el sistema es un alumno.
\item{\bf Postcondiciones:} Se muestran todos los slots que ha seleccionado el alumno.
\item{\bf Escenario principal:}
	\begin{enumerate}
	\item Se realiza el caso de uso {\em \hyperref[select_asignatura]{Seleccionar asignatura}}.
	\item Se repite el paso anterior hasta seleccionar todas las asignaturas deseadas.
	\item El sistema muestra un horario personalizado según las asignaturas elegidas.
	\end{enumerate}
\item{\bf Escenarios alternativos:}
	\begin{itemize}
		\item[*.a.] En cualquier momento el alumno decide cancelar el proceso.
		\begin{enumerate}
			\item El caso de uso finaliza.
		\end{enumerate}
	\end{itemize}
\end{itemize}

\section{Modelo conceptual de datos}

\subsection{Diagrama de clases conceptuales}

\begin{figure}[H] 
  \label{modelo-conceptual} 
	\begin{center}
    \includegraphics[scale=0.5]{./modeloconceptual.png}
  \end{center}
\caption{Diagrama del modelo conceptual de datos}
\end{figure}

\section{Modelo de comportamiento del sistema}

Para el modelo de comportamiento del sistema se mostrarán diferentes diagramas de secuencia del sistema. El diagrama define las interacciones entre actores y sistema, también se detallarán los contratos de las operaciones del sistema, para describir en detalle qué hace cada operación.
\paragraph{}
Al existir muchos casos de uso similares, sólo se detallarán los más relevantes de cada subsistema.

\subsection{Caso de uso: Registrar titulación}

\begin{figure}[H] 
  \label{comportamiento-reg-titulacion} 
	\begin{center}
    \includegraphics[scale=0.5]{./secuencia-reg-titulacion.png}
  \end{center}
\caption{Diagrama de secuencia del caso de uso Registrar titulación}
\end{figure}

\subsubsection{Contrato de la operación: introducir\_datos\_titulación}
\begin{itemize}
\item {\bf Responsabilidades:} Registrar una titulación en el sistema.
\item {\bf Referencias cruzadas:} Caso de uso {\em registrar titulación}, Caso de uso {\em editar titulación}.
\item {\bf Precondiciones:} No existe ninguna titulación con código = w\_código.
\item {\bf Postcondiciones:} Se crea una instancia T de Titulación. Se asignan w\_código y datos a T.
\end{itemize}

\subsection{Caso de uso: Registrar asignatura}

\begin{figure}[H] 
  \label{comportamiento-reg-asignatura} 
	\begin{center}
    \includegraphics[scale=0.5]{./secuencia-reg-asignatura.png}
  \end{center}
\caption{Diagrama de secuencia del caso de uso Registrar asignatura}
\end{figure}

\subsubsection{Contrato de la operación: seleccionar\_titulacion}
\begin{itemize}
\item {\bf Responsabilidades:} Seleccionar una titulación para añadirle una asignatura.
\item {\bf Referencias cruzadas:} Caso de uso {\em registrar asignatura}, Caso de uso {\em editar asignatura}
\item {\bf Precondiciones:} Existe una titulación con t.id = id\_titulación.
\item {\bf Postcondiciones:} Se devuelve una instancia T de Titulación con T.id = id\_titulación.
\end{itemize}

\subsubsection{Contrato de la operación: introducir\_datos\_asignatura}
\begin{itemize}
\item {\bf Responsabilidades:} Registrar una asignatura y asociarla a una titulación.
\item {\bf Referencias cruzadas:} Caso de uso {\em registrar asignatura}, Caso de uso {\em editar asignatura}
\item {\bf Precondiciones:} No existe ninguna asignatura con código = w\_codigo.
\item {\bf Postcondiciones:} Se crea una instancia A de Asignatura. Se asigna A.codigo = w\_codigo y A.datos = datos.
\end{itemize}

\subsection{Caso de uso: Editar asignatura}
\begin{figure}[H] 
  \label{comportamiento-edit-asignatura} 
	\begin{center}
    \includegraphics[scale=0.5]{./secuencia-edit-asignatura.png}
  \end{center}
\caption{Diagrama de secuencia del caso de uso Editar asignatura}
\end{figure}

\subsubsection{Contrato de la operación: seleccionar\_asignatura}
\begin{itemize}
\item {\bf Responsabilidades:} Devolver una instancia de Asignatura para poder editar sus datos.
\item {\bf Referencias cruzadas:} Caso de uso {\em editar asignatura}, Caso de uso {\em borrar asignatura}.
\item {\bf Precondiciones:} Debe existir una Asignatura registrada en el sistema con A.id = id\_asignatura.
\item {\bf Postcondiciones:} Se devuelve una instancia A de Asignatura con A.id = id\_asignatura.
\end{itemize}

\subsection{Caso de uso: Borrar asignatura}
\begin{figure}[H] 
  \label{comportamiento-borrar-asignatura} 
	\begin{center}
    \includegraphics[scale=0.5]{./secuencia-borrar-asignatura.png}
  \end{center}
\caption{Diagrama de secuencia del caso de uso Borrar asignatura}
\end{figure}

\subsection{Caso de uso: Importar asignatura}
\begin{figure}[H] 
  \label{comportamiento-importar-asignatura} 
	\begin{center}
    \includegraphics[scale=0.5]{./secuencia-importar-asignatura.png}
  \end{center}
\caption{Diagrama de secuencia del caso de uso Importar asignatura}
\end{figure}

\subsubsection{Contrato de la operación: subir\_archivo}
\begin{itemize}
\item {\bf Responsabilidades:} Crear asignaturas masivamente capturando la información de las líneas de un archivo.
\item {\bf Referencias cruzadas:} Caso de uso {\em importar asignatura}, Caso de uso {\em importar titulación}, Caso de uso {\em importar plan docente}.
\item {\bf Precondiciones:} Por cada línea del archivo, no existe una instancia de la clase a la que se pretende añadir información con la misma clave que aparece en la línea.
\item {\bf Postcondiciones:} Por cada línea se crea una instancia de la clase con id = id\_linea.
\end{itemize}

\subsection{Caso de uso: Crear plan docente}
\begin{figure}[H] 
  \label{comportamiento-crear-plandocente} 
	\begin{center}
    \includegraphics[scale=0.5]{./secuencia-crear-plandocente.png}
  \end{center}
\caption{Diagrama de secuencia del caso de uso Crear plan docente}
\end{figure}

\subsubsection{Contrato de la operación: introducir\_horas\_actividad}
\begin{itemize}
\item {\bf Responsabilidades:} Añadir la planificación docente de una actividad concreta al sistema.
\item {\bf Referencias cruzadas:} Caso de uso {\em crear plan docente}, Caso de uso {\em editar plan docente}
\item {\bf Precondiciones:} La actividad existe en el sistema.
\item {\bf Postcondiciones:} Se crea una instancia P de Plan\_Actividad con P.actividad = actividad, P.horas = horas y  P.horas\_semanales = horas\_semanales, además se asocia con la asignatura A.
\end{itemize}

\subsection{Caso de uso: Generar informe de asignatura}
\begin{figure}[H] 
  \label{comportamiento-generar-informe} 
	\begin{center}
    \includegraphics[scale=0.5]{./secuencia-gen-informe.png}
  \end{center}
\caption{Diagrama de secuencia del caso de uso Generar informe de asignatura}

\subsection{Caso de uso: Añadir evento al calendario}
\begin{figure}[H] 
  \label{comportamiento-anadir-evento} 
	\begin{center}
    \includegraphics[scale=0.5]{./secuencia-crear-evento.png}
  \end{center}
\caption{Diagrama de secuencia del caso de uso Añadir evento al calendario}
\end{figure}
\end{figure}

\subsubsection{Contrato de la operación: seleccionar curso}
\begin{itemize}
\item {\bf Responsabilidades:} Seleccionar un curso para hacer operaciones con el o asociarlo con otras clases.
\item {\bf Referencias cruzadas:} Caso de uso {\em Añadir evento al calendario}.
\item {\bf Precondiciones:} Existe un curso C con C.id\_curso = id\_curso.
\item {\bf Postcondiciones:} Se devuelve una instancia C de curso.
\end{itemize}

\subsubsection{Contrato de la operación: introduce\_datos\_evento}
\begin{itemize}
\item {\bf Responsabilidades:} Crear un evento en el calendario del curso C.
\item {\bf Referencias cruzadas:} Caso de uso {\em Añadir evento al calendario}.
\item {\bf Precondiciones:} No existe ningún evento con fechas entre las introducidas.
\item {\bf Postcondiciones:} Se crea una instancia E de Evento con E.fecha\_inicial = fecha\_inicial, E.fecha\_final = fecha\_final y E.nombre = nombre y se asocia a C.
\end{itemize}

\subsection{Caso de uso: Exportar calendario}
\begin{figure}[H] 
  \label{comportamiento-exportar-calendario} 
	\begin{center}
    \includegraphics[scale=0.5]{./secuencia-exportar-calendario.png}
  \end{center}
\caption{Diagrama de secuencia del caso de uso Exportar calendario}
\end{figure}


\subsection{Caso de uso: Seleccionar grupos de teoría}
\begin{figure}[H] 
  \label{comportamiento-seleccionar-grupos} 
	\begin{center}
    \includegraphics[scale=0.5]{./secuencia-seleccionar-grupo.png}
  \end{center}
\caption{Diagrama de secuencia del caso de uso Seleccionar grupos de teoría}
\end{figure}

\subsubsection{Contrato de la operación: seleccionar\_grupo}
\begin{itemize}
\item {\bf Responsabilidades:} Seleccionar un grupo de una titulación para hacer uso de él en otro caso de uso.
\item {\bf Referencias cruzadas:} Caso de uso {\em seleccionar grupos de teoría}
\item {\bf Precondiciones:} El curso indicado está dentro de los cursos de la titulación.
\\El grupo ha sido creado.
\item {\bf Postcondiciones:} Se devuelve una instancia G del Grupo.
\end{itemize}

\subsection{Caso de uso: Añadir grupo de teoría}
\begin{figure}[H] 
  \label{comportamiento-anadir-grupos} 
	\begin{center}
    \includegraphics[scale=0.5]{./secuencia-anadir-grupo.png}
  \end{center}
\caption{Diagrama de secuencia del caso de uso Añadir grupo de teoría}
\end{figure}

\subsection{Caso de uso: Eliminar grupo de teoría}
\begin{figure}[H] 
  \label{comportamiento-borrar-grupos} 
	\begin{center}
    \includegraphics[scale=0.5]{./secuencia-borrar-grupo.png}
  \end{center}
\caption{Diagrama de secuencia del caso de uso Eliminar grupo de teoría}
\end{figure}

\subsection{Caso de uso: Ubicar slot de horario}
\begin{figure}[H] 
  \label{comportamiento-ubicar-slot} 
	\begin{center}
    \includegraphics[scale=0.5]{./secuencia-ubicar-slot.png}
  \end{center}
\caption{Diagrama de secuencia del caso de uso Ubicar slot de horario}
\end{figure}

\subsubsection{Contrato de la operación: seleccionar\_actividad}
\begin{itemize}
\item {\bf Responsabilidades:} Seleccionar una actividad para introducirla en el horario
\item {\bf Referencias cruzadas:} Caso de uso {\em ubicar slot de horario}.
\item {\bf Precondiciones:} La asignatura elegida dispone de horas en el plan docente para esa actividad.
\item {\bf Postcondiciones:} Se devuelve una instancia de un Slot de horario para esa asignatura y actividad.
\end{itemize}

\subsubsection{Contrato de la operación: introducir\_datos\_slot}
\begin{itemize}
\item {\bf Responsabilidades:} Ubicar en el horario un slot de una actividad de una asignatura.
\item {\bf Referencias cruzadas:} Caso de uso {\em ubicar slot de horario}
\item {\bf Precondiciones:} Si el slot es de teoría no se solapa con otro slot. No se debe solapar con otro slot con el mismo id\_aula.
\item {\bf Postcondiciones:} El slot queda ubicado en el horario.
\end{itemize}

\subsection{Caso de uso: Editar horario tipo}
\begin{figure}[H] 
  \label{comportamiento-editar-tipo} 
	\begin{center}
    \includegraphics[scale=0.5]{./secuencia-editar-tipo.png}
  \end{center}
\caption{Diagrama de secuencia del caso de uso Editar horario tipo}
\end{figure}

\subsubsection{Contrato de la operación: seleccionar\_curso\_titulacion}
\begin{itemize}
\item {\bf Responsabilidades:} Seleccionar una instancia del Horario
\item {\bf Referencias cruzadas:} Caso de uso {\em editar horario tipo}, Caso de uso {\em editar horario semana inicial}.
\item {\bf Precondiciones:} El curso está dentro de los que tiene la titulación.
\item {\bf Postcondiciones:} Se devuelve una instancia de Horario (o se crea una si no existe).
\end{itemize}

\subsection{Caso de uso: Comprobar grupo de teoría}
\begin{figure}[H] 
  \label{comportamiento-comprobar-grupo} 
	\begin{center}
    \includegraphics[scale=0.5]{./secuencia-comprobar-grupo.png}
  \end{center}
\caption{Diagrama de secuencia del caso de uso Comprobar grupo de teoría}
\end{figure}

\subsection{Caso de uso: Crear aula}

\begin{figure}[H] 
  \label{comportamiento-crear-aula} 
	\begin{center}
    \includegraphics[scale=0.5]{./secuencia-crear-aula.png}
  \end{center}
\caption{Diagrama de secuencia del caso de uso Crear aula}
\end{figure}

\subsubsection{Contrato de la operación: introducir\_datos\_aula}
\begin{itemize}
\item {\bf Responsabilidades:} Crear un aula en el sistema.
\item {\bf Referencias cruzadas:} Caso de uso {\em crear aula}, Caso de uso {\em Editar aula}.
\item {\bf Precondiciones:} No existe ningún aula en el sistema con A.nombre = nombre.
\item {\bf Postcondiciones:} Se crea una instancia A de Aula con A.nombre = nombre y A.actividades = actividades.
\end{itemize}

\subsection{Caso de uso: Borrar aula}
\begin{figure}[H] 
  \label{comportamiento-borrar-aula} 
	\begin{center}
    \includegraphics[scale=0.5]{./secuencia-borrar-aula.png}
  \end{center}
\caption{Diagrama de secuencia del caso de uso Borrar aula}
\end{figure}

\subsubsection{Contrato de la operación: seleccionar\_aula}
\begin{itemize}
\item {\bf Responsabilidades:} Seleccionar un aula para borrarla o editarla.
\item {\bf Referencias cruzadas:} Caso de uso {\em borrar aula}, Caso de uso {\em editar aula}.
\item {\bf Precondiciones:} Existe un aula en el sistema con A.id\_aula = id\_aula.
\item {\bf Postcondiciones:} Se devuelve una instancia A de aula con A.id\_aula = id\_aula.
\end{itemize}

\subsection{Caso de uso: Editar aula}
\begin{figure}[H] 
  \label{comportamiento-editar-aula} 
	\begin{center}
    \includegraphics[scale=0.5]{./secuencia-editar-aula.png}
  \end{center}
\caption{Diagrama de secuencia del caso de uso Editar aula}
\end{figure}
% Plantilla para contratos de operaciones

\chapter{Diseño}
La fase de diseño consiste en aplicar una serie de técnicas para transformar los requisitos elicitados en la fase de análisis en una estructura detallada para el sistema de forma que se pueda implementar fácilmente a partir de ese diseño.\\

Siguiendo los requisitos de la fase anterior, esta tarea es relativamente sencilla y debe resultar en una serie de diagramas y especificaciones que sirvan como guión y documentación a las personas que vayan a participar, ahora o en un futuro en el desarrollo del sistema.\\

La aplicación se rige por el patrón arquitectónico MVC. La capa del modelo equivale al SGBD y al ORM, mientras el controlador es el responsable de hacer de intermediario entre vista y modelo, llevando la lógica de la aplicación. Las vistas serán los ficheros XHTML junto con sus estilos.

\section{Controladores}

A continuación, se comentarán las acciones más importantes que componen cada uno de los controladores. Normalmente cada controlador tiene asociado un modelo, y cada acción del controlador una vista.

\subsection{Users}

Controlador encargado de llevar toda la gestión de usuarios

\begin{itemize}
\item users::add() - Muestra el formulario de creación de un usuario.
\item users::create() - Crea un usuario según los datos introducidos en el formulario.
\item users::edit() - Muestra un formulario con los datos del usuario a editar sobre él para poder ser modificados.
\item users::update() - Actualiza un usuario segun los datos introducidos.
\item users::recuperar\_password() - Muestra el formulario para recuperar una contraseña.
\item users::envio\_recuperar() - Envía y cambia la contraseña a un usuario por email a la dirección introducida en el formulario.
\end{itemize}

\subsection{Login}

Controlador encargado de gestionar la identificación de usuarios en el sistema.

\begin{itemize}
\item login::index() - Muestra el formulario de solicitud de datos de acceso.
\item login::submit() - Comprueba los datos introducidos en el formulario anterior e identifica al usuario si son correctos, sino recarga el formulario y lo indica.
\end{itemize}

\subsection{Logout}

Controlador encargado de gestionar la salida de los usuarios del sistema

\begin{itemize}
\item logout::index() - Provoca la salida del usuario del sistema.
\end{itemize}

\subsection{Admin}

Controlador encargado de algunas funciones de administración.

\begin{itemize}
\item admin::restaurar\_backup() - Muestra un formulario para cargar un archivo sql con una copia de seguridad de la base de datos, restaurando los datos que estuvieran en ese archivo.
\item admin::restaurar() - Restaura la base de datos con el archivo cargado en el formulario anterior.
\item admin::backup() - Permite la descarga de un backup del estado actual de la base de datos.
\end{itemize}

\subsection{Titulaciones}
Controlador encargado de llevar a cabo toda la gestión de las titulaciones.

\begin{itemize}
\item titulaciones::index() - Muestra las titulaciones que hay creadas en el sistema.
\item titulaciones::show\_informes() - Muestra un listado de las asignaturas de la titulación permitiendo generar un informe con las que se quieran seleccionar.
\item titulaciones::add() - Muestra un formulario de creación de titulaciones.
\item titulaciones::create() - Crea una titulación a partir de los datos introducidos en el formulario.
\item titulaciones::edit() - Permite la edición de una titulación mostrando un formulario.
\item titulaciones::update() - Actualiza los datos de una titulación existente.
\item titulaciones::delete() - Borra una titulación del sistema.
\item titulaciones::show() - Muestra las asignaturas de esa titulación.
\item titulaciones::show\_planificacion() - Muestra una tabla con la planificación docente de una titulación, con una fila por asignatura.
\item titulaciones::exportar\_planificacion() - Permite la descarga de un archivo csv con la planificación docente completa de todas las asignaturas de la titulación.
\item titulaciones::select\_titulacion() - Muestra un listado de las titulaciones permitiendo enrutar hacia a otra acción que necesite seleccionar una titulación.
\end{itemize}

\subsection{Asignaturas}
Controlador encargado de llevar a cabo toda la gestión de las asignaturas.

\begin{itemize}
\item asignaturas::show() - Muestra la planificación docente de una asignatura para un curso concreto.
\item asignaturas::add\_to() - Añade una asignatura a una titulación dada mostrando un formulario.
\item asignaturas::create() - Crea una asignatura siguiendo los datos introducidos en el formulario.
\end{itemize}

\subsection{Aulas}
Controlador encargado de llevar a cabo la gestión de las aulas.

\begin{itemize}
\item aulas::add() - Muestra el formulario de creación de un aula.
\item aulas::create() - Crea un aula a partir de los datos introducidos en el formulario.
\item aulas::index() - Muestra un listado de todas las aulas.
\item aulas::exportar\_ocupacion() - Exporta a un archivo csv la ocupación de un aula según los horarios de un curso concreto.
\end{itemize}

\subsection{Cursos}

Controlador que lleva toda la gestión de los cursos.

\begin{itemize}
\item cursos::add() - Muestra el formulario para la creación de un curso nuevo.
\item cursos::create() - Crea un curso con los datos introducidos.
\item cursos::edit() - Muestra un formulario para editar un curso ya creado.
\item cursos::update() - Actualiza un curso existente con los datos introducidos.
\item cursos::delete() - Borra un curso existente.
\item cursos::select\_curso() - Muestra un listado de los cursos permitiendo enrutar hacia a otra acción que necesite seleccionar uno.
\end{itemize}

\subsection{Eventos}

Controlador encargado de la gestión del calendario.

\begin{itemize}
\item eventos::index() - Muestra el calendario del sistema permitiendo crear nuevos eventos sobre el.
\item eventos::add() - Muestra un formulario para la creación de eventos.
\item eventos::create() - Crea un evento con los datos introducidos en el formulario.
\item eventos::delete() - Borra un evento existente.
\item eventos::fetch\_events() - Función que devuelve en formato JSON los datos de los eventos del sistema para mostrarlos sobreimpresos en el calendario.
\item eventos::export\_calendar() - Exporta y permite la descarga del calendario en formato csv.
\end{itemize}

\subsection{Horarios}

Controlador encargado de la gestión de los horarios.

\begin{itemize}
\item horarios::select\_grupo() - Muestra los cursos de una titulación y sus grupos creados, con enlaces a los diferentes horarios.
\item horarios::add\_grupo() - Añade un grupo a un curso de una titulación concreta.
\item horarios::edit() - Muestra un horario editable permitiendo ubicar los diferentes slots de las asignaturas.
\item horarios::ocupacion\_aula() - Muestra la ocupación de un aula concreta.
\item horarios::exportar\_ocupacion() - Permite exportar y descargar un archivo csv con la ocupación de un aula.
\item horarios::edit\_teoria() - Muestra y permite editar un horario de una semana que solo tiene asignaturas de teoría.
\item horarios::check\_grupo() - Realiza la comprobación de las horas planificadas y asignadas en los horarios de un grupo de una titulación, mostrando el resultado.
\item horarios::save\_line() - Guarda un slot de una asignatura en un horario concreto.
\item horarios::delete() - Borra un horario.
\item horarios::delete\_line() - Borra un slot de un horario, dejándolo sin asignar.
\item horarios::exportar() - Permite exportar un horario concreto a csv permitiendo la descarga del archivo.
\item horarios::add\_extra\_slot() - Permite añadir un slot extra de una asignatura en un horario de una semana de teoria.
\item horarios::visualizacion\_asignaturas() - Permite a un alumno la visualización de un horario personalizado. Para ello esta función muestra un listado de las asignaturas disponibles permitiendo seleccionar las deseadas.
\item horarios::visualizacion\_mostrar\_grupos() - Muestra las asignaturas seleccionadas en el paso anterior y permite seleccionar los grupos deseados.
\item horarios::visualizacion\_mostrar\_horario() - Muestra el horario personalizado en los dos pasos anteriores.
\end{itemize}


\subsection{PlanesDocentes}

Controlador encargado de la gestión de la planificación docente.

\begin{itemize}
\item planesdocentes::add\_carga() - Muestra un formulario para añadir la planificación docente de un curso a una asignatura.
\item planesdocentes::create() - Crea en el sistema la planificación docente según los datos introducidos en el formulario.
\item planesdocentes::edit() - Muestra un formulario para editar la planificación docente ya existente de una asignatura.
\item planesdocentes::update() - Actualiza la planificación docente de una asignatura según los datos ya introducidos.
\item planesdocentes::make\_upload() - Muestra un formulario para subir un archivo con un csv con la planificación docente de una o varias asignaturas.
\item planesdocentes::upload\_file() - Parsea el fichero subido en el paso anterior y crea los planes docentes nuevos.
\item planesdocentes::informe\_asignatura() - Genera un informe en pdf de una o varias asignaturas, permitiendo su descarga.
\end{itemize}

\section{Base de datos}
Para el diseño de la base de datos en la que se guardarán los datos manejados por la aplicación se usará un modelo relacional. Se usará MySQL como sistema de gestión de base de datos. 
\subsection{Modelo entidad-relación}

\subsection{Tablas y atributos}

% phpMyAdmin LaTeX Dump
% version 3.4.5
% http://www.phpmyadmin.net
%
% Servidor: localhost
% Tiempo de generación: 08-12-2011 a las 15:07:50
% Versión del servidor: 5.5.16
% Versión de PHP: 5.3.8
% 
% Base de datos: 'pfc_development'
% 

%
% Estructura: actividades
%
 \begin{longtable}{|l|c|c|c|} 
 \caption{Estructura de la tabla actividades} \label{tab:actividades-structure} \\
 \hline \multicolumn{1}{|c|}{\textbf{Columna}} & \multicolumn{1}{|c|}{\textbf{Tipo}} & \multicolumn{1}{|c|}{\textbf{Nulo}} & \multicolumn{1}{|c|}{\textbf{Predeterminado}} \\ \hline \hline
\endfirsthead
 \caption{Estructura de la tabla actividades (continúa)} \\ 
 \hline \multicolumn{1}{|c|}{\textbf{Columna}} & \multicolumn{1}{|c|}{\textbf{Tipo}} & \multicolumn{1}{|c|}{\textbf{Nulo}} & \multicolumn{1}{|c|}{\textbf{Predeterminado}} \\ \hline \hline \endhead \endfoot 
\textbf{\textit{id}} & bigint(20)  & No &  \\ \hline 
descripcion & varchar(100) & No &  \\ \hline 
identificador & varchar(1) & No &  \\ \hline 
 \end{longtable}

%
% Estructura: actividades
%
 \begin{longtable}{|l|c|c|c|} 
 \caption{Estructura de la tabla actividades} \label{tab:actividades-structure} \\
 \hline \multicolumn{1}{|c|}{\textbf{Columna}} & \multicolumn{1}{|c|}{\textbf{Tipo}} & \multicolumn{1}{|c|}{\textbf{Nulo}} & \multicolumn{1}{|c|}{\textbf{Predeterminado}} \\ \hline \hline
\endfirsthead
 \caption{Estructura de la tabla actividades (continúa)} \\ 
 \hline \multicolumn{1}{|c|}{\textbf{Columna}} & \multicolumn{1}{|c|}{\textbf{Tipo}} & \multicolumn{1}{|c|}{\textbf{Nulo}} & \multicolumn{1}{|c|}{\textbf{Predeterminado}} \\ \hline \hline \endhead \endfoot 
\textbf{\textit{id}} & bigint(20)  & No &  \\ \hline 
descripcion & varchar(100) & No &  \\ \hline 
identificador & varchar(1) & No &  \\ \hline 
 \end{longtable}

%
% Estructura: asignaturas
%
 \begin{longtable}{|l|c|c|c|l|} 
 \caption{Estructura de la tabla asignaturas} \label{tab:asignaturas-structure} \\
 \hline \multicolumn{1}{|c|}{\textbf{Columna}} & \multicolumn{1}{|c|}{\textbf{Tipo}} & \multicolumn{1}{|c|}{\textbf{Nulo}} & \multicolumn{1}{|c|}{\textbf{Predeterminado}} & \multicolumn{1}{|c|}{\textbf{Enlaces a}} \\ \hline \hline
\endfirsthead
 \caption{Estructura de la tabla asignaturas (continúa)} \\ 
 \hline \multicolumn{1}{|c|}{\textbf{Columna}} & \multicolumn{1}{|c|}{\textbf{Tipo}} & \multicolumn{1}{|c|}{\textbf{Nulo}} & \multicolumn{1}{|c|}{\textbf{Predeterminado}} & \multicolumn{1}{|c|}{\textbf{Enlaces a}} \\ \hline \hline \endhead \endfoot 
\textbf{\textit{id}} & int(11) & No &  &  \\ \hline 
\textbf{codigo} & varchar(3) & No &  &  \\ \hline 
nombre & varchar(200) & No &  &  \\ \hline 
abreviatura & varchar(5) & No &  &  \\ \hline 
creditos & int(11) & No &  &  \\ \hline 
materia & varchar(100) & No &  &  \\ \hline 
departamento & varchar(200) & No &  &  \\ \hline 
curso & int(10)  & No &  &  \\ \hline 
semestre & varchar(255) & No &  &  \\ \hline 
titulacion\_id & int(11) & No &  & titulaciones (id) \\ \hline 
 \end{longtable}

%
% Estructura: asignaturas
%
 \begin{longtable}{|l|c|c|c|l|} 
 \caption{Estructura de la tabla asignaturas} \label{tab:asignaturas-structure} \\
 \hline \multicolumn{1}{|c|}{\textbf{Columna}} & \multicolumn{1}{|c|}{\textbf{Tipo}} & \multicolumn{1}{|c|}{\textbf{Nulo}} & \multicolumn{1}{|c|}{\textbf{Predeterminado}} & \multicolumn{1}{|c|}{\textbf{Enlaces a}} \\ \hline \hline
\endfirsthead
 \caption{Estructura de la tabla asignaturas (continúa)} \\ 
 \hline \multicolumn{1}{|c|}{\textbf{Columna}} & \multicolumn{1}{|c|}{\textbf{Tipo}} & \multicolumn{1}{|c|}{\textbf{Nulo}} & \multicolumn{1}{|c|}{\textbf{Predeterminado}} & \multicolumn{1}{|c|}{\textbf{Enlaces a}} \\ \hline \hline \endhead \endfoot 
\textbf{\textit{id}} & int(11) & No &  &  \\ \hline 
\textbf{codigo} & varchar(3) & No &  &  \\ \hline 
nombre & varchar(200) & No &  &  \\ \hline 
abreviatura & varchar(5) & No &  &  \\ \hline 
creditos & int(11) & No &  &  \\ \hline 
materia & varchar(100) & No &  &  \\ \hline 
departamento & varchar(200) & No &  &  \\ \hline 
curso & int(10)  & No &  &  \\ \hline 
semestre & varchar(255) & No &  &  \\ \hline 
titulacion\_id & int(11) & No &  & titulaciones (id) \\ \hline 
 \end{longtable}

%
% Estructura: aulaactividades
%
 \begin{longtable}{|l|c|c|c|l|} 
 \caption{Estructura de la tabla aulaactividades} \label{tab:aulaactividades-structure} \\
 \hline \multicolumn{1}{|c|}{\textbf{Columna}} & \multicolumn{1}{|c|}{\textbf{Tipo}} & \multicolumn{1}{|c|}{\textbf{Nulo}} & \multicolumn{1}{|c|}{\textbf{Predeterminado}} & \multicolumn{1}{|c|}{\textbf{Enlaces a}} \\ \hline \hline
\endfirsthead
 \caption{Estructura de la tabla aulaactividades (continúa)} \\ 
 \hline \multicolumn{1}{|c|}{\textbf{Columna}} & \multicolumn{1}{|c|}{\textbf{Tipo}} & \multicolumn{1}{|c|}{\textbf{Nulo}} & \multicolumn{1}{|c|}{\textbf{Predeterminado}} & \multicolumn{1}{|c|}{\textbf{Enlaces a}} \\ \hline \hline \endhead \endfoot 
\textbf{\textit{id\_actividad}} & bigint(20)  & No &  & actividades (id) \\ \hline 
\textbf{\textit{id\_aula}} & bigint(20)  & No &  & aulas (id) \\ \hline 
 \end{longtable}

%
% Estructura: aulaactividades
%
 \begin{longtable}{|l|c|c|c|l|} 
 \caption{Estructura de la tabla aulaactividades} \label{tab:aulaactividades-structure} \\
 \hline \multicolumn{1}{|c|}{\textbf{Columna}} & \multicolumn{1}{|c|}{\textbf{Tipo}} & \multicolumn{1}{|c|}{\textbf{Nulo}} & \multicolumn{1}{|c|}{\textbf{Predeterminado}} & \multicolumn{1}{|c|}{\textbf{Enlaces a}} \\ \hline \hline
\endfirsthead
 \caption{Estructura de la tabla aulaactividades (continúa)} \\ 
 \hline \multicolumn{1}{|c|}{\textbf{Columna}} & \multicolumn{1}{|c|}{\textbf{Tipo}} & \multicolumn{1}{|c|}{\textbf{Nulo}} & \multicolumn{1}{|c|}{\textbf{Predeterminado}} & \multicolumn{1}{|c|}{\textbf{Enlaces a}} \\ \hline \hline \endhead \endfoot 
\textbf{\textit{id\_actividad}} & bigint(20)  & No &  & actividades (id) \\ \hline 
\textbf{\textit{id\_aula}} & bigint(20)  & No &  & aulas (id) \\ \hline 
 \end{longtable}

%
% Estructura: aulas
%
 \begin{longtable}{|l|c|c|c|} 
 \caption{Estructura de la tabla aulas} \label{tab:aulas-structure} \\
 \hline \multicolumn{1}{|c|}{\textbf{Columna}} & \multicolumn{1}{|c|}{\textbf{Tipo}} & \multicolumn{1}{|c|}{\textbf{Nulo}} & \multicolumn{1}{|c|}{\textbf{Predeterminado}} \\ \hline \hline
\endfirsthead
 \caption{Estructura de la tabla aulas (continúa)} \\ 
 \hline \multicolumn{1}{|c|}{\textbf{Columna}} & \multicolumn{1}{|c|}{\textbf{Tipo}} & \multicolumn{1}{|c|}{\textbf{Nulo}} & \multicolumn{1}{|c|}{\textbf{Predeterminado}} \\ \hline \hline \endhead \endfoot 
\textbf{\textit{id}} & bigint(20)  & No &  \\ \hline 
nombre & varchar(100) & No &  \\ \hline 
 \end{longtable}

%
% Estructura: aulas
%
 \begin{longtable}{|l|c|c|c|} 
 \caption{Estructura de la tabla aulas} \label{tab:aulas-structure} \\
 \hline \multicolumn{1}{|c|}{\textbf{Columna}} & \multicolumn{1}{|c|}{\textbf{Tipo}} & \multicolumn{1}{|c|}{\textbf{Nulo}} & \multicolumn{1}{|c|}{\textbf{Predeterminado}} \\ \hline \hline
\endfirsthead
 \caption{Estructura de la tabla aulas (continúa)} \\ 
 \hline \multicolumn{1}{|c|}{\textbf{Columna}} & \multicolumn{1}{|c|}{\textbf{Tipo}} & \multicolumn{1}{|c|}{\textbf{Nulo}} & \multicolumn{1}{|c|}{\textbf{Predeterminado}} \\ \hline \hline \endhead \endfoot 
\textbf{\textit{id}} & bigint(20)  & No &  \\ \hline 
nombre & varchar(100) & No &  \\ \hline 
 \end{longtable}

%
% Estructura: calendarios
%
 \begin{longtable}{|l|c|c|c|} 
 \caption{Estructura de la tabla calendarios} \label{tab:calendarios-structure} \\
 \hline \multicolumn{1}{|c|}{\textbf{Columna}} & \multicolumn{1}{|c|}{\textbf{Tipo}} & \multicolumn{1}{|c|}{\textbf{Nulo}} & \multicolumn{1}{|c|}{\textbf{Predeterminado}} \\ \hline \hline
\endfirsthead
 \caption{Estructura de la tabla calendarios (continúa)} \\ 
 \hline \multicolumn{1}{|c|}{\textbf{Columna}} & \multicolumn{1}{|c|}{\textbf{Tipo}} & \multicolumn{1}{|c|}{\textbf{Nulo}} & \multicolumn{1}{|c|}{\textbf{Predeterminado}} \\ \hline \hline \endhead \endfoot 
\textbf{\textit{id}} & int(11) & No &  \\ \hline 
\textbf{codigo} & varchar(4) & No &  \\ \hline 
 \end{longtable}

%
% Estructura: calendarios
%
 \begin{longtable}{|l|c|c|c|} 
 \caption{Estructura de la tabla calendarios} \label{tab:calendarios-structure} \\
 \hline \multicolumn{1}{|c|}{\textbf{Columna}} & \multicolumn{1}{|c|}{\textbf{Tipo}} & \multicolumn{1}{|c|}{\textbf{Nulo}} & \multicolumn{1}{|c|}{\textbf{Predeterminado}} \\ \hline \hline
\endfirsthead
 \caption{Estructura de la tabla calendarios (continúa)} \\ 
 \hline \multicolumn{1}{|c|}{\textbf{Columna}} & \multicolumn{1}{|c|}{\textbf{Tipo}} & \multicolumn{1}{|c|}{\textbf{Nulo}} & \multicolumn{1}{|c|}{\textbf{Predeterminado}} \\ \hline \hline \endhead \endfoot 
\textbf{\textit{id}} & int(11) & No &  \\ \hline 
\textbf{codigo} & varchar(4) & No &  \\ \hline 
 \end{longtable}

%
% Estructura: cargas_semanales
%
 \begin{longtable}{|l|c|c|c|l|} 
 \caption{Estructura de la tabla cargas\_semanales} \label{tab:cargas_semanales-structure} \\
 \hline \multicolumn{1}{|c|}{\textbf{Columna}} & \multicolumn{1}{|c|}{\textbf{Tipo}} & \multicolumn{1}{|c|}{\textbf{Nulo}} & \multicolumn{1}{|c|}{\textbf{Predeterminado}} & \multicolumn{1}{|c|}{\textbf{Enlaces a}} \\ \hline \hline
\endfirsthead
 \caption{Estructura de la tabla cargas\_semanales (continúa)} \\ 
 \hline \multicolumn{1}{|c|}{\textbf{Columna}} & \multicolumn{1}{|c|}{\textbf{Tipo}} & \multicolumn{1}{|c|}{\textbf{Nulo}} & \multicolumn{1}{|c|}{\textbf{Predeterminado}} & \multicolumn{1}{|c|}{\textbf{Enlaces a}} \\ \hline \hline \endhead \endfoot 
\textbf{\textit{id}} & int(11) & No &  &  \\ \hline 
num\_semana & int(10)  & No &  &  \\ \hline 
horas\_teoria & int(10)  & No &  &  \\ \hline 
horas\_problemas & int(10)  & No &  &  \\ \hline 
horas\_informatica & int(10)  & No &  &  \\ \hline 
horas\_lab & int(10)  & No &  &  \\ \hline 
horas\_campo & int(10)  & No &  &  \\ \hline 
entrega\_trabajo & tinyint(1) & No &  &  \\ \hline 
examen & tinyint(1) & No &  &  \\ \hline 
plandocente\_id & int(11) & No &  & planesdocentes (id) \\ \hline 
 \end{longtable}

%
% Estructura: cargas_semanales
%
 \begin{longtable}{|l|c|c|c|l|} 
 \caption{Estructura de la tabla cargas\_semanales} \label{tab:cargas_semanales-structure} \\
 \hline \multicolumn{1}{|c|}{\textbf{Columna}} & \multicolumn{1}{|c|}{\textbf{Tipo}} & \multicolumn{1}{|c|}{\textbf{Nulo}} & \multicolumn{1}{|c|}{\textbf{Predeterminado}} & \multicolumn{1}{|c|}{\textbf{Enlaces a}} \\ \hline \hline
\endfirsthead
 \caption{Estructura de la tabla cargas\_semanales (continúa)} \\ 
 \hline \multicolumn{1}{|c|}{\textbf{Columna}} & \multicolumn{1}{|c|}{\textbf{Tipo}} & \multicolumn{1}{|c|}{\textbf{Nulo}} & \multicolumn{1}{|c|}{\textbf{Predeterminado}} & \multicolumn{1}{|c|}{\textbf{Enlaces a}} \\ \hline \hline \endhead \endfoot 
\textbf{\textit{id}} & int(11) & No &  &  \\ \hline 
num\_semana & int(10)  & No &  &  \\ \hline 
horas\_teoria & int(10)  & No &  &  \\ \hline 
horas\_problemas & int(10)  & No &  &  \\ \hline 
horas\_informatica & int(10)  & No &  &  \\ \hline 
horas\_lab & int(10)  & No &  &  \\ \hline 
horas\_campo & int(10)  & No &  &  \\ \hline 
entrega\_trabajo & tinyint(1) & No &  &  \\ \hline 
examen & tinyint(1) & No &  &  \\ \hline 
plandocente\_id & int(11) & No &  & planesdocentes (id) \\ \hline 
 \end{longtable}

%
% Estructura: ci_sessions
%
 \begin{longtable}{|l|c|c|c|} 
 \caption{Estructura de la tabla ci\_sessions} \label{tab:ci_sessions-structure} \\
 \hline \multicolumn{1}{|c|}{\textbf{Columna}} & \multicolumn{1}{|c|}{\textbf{Tipo}} & \multicolumn{1}{|c|}{\textbf{Nulo}} & \multicolumn{1}{|c|}{\textbf{Predeterminado}} \\ \hline \hline
\endfirsthead
 \caption{Estructura de la tabla ci\_sessions (continúa)} \\ 
 \hline \multicolumn{1}{|c|}{\textbf{Columna}} & \multicolumn{1}{|c|}{\textbf{Tipo}} & \multicolumn{1}{|c|}{\textbf{Nulo}} & \multicolumn{1}{|c|}{\textbf{Predeterminado}} \\ \hline \hline \endhead \endfoot 
\textbf{\textit{session\_id}} & varchar(40) & No &  \\ \hline 
ip\_address & varchar(16) & No & 0 \\ \hline 
user\_agent & varchar(120) & No &  \\ \hline 
last\_activity & int(10)  & No & 0 \\ \hline 
user\_data & text & Sí & NULL \\ \hline 
 \end{longtable}

%
% Estructura: ci_sessions
%
 \begin{longtable}{|l|c|c|c|} 
 \caption{Estructura de la tabla ci\_sessions} \label{tab:ci_sessions-structure} \\
 \hline \multicolumn{1}{|c|}{\textbf{Columna}} & \multicolumn{1}{|c|}{\textbf{Tipo}} & \multicolumn{1}{|c|}{\textbf{Nulo}} & \multicolumn{1}{|c|}{\textbf{Predeterminado}} \\ \hline \hline
\endfirsthead
 \caption{Estructura de la tabla ci\_sessions (continúa)} \\ 
 \hline \multicolumn{1}{|c|}{\textbf{Columna}} & \multicolumn{1}{|c|}{\textbf{Tipo}} & \multicolumn{1}{|c|}{\textbf{Nulo}} & \multicolumn{1}{|c|}{\textbf{Predeterminado}} \\ \hline \hline \endhead \endfoot 
\textbf{\textit{session\_id}} & varchar(40) & No &  \\ \hline 
ip\_address & varchar(16) & No & 0 \\ \hline 
user\_agent & varchar(120) & No &  \\ \hline 
last\_activity & int(10)  & No & 0 \\ \hline 
user\_data & text & Sí & NULL \\ \hline 
 \end{longtable}

%
% Estructura: cursos
%
 \begin{longtable}{|l|c|c|c|} 
 \caption{Estructura de la tabla cursos} \label{tab:cursos-structure} \\
 \hline \multicolumn{1}{|c|}{\textbf{Columna}} & \multicolumn{1}{|c|}{\textbf{Tipo}} & \multicolumn{1}{|c|}{\textbf{Nulo}} & \multicolumn{1}{|c|}{\textbf{Predeterminado}} \\ \hline \hline
\endfirsthead
 \caption{Estructura de la tabla cursos (continúa)} \\ 
 \hline \multicolumn{1}{|c|}{\textbf{Columna}} & \multicolumn{1}{|c|}{\textbf{Tipo}} & \multicolumn{1}{|c|}{\textbf{Nulo}} & \multicolumn{1}{|c|}{\textbf{Predeterminado}} \\ \hline \hline \endhead \endfoot 
\textbf{\textit{id}} & int(11) & No &  \\ \hline 
num\_semanas\_teoria & int(11) & No &  \\ \hline 
num\_semanas\_semestre1 & int(11) & No & 0 \\ \hline 
num\_semanas\_semestre2 & int(11) & No & 0 \\ \hline 
horas\_por\_credito & int(11) & No &  \\ \hline 
slot\_minimo & bigint(20) & No & 30 \\ \hline 
hora\_inicial & time & No & 09:00:00 \\ \hline 
hora\_final & time & No & 22:00:00 \\ \hline 
inicio\_semestre1 & date & No &  \\ \hline 
fin\_semestre1 & date & No &  \\ \hline 
inicio\_semestre2 & date & No &  \\ \hline 
fin\_semestre2 & date & No &  \\ \hline 
inicio\_examenes\_enero & date & No &  \\ \hline 
fin\_examenes\_enero & date & No &  \\ \hline 
inicio\_examenes\_junio & date & No &  \\ \hline 
fin\_examenes\_junio & date & No &  \\ \hline 
inicio\_examenes\_sept & date & No &  \\ \hline 
fin\_examenes\_sept & date & No &  \\ \hline 
 \end{longtable}

%
% Estructura: cursos
%
 \begin{longtable}{|l|c|c|c|} 
 \caption{Estructura de la tabla cursos} \label{tab:cursos-structure} \\
 \hline \multicolumn{1}{|c|}{\textbf{Columna}} & \multicolumn{1}{|c|}{\textbf{Tipo}} & \multicolumn{1}{|c|}{\textbf{Nulo}} & \multicolumn{1}{|c|}{\textbf{Predeterminado}} \\ \hline \hline
\endfirsthead
 \caption{Estructura de la tabla cursos (continúa)} \\ 
 \hline \multicolumn{1}{|c|}{\textbf{Columna}} & \multicolumn{1}{|c|}{\textbf{Tipo}} & \multicolumn{1}{|c|}{\textbf{Nulo}} & \multicolumn{1}{|c|}{\textbf{Predeterminado}} \\ \hline \hline \endhead \endfoot 
\textbf{\textit{id}} & int(11) & No &  \\ \hline 
num\_semanas\_teoria & int(11) & No &  \\ \hline 
num\_semanas\_semestre1 & int(11) & No & 0 \\ \hline 
num\_semanas\_semestre2 & int(11) & No & 0 \\ \hline 
horas\_por\_credito & int(11) & No &  \\ \hline 
slot\_minimo & bigint(20) & No & 30 \\ \hline 
hora\_inicial & time & No & 09:00:00 \\ \hline 
hora\_final & time & No & 22:00:00 \\ \hline 
inicio\_semestre1 & date & No &  \\ \hline 
fin\_semestre1 & date & No &  \\ \hline 
inicio\_semestre2 & date & No &  \\ \hline 
fin\_semestre2 & date & No &  \\ \hline 
inicio\_examenes\_enero & date & No &  \\ \hline 
fin\_examenes\_enero & date & No &  \\ \hline 
inicio\_examenes\_junio & date & No &  \\ \hline 
fin\_examenes\_junio & date & No &  \\ \hline 
inicio\_examenes\_sept & date & No &  \\ \hline 
fin\_examenes\_sept & date & No &  \\ \hline 
 \end{longtable}

%
% Estructura: cursos_compartidos
%
 \begin{longtable}{|l|c|c|c|l|} 
 \caption{Estructura de la tabla cursos\_compartidos} \label{tab:cursos_compartidos-structure} \\
 \hline \multicolumn{1}{|c|}{\textbf{Columna}} & \multicolumn{1}{|c|}{\textbf{Tipo}} & \multicolumn{1}{|c|}{\textbf{Nulo}} & \multicolumn{1}{|c|}{\textbf{Predeterminado}} & \multicolumn{1}{|c|}{\textbf{Enlaces a}} \\ \hline \hline
\endfirsthead
 \caption{Estructura de la tabla cursos\_compartidos (continúa)} \\ 
 \hline \multicolumn{1}{|c|}{\textbf{Columna}} & \multicolumn{1}{|c|}{\textbf{Tipo}} & \multicolumn{1}{|c|}{\textbf{Nulo}} & \multicolumn{1}{|c|}{\textbf{Predeterminado}} & \multicolumn{1}{|c|}{\textbf{Enlaces a}} \\ \hline \hline \endhead \endfoot 
\textbf{\textit{id\_plandocente}} & int(11) & No &  & planesdocentes (id) \\ \hline 
\textbf{\textit{num\_curso\_compartido}} & int(11) & No &  &  \\ \hline 
 \end{longtable}

%
% Estructura: cursos_compartidos
%
 \begin{longtable}{|l|c|c|c|l|} 
 \caption{Estructura de la tabla cursos\_compartidos} \label{tab:cursos_compartidos-structure} \\
 \hline \multicolumn{1}{|c|}{\textbf{Columna}} & \multicolumn{1}{|c|}{\textbf{Tipo}} & \multicolumn{1}{|c|}{\textbf{Nulo}} & \multicolumn{1}{|c|}{\textbf{Predeterminado}} & \multicolumn{1}{|c|}{\textbf{Enlaces a}} \\ \hline \hline
\endfirsthead
 \caption{Estructura de la tabla cursos\_compartidos (continúa)} \\ 
 \hline \multicolumn{1}{|c|}{\textbf{Columna}} & \multicolumn{1}{|c|}{\textbf{Tipo}} & \multicolumn{1}{|c|}{\textbf{Nulo}} & \multicolumn{1}{|c|}{\textbf{Predeterminado}} & \multicolumn{1}{|c|}{\textbf{Enlaces a}} \\ \hline \hline \endhead \endfoot 
\textbf{\textit{id\_plandocente}} & int(11) & No &  & planesdocentes (id) \\ \hline 
\textbf{\textit{num\_curso\_compartido}} & int(11) & No &  &  \\ \hline 
 \end{longtable}

%
% Estructura: eventos
%
 \begin{longtable}{|l|c|c|c|l|} 
 \caption{Estructura de la tabla eventos} \label{tab:eventos-structure} \\
 \hline \multicolumn{1}{|c|}{\textbf{Columna}} & \multicolumn{1}{|c|}{\textbf{Tipo}} & \multicolumn{1}{|c|}{\textbf{Nulo}} & \multicolumn{1}{|c|}{\textbf{Predeterminado}} & \multicolumn{1}{|c|}{\textbf{Enlaces a}} \\ \hline \hline
\endfirsthead
 \caption{Estructura de la tabla eventos (continúa)} \\ 
 \hline \multicolumn{1}{|c|}{\textbf{Columna}} & \multicolumn{1}{|c|}{\textbf{Tipo}} & \multicolumn{1}{|c|}{\textbf{Nulo}} & \multicolumn{1}{|c|}{\textbf{Predeterminado}} & \multicolumn{1}{|c|}{\textbf{Enlaces a}} \\ \hline \hline \endhead \endfoot 
\textbf{\textit{id}} & int(11) & No &  &  \\ \hline 
nombre\_evento & varchar(255) & No &  &  \\ \hline 
tipo\_evento & varchar(255) & No &  &  \\ \hline 
fecha\_individual & tinyint(1) & No &  &  \\ \hline 
fecha\_inicial & date & No &  &  \\ \hline 
fecha\_final & date & No &  &  \\ \hline 
curso\_id & int(11) & No &  & cursos (id) \\ \hline 
 \end{longtable}

%
% Estructura: eventos
%
 \begin{longtable}{|l|c|c|c|l|} 
 \caption{Estructura de la tabla eventos} \label{tab:eventos-structure} \\
 \hline \multicolumn{1}{|c|}{\textbf{Columna}} & \multicolumn{1}{|c|}{\textbf{Tipo}} & \multicolumn{1}{|c|}{\textbf{Nulo}} & \multicolumn{1}{|c|}{\textbf{Predeterminado}} & \multicolumn{1}{|c|}{\textbf{Enlaces a}} \\ \hline \hline
\endfirsthead
 \caption{Estructura de la tabla eventos (continúa)} \\ 
 \hline \multicolumn{1}{|c|}{\textbf{Columna}} & \multicolumn{1}{|c|}{\textbf{Tipo}} & \multicolumn{1}{|c|}{\textbf{Nulo}} & \multicolumn{1}{|c|}{\textbf{Predeterminado}} & \multicolumn{1}{|c|}{\textbf{Enlaces a}} \\ \hline \hline \endhead \endfoot 
\textbf{\textit{id}} & int(11) & No &  &  \\ \hline 
nombre\_evento & varchar(255) & No &  &  \\ \hline 
tipo\_evento & varchar(255) & No &  &  \\ \hline 
fecha\_individual & tinyint(1) & No &  &  \\ \hline 
fecha\_inicial & date & No &  &  \\ \hline 
fecha\_final & date & No &  &  \\ \hline 
curso\_id & int(11) & No &  & cursos (id) \\ \hline 
 \end{longtable}

%
% Estructura: horarios
%
 \begin{longtable}{|l|c|c|c|l|} 
 \caption{Estructura de la tabla horarios} \label{tab:horarios-structure} \\
 \hline \multicolumn{1}{|c|}{\textbf{Columna}} & \multicolumn{1}{|c|}{\textbf{Tipo}} & \multicolumn{1}{|c|}{\textbf{Nulo}} & \multicolumn{1}{|c|}{\textbf{Predeterminado}} & \multicolumn{1}{|c|}{\textbf{Enlaces a}} \\ \hline \hline
\endfirsthead
 \caption{Estructura de la tabla horarios (continúa)} \\ 
 \hline \multicolumn{1}{|c|}{\textbf{Columna}} & \multicolumn{1}{|c|}{\textbf{Tipo}} & \multicolumn{1}{|c|}{\textbf{Nulo}} & \multicolumn{1}{|c|}{\textbf{Predeterminado}} & \multicolumn{1}{|c|}{\textbf{Enlaces a}} \\ \hline \hline \endhead \endfoot 
\textbf{\textit{id}} & int(11) & No &  &  \\ \hline 
id\_curso & int(11) & No &  & cursos (id) \\ \hline 
id\_titulacion & int(11) & No &  & titulaciones (id) \\ \hline 
num\_curso\_titulacion & int(11) & No &  &  \\ \hline 
semestre & varchar(255) & No &  &  \\ \hline 
num\_grupo\_titulacion & int(11) & No &  &  \\ \hline 
num\_semana & int(10)  & No &  &  \\ \hline 
 \end{longtable}

%
% Estructura: horarios
%
 \begin{longtable}{|l|c|c|c|l|} 
 \caption{Estructura de la tabla horarios} \label{tab:horarios-structure} \\
 \hline \multicolumn{1}{|c|}{\textbf{Columna}} & \multicolumn{1}{|c|}{\textbf{Tipo}} & \multicolumn{1}{|c|}{\textbf{Nulo}} & \multicolumn{1}{|c|}{\textbf{Predeterminado}} & \multicolumn{1}{|c|}{\textbf{Enlaces a}} \\ \hline \hline
\endfirsthead
 \caption{Estructura de la tabla horarios (continúa)} \\ 
 \hline \multicolumn{1}{|c|}{\textbf{Columna}} & \multicolumn{1}{|c|}{\textbf{Tipo}} & \multicolumn{1}{|c|}{\textbf{Nulo}} & \multicolumn{1}{|c|}{\textbf{Predeterminado}} & \multicolumn{1}{|c|}{\textbf{Enlaces a}} \\ \hline \hline \endhead \endfoot 
\textbf{\textit{id}} & int(11) & No &  &  \\ \hline 
id\_curso & int(11) & No &  & cursos (id) \\ \hline 
id\_titulacion & int(11) & No &  & titulaciones (id) \\ \hline 
num\_curso\_titulacion & int(11) & No &  &  \\ \hline 
semestre & varchar(255) & No &  &  \\ \hline 
num\_grupo\_titulacion & int(11) & No &  &  \\ \hline 
num\_semana & int(10)  & No &  &  \\ \hline 
 \end{longtable}

%
% Estructura: horario_reference
%
 \begin{longtable}{|l|c|c|c|} 
 \caption{Estructura de la tabla horario\_reference} \label{tab:horario_reference-structure} \\
 \hline \multicolumn{1}{|c|}{\textbf{Columna}} & \multicolumn{1}{|c|}{\textbf{Tipo}} & \multicolumn{1}{|c|}{\textbf{Nulo}} & \multicolumn{1}{|c|}{\textbf{Predeterminado}} \\ \hline \hline
\endfirsthead
 \caption{Estructura de la tabla horario\_reference (continúa)} \\ 
 \hline \multicolumn{1}{|c|}{\textbf{Columna}} & \multicolumn{1}{|c|}{\textbf{Tipo}} & \multicolumn{1}{|c|}{\textbf{Nulo}} & \multicolumn{1}{|c|}{\textbf{Predeterminado}} \\ \hline \hline \endhead \endfoot 
\textbf{\textit{id\_tipo}} & bigint(20)  & No &  \\ \hline 
\textbf{\textit{id\_teoria}} & bigint(20)  & No &  \\ \hline 
 \end{longtable}

%
% Estructura: horario_reference
%
 \begin{longtable}{|l|c|c|c|} 
 \caption{Estructura de la tabla horario\_reference} \label{tab:horario_reference-structure} \\
 \hline \multicolumn{1}{|c|}{\textbf{Columna}} & \multicolumn{1}{|c|}{\textbf{Tipo}} & \multicolumn{1}{|c|}{\textbf{Nulo}} & \multicolumn{1}{|c|}{\textbf{Predeterminado}} \\ \hline \hline
\endfirsthead
 \caption{Estructura de la tabla horario\_reference (continúa)} \\ 
 \hline \multicolumn{1}{|c|}{\textbf{Columna}} & \multicolumn{1}{|c|}{\textbf{Tipo}} & \multicolumn{1}{|c|}{\textbf{Nulo}} & \multicolumn{1}{|c|}{\textbf{Predeterminado}} \\ \hline \hline \endhead \endfoot 
\textbf{\textit{id\_tipo}} & bigint(20)  & No &  \\ \hline 
\textbf{\textit{id\_teoria}} & bigint(20)  & No &  \\ \hline 
 \end{longtable}

%
% Estructura: lineashorarios
%
 \begin{longtable}{|l|c|c|c|l|} 
 \caption{Estructura de la tabla lineashorarios} \label{tab:lineashorarios-structure} \\
 \hline \multicolumn{1}{|c|}{\textbf{Columna}} & \multicolumn{1}{|c|}{\textbf{Tipo}} & \multicolumn{1}{|c|}{\textbf{Nulo}} & \multicolumn{1}{|c|}{\textbf{Predeterminado}} & \multicolumn{1}{|c|}{\textbf{Enlaces a}} \\ \hline \hline
\endfirsthead
 \caption{Estructura de la tabla lineashorarios (continúa)} \\ 
 \hline \multicolumn{1}{|c|}{\textbf{Columna}} & \multicolumn{1}{|c|}{\textbf{Tipo}} & \multicolumn{1}{|c|}{\textbf{Nulo}} & \multicolumn{1}{|c|}{\textbf{Predeterminado}} & \multicolumn{1}{|c|}{\textbf{Enlaces a}} \\ \hline \hline \endhead \endfoot 
\textbf{\textit{id}} & int(11) & No &  &  \\ \hline 
id\_horario & int(11) & No &  & horarios (id) \\ \hline 
id\_asignatura & int(11) & No &  & asignaturas (id) \\ \hline 
hora\_inicial & time & Sí & NULL &  \\ \hline 
hora\_final & time & Sí & NULL &  \\ \hline 
dia\_semana & tinyint(3)  & Sí & NULL &  \\ \hline 
id\_actividad & bigint(20)  & Sí & NULL & actividades (id) \\ \hline 
num\_grupo\_actividad & bigint(20)  & No &  &  \\ \hline 
slot\_minimo & float(18,2) & No &  &  \\ \hline 
color & varchar(7) & Sí & NULL &  \\ \hline 
id\_aula & bigint(20)  & Sí & NULL & aulas (id) \\ \hline 
 \end{longtable}

%
% Estructura: lineashorarios
%
 \begin{longtable}{|l|c|c|c|l|} 
 \caption{Estructura de la tabla lineashorarios} \label{tab:lineashorarios-structure} \\
 \hline \multicolumn{1}{|c|}{\textbf{Columna}} & \multicolumn{1}{|c|}{\textbf{Tipo}} & \multicolumn{1}{|c|}{\textbf{Nulo}} & \multicolumn{1}{|c|}{\textbf{Predeterminado}} & \multicolumn{1}{|c|}{\textbf{Enlaces a}} \\ \hline \hline
\endfirsthead
 \caption{Estructura de la tabla lineashorarios (continúa)} \\ 
 \hline \multicolumn{1}{|c|}{\textbf{Columna}} & \multicolumn{1}{|c|}{\textbf{Tipo}} & \multicolumn{1}{|c|}{\textbf{Nulo}} & \multicolumn{1}{|c|}{\textbf{Predeterminado}} & \multicolumn{1}{|c|}{\textbf{Enlaces a}} \\ \hline \hline \endhead \endfoot 
\textbf{\textit{id}} & int(11) & No &  &  \\ \hline 
id\_horario & int(11) & No &  & horarios (id) \\ \hline 
id\_asignatura & int(11) & No &  & asignaturas (id) \\ \hline 
hora\_inicial & time & Sí & NULL &  \\ \hline 
hora\_final & time & Sí & NULL &  \\ \hline 
dia\_semana & tinyint(3)  & Sí & NULL &  \\ \hline 
id\_actividad & bigint(20)  & Sí & NULL & actividades (id) \\ \hline 
num\_grupo\_actividad & bigint(20)  & No &  &  \\ \hline 
slot\_minimo & float(18,2) & No &  &  \\ \hline 
color & varchar(7) & Sí & NULL &  \\ \hline 
id\_aula & bigint(20)  & Sí & NULL & aulas (id) \\ \hline 
 \end{longtable}

%
% Estructura: planactividades
%
 \begin{longtable}{|l|c|c|c|l|} 
 \caption{Estructura de la tabla planactividades} \label{tab:planactividades-structure} \\
 \hline \multicolumn{1}{|c|}{\textbf{Columna}} & \multicolumn{1}{|c|}{\textbf{Tipo}} & \multicolumn{1}{|c|}{\textbf{Nulo}} & \multicolumn{1}{|c|}{\textbf{Predeterminado}} & \multicolumn{1}{|c|}{\textbf{Enlaces a}} \\ \hline \hline
\endfirsthead
 \caption{Estructura de la tabla planactividades (continúa)} \\ 
 \hline \multicolumn{1}{|c|}{\textbf{Columna}} & \multicolumn{1}{|c|}{\textbf{Tipo}} & \multicolumn{1}{|c|}{\textbf{Nulo}} & \multicolumn{1}{|c|}{\textbf{Predeterminado}} & \multicolumn{1}{|c|}{\textbf{Enlaces a}} \\ \hline \hline \endhead \endfoot 
\textbf{\textit{id}} & int(11) & No &  &  \\ \hline 
id\_plandocente & int(11) & No &  & planesdocentes (id) \\ \hline 
id\_actividad & bigint(20)  & No &  & actividades (id) \\ \hline 
horas & int(10)  & No & 0 &  \\ \hline 
grupos & int(10)  & No & 0 &  \\ \hline 
horas\_semanales & int(10)  & No & 0 &  \\ \hline 
alternas & tinyint(1) & Sí & NULL &  \\ \hline 
 \end{longtable}

%
% Estructura: planactividades
%
 \begin{longtable}{|l|c|c|c|l|} 
 \caption{Estructura de la tabla planactividades} \label{tab:planactividades-structure} \\
 \hline \multicolumn{1}{|c|}{\textbf{Columna}} & \multicolumn{1}{|c|}{\textbf{Tipo}} & \multicolumn{1}{|c|}{\textbf{Nulo}} & \multicolumn{1}{|c|}{\textbf{Predeterminado}} & \multicolumn{1}{|c|}{\textbf{Enlaces a}} \\ \hline \hline
\endfirsthead
 \caption{Estructura de la tabla planactividades (continúa)} \\ 
 \hline \multicolumn{1}{|c|}{\textbf{Columna}} & \multicolumn{1}{|c|}{\textbf{Tipo}} & \multicolumn{1}{|c|}{\textbf{Nulo}} & \multicolumn{1}{|c|}{\textbf{Predeterminado}} & \multicolumn{1}{|c|}{\textbf{Enlaces a}} \\ \hline \hline \endhead \endfoot 
\textbf{\textit{id}} & int(11) & No &  &  \\ \hline 
id\_plandocente & int(11) & No &  & planesdocentes (id) \\ \hline 
id\_actividad & bigint(20)  & No &  & actividades (id) \\ \hline 
horas & int(10)  & No & 0 &  \\ \hline 
grupos & int(10)  & No & 0 &  \\ \hline 
horas\_semanales & int(10)  & No & 0 &  \\ \hline 
alternas & tinyint(1) & Sí & NULL &  \\ \hline 
 \end{longtable}

%
% Estructura: planesdocentes
%
 \begin{longtable}{|l|c|c|c|l|} 
 \caption{Estructura de la tabla planesdocentes} \label{tab:planesdocentes-structure} \\
 \hline \multicolumn{1}{|c|}{\textbf{Columna}} & \multicolumn{1}{|c|}{\textbf{Tipo}} & \multicolumn{1}{|c|}{\textbf{Nulo}} & \multicolumn{1}{|c|}{\textbf{Predeterminado}} & \multicolumn{1}{|c|}{\textbf{Enlaces a}} \\ \hline \hline
\endfirsthead
 \caption{Estructura de la tabla planesdocentes (continúa)} \\ 
 \hline \multicolumn{1}{|c|}{\textbf{Columna}} & \multicolumn{1}{|c|}{\textbf{Tipo}} & \multicolumn{1}{|c|}{\textbf{Nulo}} & \multicolumn{1}{|c|}{\textbf{Predeterminado}} & \multicolumn{1}{|c|}{\textbf{Enlaces a}} \\ \hline \hline \endhead \endfoot 
\textbf{\textit{id}} & int(11) & No &  &  \\ \hline 
id\_asignatura & int(11) & No &  & asignaturas (id) \\ \hline 
id\_curso & int(11) & No &  & cursos (id) \\ \hline 
 \end{longtable}

%
% Estructura: planesdocentes
%
 \begin{longtable}{|l|c|c|c|l|} 
 \caption{Estructura de la tabla planesdocentes} \label{tab:planesdocentes-structure} \\
 \hline \multicolumn{1}{|c|}{\textbf{Columna}} & \multicolumn{1}{|c|}{\textbf{Tipo}} & \multicolumn{1}{|c|}{\textbf{Nulo}} & \multicolumn{1}{|c|}{\textbf{Predeterminado}} & \multicolumn{1}{|c|}{\textbf{Enlaces a}} \\ \hline \hline
\endfirsthead
 \caption{Estructura de la tabla planesdocentes (continúa)} \\ 
 \hline \multicolumn{1}{|c|}{\textbf{Columna}} & \multicolumn{1}{|c|}{\textbf{Tipo}} & \multicolumn{1}{|c|}{\textbf{Nulo}} & \multicolumn{1}{|c|}{\textbf{Predeterminado}} & \multicolumn{1}{|c|}{\textbf{Enlaces a}} \\ \hline \hline \endhead \endfoot 
\textbf{\textit{id}} & int(11) & No &  &  \\ \hline 
id\_asignatura & int(11) & No &  & asignaturas (id) \\ \hline 
id\_curso & int(11) & No &  & cursos (id) \\ \hline 
 \end{longtable}

%
% Estructura: titulaciones
%
 \begin{longtable}{|l|c|c|c|} 
 \caption{Estructura de la tabla titulaciones} \label{tab:titulaciones-structure} \\
 \hline \multicolumn{1}{|c|}{\textbf{Columna}} & \multicolumn{1}{|c|}{\textbf{Tipo}} & \multicolumn{1}{|c|}{\textbf{Nulo}} & \multicolumn{1}{|c|}{\textbf{Predeterminado}} \\ \hline \hline
\endfirsthead
 \caption{Estructura de la tabla titulaciones (continúa)} \\ 
 \hline \multicolumn{1}{|c|}{\textbf{Columna}} & \multicolumn{1}{|c|}{\textbf{Tipo}} & \multicolumn{1}{|c|}{\textbf{Nulo}} & \multicolumn{1}{|c|}{\textbf{Predeterminado}} \\ \hline \hline \endhead \endfoot 
\textbf{\textit{id}} & int(11) & No &  \\ \hline 
\textbf{codigo} & varchar(4) & No &  \\ \hline 
\textbf{nombre} & varchar(200) & No &  \\ \hline 
creditos & int(10)  & No &  \\ \hline 
num\_cursos & int(10)  & No &  \\ \hline 
 \end{longtable}

%
% Estructura: titulaciones
%
 \begin{longtable}{|l|c|c|c|} 
 \caption{Estructura de la tabla titulaciones} \label{tab:titulaciones-structure} \\
 \hline \multicolumn{1}{|c|}{\textbf{Columna}} & \multicolumn{1}{|c|}{\textbf{Tipo}} & \multicolumn{1}{|c|}{\textbf{Nulo}} & \multicolumn{1}{|c|}{\textbf{Predeterminado}} \\ \hline \hline
\endfirsthead
 \caption{Estructura de la tabla titulaciones (continúa)} \\ 
 \hline \multicolumn{1}{|c|}{\textbf{Columna}} & \multicolumn{1}{|c|}{\textbf{Tipo}} & \multicolumn{1}{|c|}{\textbf{Nulo}} & \multicolumn{1}{|c|}{\textbf{Predeterminado}} \\ \hline \hline \endhead \endfoot 
\textbf{\textit{id}} & int(11) & No &  \\ \hline 
\textbf{codigo} & varchar(4) & No &  \\ \hline 
\textbf{nombre} & varchar(200) & No &  \\ \hline 
creditos & int(10)  & No &  \\ \hline 
num\_cursos & int(10)  & No &  \\ \hline 
 \end{longtable}

%
% Estructura: users
%
 \begin{longtable}{|l|c|c|c|l|} 
 \caption{Estructura de la tabla users} \label{tab:users-structure} \\
 \hline \multicolumn{1}{|c|}{\textbf{Columna}} & \multicolumn{1}{|c|}{\textbf{Tipo}} & \multicolumn{1}{|c|}{\textbf{Nulo}} & \multicolumn{1}{|c|}{\textbf{Predeterminado}} & \multicolumn{1}{|c|}{\textbf{Enlaces a}} \\ \hline \hline
\endfirsthead
 \caption{Estructura de la tabla users (continúa)} \\ 
 \hline \multicolumn{1}{|c|}{\textbf{Columna}} & \multicolumn{1}{|c|}{\textbf{Tipo}} & \multicolumn{1}{|c|}{\textbf{Nulo}} & \multicolumn{1}{|c|}{\textbf{Predeterminado}} & \multicolumn{1}{|c|}{\textbf{Enlaces a}} \\ \hline \hline \endhead \endfoot 
\textbf{\textit{id}} & int(11) & No &  &  \\ \hline 
password & varchar(255) & No &  &  \\ \hline 
nombre & varchar(50) & No &  &  \\ \hline 
apellidos & varchar(50) & No &  &  \\ \hline 
\textbf{dni} & varchar(9) & No &  &  \\ \hline 
\textbf{email} & varchar(30) & No &  &  \\ \hline 
id\_titulacion & int(11) & No &  & titulaciones (id) \\ \hline 
level & int(10)  & No & 0 &  \\ \hline 
 \end{longtable}

%
% Estructura: users
%
 \begin{longtable}{|l|c|c|c|l|} 
 \caption{Estructura de la tabla users} \label{tab:users-structure} \\
 \hline \multicolumn{1}{|c|}{\textbf{Columna}} & \multicolumn{1}{|c|}{\textbf{Tipo}} & \multicolumn{1}{|c|}{\textbf{Nulo}} & \multicolumn{1}{|c|}{\textbf{Predeterminado}} & \multicolumn{1}{|c|}{\textbf{Enlaces a}} \\ \hline \hline
\endfirsthead
 \caption{Estructura de la tabla users (continúa)} \\ 
 \hline \multicolumn{1}{|c|}{\textbf{Columna}} & \multicolumn{1}{|c|}{\textbf{Tipo}} & \multicolumn{1}{|c|}{\textbf{Nulo}} & \multicolumn{1}{|c|}{\textbf{Predeterminado}} & \multicolumn{1}{|c|}{\textbf{Enlaces a}} \\ \hline \hline \endhead \endfoot 
\textbf{\textit{id}} & int(11) & No &  &  \\ \hline 
password & varchar(255) & No &  &  \\ \hline 
nombre & varchar(50) & No &  &  \\ \hline 
apellidos & varchar(50) & No &  &  \\ \hline 
\textbf{dni} & varchar(9) & No &  &  \\ \hline 
\textbf{email} & varchar(30) & No &  &  \\ \hline 
id\_titulacion & int(11) & No &  & titulaciones (id) \\ \hline 
level & int(10)  & No & 0 &  \\ \hline 
 \end{longtable}

\chapter{Implementación}
Llegada la etapa de implementación, hay que transformar a código lo analizado y diseñado en etapas anteriores. La petición que se nos hacía era que fuera una aplicación web, para ello se toman una serie de decisiones en cuanto a herramientas y lenguajes que se detallarán a continuación.

\section{Lenguajes}

Durante la carrera hay muy poca formación en cuanto a desarrollo web se refiere, solo una asignatura, que toca por encima el desarrollo en el lado del servidor con algunas nociones de PHP. La razón por la cual se elige entonces PHP para el desarrollo del proyecto es ampliar los conocimientos en dicho lenguaje, además de ser uno de los más usados, lo que implica que será sencillo encontrar solución a los posibles problemas que vayan apareciendo debido a la amplia documentación que hay disponible.\\

PHP es un lenguaje fácil de aprender, dado que posee mucha similitud con lenguajes como C, que si han sido aprendidos durante la carrera.\\

Tras comenzar el aprendizaje del lenguaje, surge el problema de que para hacer el desarrollo mínimamente organizado habría que implementar una serie de clases base que nos ayudaran con la implementación del sistema. Esto implica un trabajo tedioso que puede ahorrarse utilizando algún framework que nos proporcione herramientas que hagan mucho más fácil el trabajo. Es por ello que se decide utilizar {\em CodeIgniter}.
\\

{\em CodeIgniter} es un framework escrito en PHP, basado en el patrón arquitectónico MVC, a diferencia de otros frameworks existentes para PHP, es una herramienta realmente ligera, poco intrusiva y que facilita muchísimo el trabajo. Para ello pone a disposición algunas librerías y helpers, aumentando notablemente la productividad del desarrollador. \\

Otra gran ventaja de {\em CodeIgniter} es su documentación y su comunidad de desarrolladores, además de la Wiki en la que los usuarios van publicando plugins que pueden ser útiles para nuestro trabajo.\\

Uno de los problemas que encontramos con {\em CodeIgniter} es que a diferencia de otros frameworks MVC, no disponían de un ORM, es decir un sistema de persistencia que nos permitiera obtener elementos de la base de datos y transformarlos en objetos de nuestro sistema. Para ello se decide utilizar {\em Doctrine}, un ORM fácilmente integrable con {\em CodeIgniter}, y que nos permitía tener una base de datos virtual sobre nuestro sistema.\\

Pero la parte del servidor no es la única que hay que implementar, además necesitamos un lenguaje para las vistas. Para ello se elige XHTML, lenguaje de marcado ampliamente usado y recomendado por la W3C.\\

Otro objetivo es tener una buena interacción con el usuario, para ello es necesario el uso de un lenguaje que pueda interactuar con el DOM y modificarlo sin necesidad de pasar por el servidor, para ello elegimos JavaScript, soportado por la inmensa mayoría de navegadores. Se hace necesario para mejorar la interacción, el uso de una librería que facilite el trabajo con este lenguaje, para ello elegimos {\em jQuery}, la librería de JavaScript más usada, con multitud de plugins y una documentación muy bien estructurada.\\

\section{Extensiones y librerías}

El proyecto requiere la realización de algunas tareas complejas, como la generación de informes y la configuración de horarios, para ello se han utilizado una serie de librerías para obtener algunas funcionalidades difíciles de implementar. Éstas son:

\begin{itemize}

\item {\bf FullCalendar} - Plugin para {\em jQuery} que permite mostrar un calendario interactivo, con una API a disposición del desarrollador para responder a multitud de eventos, lo que permite una máxima personalización.
\item {\bf jQueryUI} - Plugin para {\em jQuery} con una gran variedad de widgets para mejorar la interacción con el usuario.
\item {\bf Farbtastic} - Plugin para {\em jQuery} que nos permite integrar un selector de color en una página.
\item {\bf FPDF} - Librería para PHP para la generación de documentos PDF.
\item {\bf PHPMailer} - Librería para PHP que facilita el envío de correos electrónicos.
\end{itemize}

\section{Herramientas utilizadas}

Para el desarrollo del proyecto se hace necesario el uso de una serie de herramientas, como editores de código, sistemas de control de versiones, etc. A continuación se detallarán todas las herramientas usadas en este proyecto.\\

La herramienta principal que se ha utilizado ha sido un IDE, en este caso {\em NetBeans} en su versión PHP. {\em NetBeans} es un entorno escrito en Java y pensado en un principio para desarrollar en este mismo lenguaje, pero conforme ha avanzado el tiempo se ha ido ampliando a más lenguajes, como por ejemplo PHP. La integración con este último es perfecta, proporcionando útiles herramientas como el autocompletado.\\

Para la detección de errores se hace casi obligado el uso de un {\em debugger}, en este caso hemos usado {\em XDebugger} que viene integrado en {\em NetBeans}, permitiendo utilizar puntos de ruptura en el código para comprobar el estado del sistema en un momento dado.\\

Para el despliegue de la aplicación se ha utilizado un entorno compuesto por un servidor {\em Apache}, base de datos {\em MySQL} y el intérprete de PHP, todo ello sobre un sistema GNU/Linux.\\

Otra herramienta utilizada que facilita el trabajo enormemente ha sido {\em Git}. {\em Git} es un sistema de control de versiones que facilita el desarrollo colaborativo y el mantenimiento de un software, versionando todos los cambios que se vayan produciendo en el código. Esto permite que si queremos volver a una versión anterior del sistema podamos hacerlo sin problema alguno, además de la creación de ramas de desarrollo, pudiendo fusionar ramas sin problema alguno.\\

\section{Detalles de la implementación de la arquitectura del sistema}

\subsection{Capa modelo}

Como hemos dicho, se ha utilizado la librería Doctrine como ORM. Esta librería sustituye completamente a la capa modelo del MVC de {\em CodeIgniter}, para ello, cada tabla de la base de datos se corresponde con un modelo. Para construir un modelo definimos sus atributos en una clase, además de sus relaciones, de esta forma, {\em Doctrine} generará automáticamente las tablas de la base de datos a partir de los modelos, aplicando todas las reglas de integridad definidas.\\

Un ejemplo de un modelo es el siguiente, que corresponde al de la tabla titulaciones:

\begin{lstlisting}[style=PHP]
class Titulacion extends Doctrine_Record
{
  public function setTableDefinition()
  {
    $this->setTableName('titulaciones');
    $this->hasColumn('id', 'integer', 4, array(
					       'type' => 'integer',
					       'length' => 4,
					       'primary' => true,
					       'autoincrement' => true,
					       'unsigned' => false,
					       'fixed' => false
					       ));
    $this->hasColumn('codigo', 'string', 4, array(
    					  'minlength' => 4,	
						  'length' => 4,
						  'notnull',
						  'notblank',						  
						  'unique',
						  'regexp' => '/[0-9]{4}/',
						  'unsigned' => false
						  ));
    $this->hasColumn('nombre', 'string', 200, array(
						    'type' => 'string',
						    'minlength' => 5,
						    'length' => 200,
						    'notnull' => true,
						    'unique' => true,
						    'notblank' => true,
						    'unsigned' => false
						    ));
    $this->hasColumn('creditos', 'integer', 4, array(
						     'type' => 'integer',
						     'length' => 4,
						     'unsigned' => true,
						     'notnull' => true,
						     'notblank' => true
						     ));
    $this->hasColumn('num_cursos', 'integer', 4, array(
                                'type' => 'integer',
                                'length' => 4,
                                'unsigned' => true,
                                'notnull' => true,
                                'notblank' => true
                             ));
  }

  public function getPlanificacion($id_curso)
  {
        $asignaturas = $this->asignaturas;
        $salida_total = array();
        foreach($asignaturas as $asignatura)
        {
            $q = Doctrine_Query::create()->select('c.*, p.*, a.descripcion')
                    ->from('PlanActividad p')
                    ->innerJoin('p.plandocente c')
                    ->innerJoin('p.actividad a')
                    ->where('c.id_curso = ? AND c.id_asignatura = ?', array($id_curso, $asignatura->id));
            $resultado = $q->execute();
            $salida = array();
            $salida[0] =  $asignatura->nombre;
            foreach($resultado as $actividad)
            {
                $salida[$actividad->id_actividad] = array($actividad->horas, $actividad->grupos, $actividad->horas_semanales);
            }
            $salida_total[] = $salida;
      }
      
      return $salida_total;
  }
  
  public function setUp()
  {
    parent::setUp();
    $this->hasMany('Asignatura as asignaturas', array('local' => 'id', 'foreign' => 'titulacion_id', 'onDelete' => 'CASCADE'));
  }
}
\end{lstlisting}

Como se puede ver en el código, se definen todos los atributos y sus reglas de integridad. En el método setUp definimos las relaciones, y además podemos definir nuestros propios métodos que devuelvan información personalizada de la base de datos.

\subsection{Capa controlador}

La capa controlador es la principal del sistema, la que recibe las peticiones, las procesa y las pasa a la vista para renderizar la página pedida. Al igual que en el modelo existe uno por cada subsistema.\\

Un controlador está compuesto de acciones, y cada una de ellas es llamada desde una URL del navegador, el controlador procesa la petición y realiza la lógica necesaria antes de devolver una respuesta, la estructura de un controlador es la siguiente:

\begin{lstlisting}[style=PHP]
class Titulaciones extends MY_Controller {

    function __construct() {
        parent::__construct();
        $this->titulaciones_table = Doctrine::getTable('Titulacion');
        $this->asignaturas_table = Doctrine::getTable('Asignatura');
        $this->layout = '';
        $this->notices = '';
        $this->alerts = '';
        $this->_filter(array('add', 'create', 'delete', 'edit', 'update', 'show_informes', 'show', 'exportar_planificacion'), array($this, 'authenticate'), 1); 
    }

    public function index() {

        $titulaciones = $this -> titulaciones_table -> findAll();

        //Conseguimos los items mediante el modelo
        $data['titulaciones'] = $titulaciones;
        $data['page_title'] = 'INDEX TITULACIONES';
        $data['enlace'] = 'titulaciones/show/';
        if($this -> input -> post('js') == '1') {
            unset($this -> layout);
            $this -> load -> view('titulaciones/_titulaciones', $data);
        } else {
            //Mostramos
            $this -> load -> view('titulaciones/index', $data);
        }
    }
\end{lstlisting}

Tenemos un constructor que es invocado en cada petición e inicializa algunos parámetros necesarios, y luego tenemos una acción, en este caso index, que busca todas las titulaciones en la base de datos y las pasa a la vista.

\subsection{Capa vista}

Esta capa es la que realmente ve el usuario, por tanto no es menos importante que las demás, aquí se utiliza un {\em layout} o plantilla por defecto que renderiza la parte que es común a todas las páginas, de forma que no se repite código en cada una de las vistas. Esta estructura común es la siguiente:

\begin{figure}[H] 
  \label{captura-layout} 
	\begin{center}
    \includegraphics[scale=0.40]{./layout.png}
  \end{center}
\caption{Captura de la estructura de una página}
\end{figure}

En esta estructura tenemos un menú a la izquierda, una cabecera y un pié de página, además de un panel sobre el menú que indica el usuario que está conectado y permite su salida del sistema.\\

Todas las páginas siguen un estilo similar, por ejemplo una página que contiene una tabla de elementos, como esta con el listado de titulaciones:

\begin{figure}[H] 
  \label{captura-index-titulaciones} 
	\begin{center}
    \includegraphics[scale=0.57]{./index-titulaciones.png}
  \end{center}
\caption{Captura del listado de titulaciones}
\end{figure}

La parte central del sistema es la gestión de horarios, para ello es necesaria una interacción sencilla con el usuario a la hora de construirlos y que no se convierta en una labor tediosa de realizar. Por ello se pensó que lo más sencillo sería arrastrar las asignaturas al lugar deseado en el horario, es decir, lo más intuitivo posible, esto es lo que veríamos en una de las páginas de configuración de horarios:

\begin{figure}[H] 
  \label{captura-horarios} 
	\begin{center}
    \includegraphics[scale=0.57]{./edit-horario.png}
  \end{center}
\caption{Captura de página de horarios}
\end{figure}

Tenemos tres partes diferenciadas en esta vista, un cuadro superior con las asignaturas aun no asignadas, que se podrán arrastrar al horario. Otra central con el horario de ese grupo, y otro bloque en la parte inferior en el que podemos ver la ocupación del aula que seleccionemos.\\

A la hora de asignar los slots en el horario hay que tener en cuenta diversos factores, como por ejemplo si el aula está ocupada, o si las asignaturas son solapables, para ello cada vez que se arrastra una asignatura al horario se hace una comprobación mediante la llamada a un servicio, devolviendo el resultado y dependiendo de si es satisfactorio o no, dejar la asignatura en su lugar o deshacer el cambio. Para deshacer el cambio, la API del plugin {\em FullCalendar} proporciona una función para invertir el proceso de la última acción realizada, eso hace el trabajo algo más sencillo.


\chapter{Pruebas}
Es importante en un sistema como este, y cualquiera, que todo funcione correctamente, para ello es necesario hacer una serie de pruebas para verificar el correcto funcionamiento de la aplicación y que los requisitos se cumplen tal y como fueron especificados.\\

\section{Pruebas sobre los datos}

Para verificar el correcto funcionamiento de la aplicación se realizaron una serie de pruebas que consistieron en las siguientes:

\begin{itemize}
\item {\bf Pruebas de caja negra individual} . Cada vez que se desarrollaba una nueva clase se realizaban pruebas de caja negra, esto significa que lo único que se tiene en cuenta son los datos de entrada y la salida producida, ignorando lo que pasa internamente en el sistema.

\item{\bf Pruebas de integración de subsistema} . Al terminar un subsistema se hace necesario probar que la interacción entre todos los elementos de éste es correcta. Estas pruebas nos permiten ver si el intercambio de datos entre Modelo, Vista y Controlador es correcto.

\item{\bf Pruebas del sistema o de integración entre subsistemas} . Este tipo de pruebas se realizan sobre sistemas que trabajan de forma conjunta e intercambian información entre sí, se realizaron pruebas para comprobar que esta interacción fuera correcta. Estas pruebas fueron realizadas principalmente a la finalización del desarrollo.
\end{itemize}


\section{Especificación del diseño de pruebas}

Hay dos fases temporales donde se han realizado las pruebas:

\begin{itemize}
\item {\bf Durante el desarrollo de la aplicación:} Esta etapa es la ideal para realizar las pruebas de clase individuales, ya que evitaremos propagar errores a fases posteriores. A medida que se iban desarrollando clases se iba comprobando su funcionalidad mediante la entrada de datos que cumplieran los requisitos de información especificados, así veríamos que salida daba la clase a esos datos para saber si el funcionamiento era el deseado o no, además de ver si los datos se estaban almacenando correctamente en la base de datos.

\item {\bf Una vez finalizada la aplicación:} Este es el momento de comprobar que la interacción entre los distintos subsistemas de la aplicación es correcta. Además habría que comprobar la seguridad del sistema, es decir, que un usuario con un nivel de privilegios insuficiente no pudiera acceder a un subsistema que tuviera un acceso restringido a su rol.\\

El proceso a seguir para realizar estas pruebas comenzó con la entrada con el usuario administrador a la aplicación, una vez logueado, procedimos a crear un usuario con rol planificador, para poder probar el grueso de subsistemas de la aplicación.\\

Una vez iniciada la sesión con un usuario de rol planificador, procedimos a probar los subsistemas de titulaciones, asignaturas y cursos, ya que era necesario tener registrados algunos elementos de estas clases para probar los demás subsistemas.\\

Después de dar de alta varias titulaciones y asignaturas, y un curso, se procedió a introducir planes docentes para esas asignaturas, de forma que tuvieramos horas asignadas para probar los horarios más adelante.\\

A continuación, se probó el subsistema de gestión de calendarios, que es independiente de los planes docentes pero no del curso, se probó a introducir varios eventos, y a eliminar algunos.\\

Se probó también la gestión de aulas, introduciendo algunas, eliminando y editando otras.\\

Hecho todo esto se podía pasar a la parte de gestión de horarios, comprobando la creación de grupos, edición de horarios, y chequeo de horas asignadas. Además de comprobar una vez rellenados algunos horarios, la gestión de informes de asignatura, comprobando que se generaran correctamente con los datos introducidos en los horarios.\\

Hecho todo esto también se hacía necesario probar la importación y exportación de los distintos elementos del sistema, como asignaturas, calendario o horarios.\\

Todas estas pruebas correspondían al perfil de planificador, pero también había que probar el perfil de alumno y el de profesor. Ambos son muy similares, ya que pueden ver casi lo mismo. Principalmente pueden configurar un horario, seleccionando una serie de asignaturas y grupos, aunque en el caso del alumno había que verificar que sólo pudiera ver las asignaturas de su titulación. Además el profesor tenía una funcionalidad extra que consistía en visualizar la planificación docente de una titulación.\\

Además también el rol administrador tiene su propia funcionalidad, que es la de crear usuarios, eliminarlos o editarlos, cosa que tuvo que ser probada también. 
\end{itemize}
\section{Especificación de los procedimientos de prueba}

Se realizaron pruebas sobre sistema operativo GNU/Linux en la distribución {\em Ubuntu 11.04} y en Microsoft Windows 7 sobre los siguientes navegadores:
\begin{itemize}
\item Mozilla Firefox 8
\item Internet Explorer 8
\item Opera 11.60
\item Google Chrome 15.0
\end{itemize}

Algunos problemas encontrados fueron con algunas propiedades de CSS y parte del código JavaScript, que principalmente en Internet Explorer no funcionaban correctamente, se pudieron solventar estos problemas con algunos parches encontrados.

\section{Documentación de la ejecución de las pruebas}
\begin{itemize}
\item {\bf Histórico de pruebas:} Muchos errores encontrados durante las pruebas fueron provocados por algún error simple en el código, que en ocasiones eran difíciles de encontrar por aparecer mensajes de error con poca información. Este tipo de casos se daba especialmente en JavaScript, ya que seguir la ejecución del código era complicado debido a que su consola de errores apenas daba información. También con la librería Doctrine hubo algunos problemas ya que a pesar de que si que mostraba errores, estos eran treméndamente crípticos, siendo en ocasiones provocados simplemente por la falta de declaración de un campo en un modelo.
\item {\bf Informe de incidentes ocurridos:} No hay incidencias que destacar. En cualquier caso, si se quisiera continuar el desarrollo, seria recomendable seguir usando el repositorio de Git, ya sea bien incorporándose al desarrollo, o bien haciendo un fork del repositorio, de forma que en caso de errores fuera posible volver a una versión anterior sin problema alguno.
\end{itemize}

\section{Herramientas utilizadas para las pruebas}

Al principio del desarrollo para trazar la funcionalidad de las distintas acciones, nos veíamos obligados a usar {\em echos} de las variables, mostrando su valor por el navegador, siendo este método un poco engorroso. Usando NetBeans se descubre que trae incorporada una herramienta llamada {\em XDebug}, que conjuntamente con una extensión del navegador Firefox llamada {\em EasyXDebug} facilitaba mucho el desarrollo y las pruebas, ya que nos permitía incorporar puntos de ruptura en el código, en los que usando NetBeans podíamos comprobar el estado de las variables en ese momento, sin necesidad de utilizar los ya mencionados {\em echo}.\\

Otra herramienta imprescindible para las pruebas en el desarrollo web es la extensión {\em FireBug}, de Mozilla Firefox que proporciona una consola de eventos de JavaScript, pudiendo además utilizar la funcionalidad de los puntos de ruptura en el código del cliente.


\chapter{Conclusiones}
En este capítulo se comentarán las conclusiones personales alcanzadas, así como las posibles ampliaciones futuras que se le podrían hacer a la aplicación.

\section{Opinión personal}

Con este proyecto se han abarcado aspectos tanto de software de gestión como de aplicación web, conocimientos que durante la carrera se dan de forma muy básica, por lo que proyectos como este pueden ayudarme a ampliar conocimientos para desarrollos y trabajos futuros.\\

Realmente es la primera vez que me enfrento a un desarrollo web relativamente grande, ya que mi conocimiento y experiencia no pasaba de la realización de algunas páginas estáticas mediante HTML. Este es, por tanto, el primer acercamiento real al lenguaje PHP, del que he descubierto su potencia en este proyecto. Es cierto que podría haber evitado usar algún framework y utilizar simplemente el lenguaje sin ninguna ayuda, pero creo que sólo hubiera hecho la labor más tediosa y aburrida, y considero que el uso de un framework debería ser considerado obligatorio por cualquier programador a la hora de realizar una aplicación web, sea en PHP o en cualquier otro lenguaje.\\

En cuanto a JavaScript, si bien es cierto que ya lo había usado alguna otra vez, realmente nunca había probado alguna librería como  {\em jQuery}, y me ha sorprendido gratamente su potencia y facilidad de uso, y lo que se puede hacer mejorando notablemente la experiencia del usuario.\\

En cuanto a los sistemas de control de versiones, ya había tenido experiencia con {\em Subversion}, pero Git sorprende por su versatilidad, mostrándose mucho más fuerte a la hora de realizar desarrollos colaborativos. Se utilizó {\em GitHub} como servidor {\em Git}, ya que disponía de una amplia comunidad de usuarios y parecía ser de los más recomendados en la web, además de que disponía de un sistema de seguimiento de tareas muy bueno y fácil de usar.\\

En cuanto a \LaTeX no es la primera vez que lo he usado, pero esta memoria me ha servido para ampliar un poco más mi conocimiento de este lenguaje.

\section{Ampliaciones futuras}

Este proyecto está sujeto a cambios, ya que es posible que cambie la forma en la que se plantea la planificación docente de una asignatura, y aunque la aplicación se ha intentado parametrizar lo máximo posible para poder personalizar muchos aspectos, es posible que sea necesario realizar cambios.\\

Entre las posibles ampliaciones que se pueden hacer está el que un alumno pueda loguearse con su usuario habitual del campus virtual, integrándolo con el LDAP de la Universidad, no se ha hecho debido a que el objetivo principal de la aplicación es ser usada por un usuario planificador para la configuración, teniendo el alumno una funcionalidad mínima.\\

Otra posible mejora sería una cierta automatización en la creación de los horarios, por ejemplo proponiendo un profesor su horario preferente y generándose el mejor horario posible, pudiendo ser modificado luego. En este aspecto también estaria bien la posibilidad de que un alumno propusiera una configuración mejor para un horario.




%\backmatter % Apéndices, bibliografia ...
\appendix
\chapter{Herramientas utilizadas}
\section{Lenguaje de programación}

Una de las decisiones principales que hay que tomar a la hora de realizar un proyecto es escoger el lenguaje de programación. Al tratarse de un proyecto web, se decide utilizar PHP, al ser uno de los lenguajes más extendidos en este tipo de desarrollo y ser uno de los que más se asemeja a la sintaxis de C, que es el lenguaje del que tenemos mayor base gracias a la carrera.\\

Además otra de las razones es su extensa documentación y su gran cantidad de librerías disponibles para extender el lenguaje.\\

También es importante destacar que ya que queríamos utilizar un framework que usara la arquitectura MVC, este lenguaje era el más indicado, ya que es el que dispone de más frameworks de este tipo, entre ellos {\em Codeigniter}, {\em CakePHP}, {\em Symfony} o {\em Zend Framework}.\\

De entre ellos se decide usar Codeigniter, al ser probablemente el más ligero de todos, lo que hace que sea el más rápido comparándolo a otros frameworks, siendo la velocidad un punto crítico en esta clase de librerías. 

\section{Entorno de desarrollo}

Otra elección importante es el entorno de desarrollo de código, ya que según la elección que hagamos, puede favorecer nuestra productividad o hacernos más lentos en nuestro trabajo. Aquí existe la posibilidad de decantarse por un simple editor de código o bien utilizar un entorno de desarrollo integrado (IDE), con múltiples herramientas que nos ayudan en nuestro trabajo. Nosotros nos decantamos por la segunda opción.
\\
En un principio se comenzó usando {\em Aptana Studio}, basado en {\em Eclipse}, pero poco después descubrimos {\em NetBeans}, un entorno pensado inicialmente para el desarrollo Java, pero adaptado a otros lenguajes. NetBeans proporciona entre otras cosas autocompletado de código y un debugger muy completo que nos ayuda a encontrar errores en el código.

\section{Herramienta UML}

Para la creación de los diagramas UML se utilizó {\em BoUML}, que permite realizar todo tipo de diagrama dentro del estándar UML 2. Incluso provee una herramienta de generación de código para múltiples lenguajes.
\\
Es una herramienta multiplataforma y gratuita.\\

Además para la creación de los diagramas entidad-relación se ha utilizado la herramienta {\em DIA}, también multiplataforma y gratuita, y encuadrada en el proyecto GNOME.

\section{Redacción de la memoria y resumen}

Para la realización de la memoria se ha utilizado \LaTeX, que es un lenguaje de marcado para la composición de textos científicos. Es una herramienta realmente fácil de usar y que da como resultado documentos de una gran calidad tipográfica, bastante más difícil de obtener con un procesador de textos normal.\\

\LaTeX es libre y multiplataforma y está basado en \TeX.

\section{Ediciones rápidas de código}

A veces es necesario hacer ediciones rápidas de código para lo que no se ve necesario y productivo abrir el IDE, para ello se ha utilizado {\em Emacs}, un completo editor multiplataforma que se encuentra en el proyecto GNU y que dándole un uso adecuado puede ser tan potente como un IDE

\section{Planificación del proyecto}

Para la gestión de la planificación de recursos se ha utilizado el software gratuito {\em Planner}, la elección se realizó ya que se había utilizado en otros proyectos y la experiencia fue satisfactoria.


\chapter{Manual de instalación}
\section{Prerrequisitos}

Para poder instalar la aplicación en un servidor debemos tener previamente instalados una serie de programas, disponibles tanto en Linux como en Windows. A continuación se enumeran estos paquetes necesarios para el funcionamiento:

\begin{itemize}

\item {\bf MySQL Server}: Es el sistema gestor de base de datos de la aplicación. En Linux se puede obtener de los repositorios, también se puede descargar de la página oficial:\\
\href{http://dev.mysql.com/downloads/mysql/}{http://dev.mysql.com/downloads/mysql/}\\
Es importante conocer la contraseña de root ya que será necesaria para crear la base de datos y el usuario al que será asociada la aplicación.
\item {\bf PHP}: Debemos tener instalada una versión de PHP igual o superior a la 5.3, se puede descargar sin problemas de los repositorios, o bien de la página oficial.
\item {\bf Apache httpd server}: También disponible tanto en la página oficial como en los repositorios.
\end{itemize}

\section{Instalación de la aplicación}
La aplicación se proporciona en un fichero .zip, así que solo habrá que descomprimirlo en una carpeta del servidor web, es importante saber la ruta desde la que se accede en el servidor, ya que habrá que configurar la aplicación apropiadamente más adelante.
\paragraph{}
Es importante no cambiar ningún fichero ni carpeta en la jerarquía de directorios de la aplicación, sino el funcionamiento podría alterarse. También es importante no borrar el fichero .htaccess disponible en la raíz de la aplicación.

\section{Puesta en funcionamiento}
Para que la aplicación funcione, necesita tener una base de datos creada, además del usuario con el que se conectará desde la aplicación. Para ello se deben seguir los pasos descritos a continuación:
\begin{itemize}
\item Lo primero que hay que hacer es acceder a {\em MySQL} con el usuario root, escribiendo desde el terminal lo siguiente:
\begin{lstlisting}[style=consola]
	mysql -u root -p
\end{lstlisting}
Y a continuación se nos pedirá la contraseña.
\item El siguiente paso es crear la base de datos con el nombre ''gestiongrados'', para ello escribimos:
\begin{lstlisting}[style=consola]
	mysql > CREATE DATABASE gestiongrados;
\end{lstlisting}
\item Una vez creada la base de datos, hay que crear el usuario ''gestiongrados'', escribimos lo siguiente en la consola de {\em MySQL}:
\begin{lstlisting}[style=consola]
	mysql > GRANT CREATE, SELECT, INSERT, DELETE, UPDATE 
	ON gestiongrados.* to 'gestiongrados'@'localhost' 
	IDENTIFIED BY 'ges1234';
\end{lstlisting}
Nótese que se ha asignado el password ges1234, puede ser cambiado, pero deberá ser configurado en la aplicación más adelante.
\item Aplicamos los cambios en la base de datos:
\begin{lstlisting}[style=consola]
	mysql > FLUSH PRIVILEGES
\end{lstlisting}
\end{itemize}

Ya tenemos la base de datos creada, pero ahora hay que modificar algunos parámetros de configuración en el fichero de configuración de la aplicación. Para ello abrimos el fichero './application/config/config.php'.\\
Este fichero únicamente hace asignaciones en un array asociativo \$config, donde cada clave es un parámetro de configuración. Debemos hacer la siguiente modificación:

\begin{itemize}
\item En primer lugar debemos modificar el valor de la clave ''base\_url'', que es el que contiene la url y ruta de la carpeta del servidor donde se ubica la aplicación, por ejemplo si el servidor es gestion.uca.es, y la ruta es /gestiongrados, el valor de la clave deberá ser:
\begin{lstlisting}[style=PHP]
$config['base_url'] = 'http://gestion.uca.es/gestiongrados';
\end{lstlisting}
\item No modificar ningún otro parámetro, ya que esto podría provocar un mal funcionamiento de la aplicación.
\end{itemize}

A continuación debemos ir al fichero './application/config/database.php' y hacer las siguientes modificaciones:
\begin{itemize}
\item Modificar el valor del hostname, que corresponderá al servidor donde estará ubicada la base de datos.
\begin{lstlisting}[style=PHP]
	$db['default']['hostname'] = 'localhost';
\end{lstlisting}
\item Modificar el nombre de usuario si se ha cambiado al crear la base de datos, sino dejar el que está ('gestiongrados'):
\begin{lstlisting}[style=PHP]
	$db['default']['username'] = 'gestiongrados';
\end{lstlisting}
\item Modificar el password si se ha modificado al crear el usuario en la base de datos, sino dejar el que está ('ges1234'):
\begin{lstlisting}[style=PHP]
	$db['default']['password'] = 'ges1234';
\end{lstlisting}
\item Modificar el nombre de la base de datos si se ha modificado al crearla, sino dejar el que está ('gestiongrados'):
\begin{lstlisting}[style=PHP]
	$db['default']['database'] = 'gestiongrados';
\end{lstlisting}
\item Dejar todos los demás parámetros tal cual están, sino se podría obtener un mal funcionamiento.
\end{itemize}

Una vez hecho esto, podremos finalizar la aplicación entrando en la ruta de instalación de la aplicación, que se encargará de crear la estructura de la base de datos además de un usuario administrador, al que podremos asignar una contraseña, para ello debemos escribir en el navegador la ruta base de la aplicación, seguido de "/install", pantalla en la que se nos pedirá una contraseña para finalizar la instalación de la aplicación, además de una dirección de correo electrónico que será la que se use para entrar en la aplicación. Una vez terminada la instalación, se nos indicará con un mensaje, y podremos empezar a trabajar con ella.


\chapter{Manual de usuario}
\section{Bienvenida}

Cuando accedemos a la aplicación por la ruta principal, veremos la pantalla de bienvenida, a la izquierda tendremos un menú donde podremos registrar un nuevo usuario o bien loguearnos con un usuario ya creado. La aplicación dispone de cuatro perfiles distintos de usuario, cada uno con sus funciones asociadas, estos serían: administrador, planificador, profesor y alumno. Desde esta pantalla solo podremos crear usuarios del perfil alumno y con un e-mail de la UCA válido.\\

\begin{figure}[H] 
  \label{manual-bienvenida} 
	\begin{center}
    \includegraphics[scale=0.4]{./manual-bienvenida.png}
  \end{center}
\caption{Pantalla de bienvenida de la aplicación}
\end{figure}

El manual está dividido en secciones que detallan las funcionalidades de cada perfil, en primer lugar se explicarán las funciones del perfil administrador, seguido de las de planificador, profesor y alumno.

\section{Perfil administrador}

En primer lugar debemos disponer de un usuario administrador para poder acceder a las funcionalidades de este perfil, para ello, pulsamos sobre el enlace 'login' a la izquierda, y veremos una ventana flotante donde podremos introducir el usuario y contraseña para acceder.\\

\begin{figure}[H] 
  \label{manual-login} 
	\begin{center}
    \includegraphics[scale=0.75]{./manual-ventana-login.png}
  \end{center}
\caption{Ventana de login de usuario}
\end{figure}

Una vez dentro, veremos de nuevo la pantalla de bienvenida, pero veremos que en el menú de la izquierda han aparecido nuevas funciones, concretamente los apartados 'Usuario' y 'Configuración'.

\subsection{Administración de usuarios}
\label{manual-administracion-usuarios}

Dentro del apartado usuario podemos hacer diferentes cosas, añadir usuarios, ver los usuarios disponibles en el sistema, editarlos o eliminarlos, incluyendo los datos del propio usuario administrador.

\begin{figure}[H] 
  \label{manual-menu-admin} 
	\begin{center}
    \includegraphics[scale=0.9]{./menu-user-admin.png}
  \end{center}
\caption{Menú lateral con usuario administrador en el sistema}
\end{figure}

\subsubsection{Añadir usuario}
\label{manual-anadir-usuario}
Esta función es exclusiva del perfil administrador y del perfil planificador.\\

Para entrar hacemos click en la opción 'Añadir usuario' dentro del apartado 'Usuario'.\\

A continuación veremos un formulario en el que podremos introducir los datos del usuario a crear asi como el perfil correspondiente a ese usuario, debemos introducir nombre, apellidos y e-mail (es importante que sea un e-mail válido de la UCA, ya que se enviará la contraseña a esa dirección). Una vez pulsemos en enviar, si todos los datos son correctos, se creará el usuario, enviando su contraseña al e-mail dado.

\begin{figure}[H] 
  \label{manual-add-user} 
	\begin{center}
    \includegraphics[scale=0.8]{./manual-form-add-user.png}
  \end{center}
\caption{Formulario para añadir un usuario desde el perfil administrador}
\end{figure}

\subsubsection{Ver usuarios}
Esta función es exclusiva del perfil administrador y del perfil planificador.\\

Para entrar hacemos click en la opción 'Ver usuarios' dentro de 'Usuario'.\\

Veremos una lista de todos los usuarios registrados en el sistema, junto a cada uno de ellos tenemos disponible un botón para editar y otro para eliminar el usuario. Si pulsamos sobre editar veremos un formulario similar al del apartado \hyperref[manual-anadir-usuario]{Añadir usuario} pero con los datos del usuario sobreimpresos, podemos modificar los datos que queramos y guardar el usuario, incluído la contraseña.\\

Si pulsamos sobre eliminar, borraremos el usuario del sistema.

\begin{figure}[H] 
  \label{manual-lista-users} 
	\begin{center}
    \includegraphics[scale=0.8]{./manual-lista-users.png}
  \end{center}
\caption{Lista de usuarios en 'Ver usuarios'}
\end{figure}

\subsection{Apartado Configuración}

Aquí podremos realizar backups o copias de seguridad de la base de datos de la aplicación, el sistema nos permitirá guardar el archivo .sql resultante. Además podemos restaurar una copia realizada anteriormente, teniendo en cuenta que todos los datos anteriores serán borrados del sistema, por tanto es una función que hay que utilizar con cuidado.

\section{Perfil planificador}
Este es el perfil con más funcionalidad de la aplicación, es donde se controla la funcionalidad principal de la aplicación, es decir la administración y planificación de la carga horaria y docente de las titulaciones de grado. Las funcionalidades que maneja son la creación y edición de titulaciones, asignaturas, cursos, calendario, horarios y planificación docente. Además, al igual que el administrador, también puede crear y editar usuarios.

\subsection{Administración de usuarios}
Esta sección es exactamente igual a la del perfil administrador, para más información consultar \hyperref[manual-administracion-usuarios]{Administración de usuarios}

\subsection{Gestión de titulaciones}
Esta sección es exclusiva del perfil planificador.\\

Desde aquí podemos crear nuevas titulaciones o ver las ya creadas y eliminarlas o editarlas.


\begin{figure}[H] 
  \label{manual-form-add-titu} 
	\begin{center}
    \includegraphics[scale=0.8]{./manual-form-add-titu.png}
  \end{center}
\caption{Formulario para añadir titulaciones al sistema}
\end{figure}

\subsubsection{Añadir titulaciones}
\label{manual-anadir-titulaciones}
Si hacemos click en 'Titulaciones', y a continuación en añadir titulaciones, se nos mostrará un formulario para introducir los datos de una titulación, nombre créditos, y número de cursos que tendrá.

\subsubsection{Editar titulaciones}
Para editar una titulación debemos ir a 'Titulaciones' y a continuación hacer click en 'Ver titulaciones'. Se nos mostrará un listado de las titulaciones disponibles, y tendremos un botón editar a disposición. Si hacemos click en él, veremos un formulario similar al de \hyperref[manual-anadir-titulaciones]{Añadir titulaciones} con los datos de la titulación sobreimpresos pudiendo modificarlos.

\subsubsection{Eliminar titulaciones}
Para eliminar una titulación debemos ir de nuevo al apartado 'Ver titulaciones' y ahí hacer click en el botón de eliminar. Una vez eliminada no podrá recuperarse.

\subsubsection{Ver asignaturas de una titulación}
\label{manual-ver-asignaturas}
Desde el menú ver titulaciones podemos ver las asignaturas de una titulación haciendo click sobre el botón correspondiente.

\subsection{Gestión de asignaturas}
Esta sección es exclusiva del perfil planificador.\\

Desde aquí podremos gestionar todo lo referente a las asignaturas, es decir, listarlas, eliminarlas, editarlas o añadir nuevas. También podemos importar un fichero YML para cargar asignaturas masivamente, cuyo formato se describirá en un anexo aparte.\\

\subsubsection{Añadir asignatura}
Para poder añadir asignaturas debemos listar primero las titulaciones, para ello tenemos un acceso directo en el menú 'Asignaturas'->'Asignatura'. Ahí dispondremos de un botón por cada titulación para poder añadir asignaturas, se nos mostrará un formulario para introducir los datos.

\subsubsection{Resto de operaciones}
Para acceder a ellas debemos ir de nuevo al menú 'Asignaturas'->'Asignatura', y hacer click sobre el enlace 'Ver asignaturas' de una titulación. Esto nos mostrará una lista de las asignaturas con todas las operaciones posibles.

\subsection{Gestión de cursos}
Esta sección es exclusiva del perfil planificador.\\

En el menú cursos tenemos dos accesos, 'Añadir curso', que nos mostrará un formulario para crear un nuevo curso, y 'Ver cursos' desde donde podremos eliminarlos y editarlos.

\subsection{Gestión de aulas}
Esta sección es exclusiva del perfil planificador.\\

Aquí en esta sección, tenemos dos opciones, 'Crear aula' y 'Ver aulas'.

\subsubsection{Crear aula}
\label{manual_crear_aula}
Para crear un aula, hacemos click en la opción 'Crear aula' del menú 'Aulas', esto hará que aparezca un formulario.\\

El formulario está compuesto de dos partes, un campo nombre, para introducir el nombre del aula, y un campo tipos, con varios checkboxes, que podemos marcar. Esto hará que le asignemos actividades posibles que se pueden impartir en ese aula, de forma que cuando estemos creando los horarios, solo podamos asignar un slot de una actividad concreta a un aula que tenga esa actividad asignada. Una vez rellenado el formulario, pulsando en 'Enviar' crearemos el aula.

\subsubsection{Ver aulas}

En esta sección veremos una lista de todas las aulas creadas hasta ahora, pudiendo editar y eliminar cada una. 'Editar' nos mostrará un formulario similar al de \hyperref[manual_crear_aula]{Crear aula}, con los datos asignados sobreimpresos.


\subsection{Gestión de Planificación Docente}

Desde aquí podremos visualizar la planificación docente, crear nuevos planes docentes, o bien importar un csv para hacer cargas masivas de planes docentes de diferentes asignaturas.

\subsubsection{Añadir plan docente}
Al hacer click aquí, se nos mostrará un listado con los cursos disponibles, una vez seleccionemos uno, se nos llevará a otro listado con las titulaciones, y seleccionando una, se nos mostrarán sus asignaturas, a las cuales podremos añadir un plan docente pulsando en el correspondiente enlace. Una vez ahí, se nos mostrará un formulario con una línea por cada actividad, en cada línea una caja para introducir el número de horas totales, otra para las horas semanales y otra para el número de grupos. Además si la actividad no es teoría, se nos mostrará también un checkbox para activar las semanas alternas en esa actividad.
Además tendremos una línea más para añadir los cursos con los que se comparte esa asignatura, en caso de que sea una asignatura común a varias especialidades.\\

Si dejamos alguna caja en blanco, se interpretará como un 0, es decir, si dejamos en blanco la casilla de número de grupos de problemas, significará que habrá 0 grupos de problemas.\\

Si marcamos una casilla de semanas alternas, significará que en la etapa de creación de horarios se asignarán los grupos de dos en dos, es decir, solo veremos la mitad de los grupos que hayamos asignado aquí.

\begin{figure}[H] 
  \label{manual-add-plan} 
	\begin{center}
    \includegraphics[scale=0.7]{./manual-add-plan.png}
  \end{center}
\caption{Formulario para añadir un plan docente}
\end{figure}

\subsubsection{Importar plan docente}

Desde la opción 'Importar CSV' del menú 'Plan Docente', podemos importar masivamente planes docentes de diferentes asignaturas y titulaciones, tan solo subiendo un archivo con el formato correcto.
El formato necesario se indica en un anexo posterior.

\subsubsection{Ver planificación}

Esta sección funciona de forma similar a la del \hyperref[manual_perfil_profesor]{perfil profesor}, solo que aquí podemos exportar la planificación a un archivo csv, mediante el botón situado en la parte superior izquierda de la pantalla. El formato al que se exporta se expondrá en un anexo posterior.

\begin{figure}[H] 
  \label{manual-visualizacion-plan} 
	\begin{center}
    \includegraphics[scale=0.65]{./manual-visualizacion-plan.png}
  \end{center}
\caption{Visualización global de la planificación docente}
\end{figure}

\subsection{Gestión del calendario}

En la opción 'Calendario' en el menú, veremos un calendario en el que podremos añadir eventos al curso actual.\\

La forma de añadir eventos es sencilla, tan solo hay que hacer click sobre una fecha del calendario, se nos mostrará una ventana flotante en la que únicamente tendremos que escribir el nombre del evento que se va a crear.\\

Además también se pueden añadir eventos de más de una fecha, para hacer esto, simplemente tenemos que hacer click sobre una fecha y mantener pulsado el ratón, arrastrando el puntero hasta la fecha de finalización, una vez ahí, se suelta el botón del ratón y se mostrará la ventana para introducir el nombre, se podrá observar que las fechas indicadas son las señaladas.\\

Estas fechas del calendario sirven para hacer el recuento de horas reales impartidas en un curso, esto se explicará en la sección \hyperref[manual_calculo_horas]{cálculo de horas impartidas}.
\\
Además, si queremos borrar un evento ya asignado, únicamente tendremos que hacer click sobre él y nos aparecerá una ventana con dos opciones, 'Borrar' o 'Cancelar', si clickamos en 'Borrar' el evento desaparecerá.

\begin{figure}[H] 
  \label{manual-calendario} 
	\begin{center}
    \includegraphics[scale=0.5]{./manual-calendario.png}
  \end{center}
\caption{Visualización del calendario con dos eventos}
\end{figure}


\subsection{Gestión de horarios}
Está sección es también exclusiva de este perfil.
\\
En el menú 'Horarios' solo hay dos opciones, 'Grupos y horarios' e 'Informes de asignatura'.

\subsubsection{Configuración de horarios}

Para configurar los horarios de una titulación, entramos en 'Grupos y horarios', después de seleccionar titulación y curso, veremos una tabla con todos los cursos dentro de esa titulación, a los que podremos añadir grupos o eliminarlos.\\


\begin{figure}[H] 
  \label{manual-seleccion-grupos} 
	\begin{center}
    \includegraphics[scale=0.5]{./manual-seleccion-grupos.png}
  \end{center}
\caption{Pantalla de selección de grupos}
\end{figure}

Hay que tener en cuenta que una vez que añadamos un grupo no podremos configurar el horario de asignaturas a las que añadamos planes docentes posteriormente, para ello habría que eliminar el grupo y crearlo de nuevo, por tanto hay que estar seguro de que todas las asignaturas tienen asignada su planifición docente antes de añadir un grupo. Para añadir un grupo pulsamos en el botón '+', esto hará que se creen los horarios para ese curso, tanto del primer como del segundo semestre.\\

El paso inicial es el de configurar el horario tipo, ya que este se copiará a los de las semanas iniciales cuando los editemos, para ello pulsamos sobre el enlace 'Horario tipo' del curso y semestre deseado.
Veremos un cuadro en la parte superior clasificado en pestañas por asignaturas, en cada una de ellas veremos los slots disponibles de cada actividad, con un combo con las aulas disponibles a la que le asignaremos la asignatura, y un color que es con el que aparecerá el slot en el horario. Para asignarlo al horario simplemente tenemos que hacer click sobre el slot, manteniendo pulsado y arrastrarlo al horario en la parte inferior, colocándolo en el lugar deseado. Si hay algún problema, por ejemplo que el aula esté ocupada, o que el grupo se pise con otro y no esté permitido, el slot volverá a su lugar en la parte superior y no se asignará al horario.\\

\begin{figure}[H] 
  \label{manual-configuracion-horario} 
	\begin{center}
    \includegraphics[scale=0.5]{./manual-configuracion-horario.png}
  \end{center}
\caption{Pantalla de configuración de horarios}
\end{figure}

Una vez asignado un slot, podemos moverlo sin problemas a otro lugar en el horario, o incluso borrarlo, para ello, simplemente hacemos click sobre el slot, y a continuación hacemos otro click en el botón borrar, esto hará que el slot vuelva al cuadro superior, pudiéndolo asignar de nuevo si lo deseamos.\\

Otra posibilidad es visualizar la ocupación de un aula concreta, para ello la seleccionamos en el combo correspondiente y hacemos click en 'Ver ocupación de aula' en la parte inferior. Esta ocupación la podemos exportar haciendo click en el botón con el icono de CSV junto al de 'Ver ocupación de aula'.
\\
Podemos ir directamente a un horario de las semanas iniciales con los botones de la parte superior, el funcionamiento para editar estos horarios es exactamente el mismo, solo que no veremos los grupos que no sean de teoría.

\subsubsection{Exportar horarios}

Para exportar un horario a CSV debemos ir al menú 'Grupos y horarios', y en la columna exportar pulsar sobre el botón correspondiente, esto exportará todos los horarios de ese semestre para esa titulación y curso, es decir, tanto semanas iniciales como horario tipo.

\subsubsection{Comprobación de horas impartidas}

Una vez configurados los horarios, en el menú 'Grupos y horarios' tenemos una columna con el botón 'Comprobación', que nos mostrará una tabla con las horas planificadas y las que se impartirán realmente según la configuración de los horarios, de forma que si faltan horas cambiemos lo que sea necesario.  

\subsubsection{Informes de asignatura}
En el menú 'Informes de asignatura' podremos realizar informes detallados de las horas de las asignaturas que deseemos, estos informes se exportarán a PDF, e incluirán las asignaturas que seleccionemos en la pantalla correspondiente. En el informe se verán las horas impartidas por cada semana, actividad y grupo.

\begin{figure}[H] 
  \label{manual-informe} 
	\begin{center}
    \includegraphics[scale=0.3]{./manual-informe.png}
  \end{center}
\caption{Parte de un informe de una asignatura}
\end{figure}

\section{Perfil profesor}
\label{manual_perfil_profesor}
El perfil profesor solo tiene una finalidad, que es el de la visualización de la planificación global para un curso de las titulaciones, la visualización se hace a través de las titulaciones, y no desde una asignatura como si puede hacer el planificador.\\

Para ello, el profesor selecciona 'Planificación Docente' en el menú de la izquierda y a continuación la opción 'Ver planificación', que es la única disponible. Al usuario se le pedirá seleccionar un curso, y a continuación, una titulación. Se visualizará entonces una tabla con la planificación docente de todas las asignaturas de esa titulación, clasificado por actividad.\\

Además de esto, el profesor, al igual que todos los demás usuarios, puede cambiar su contraseña.
\\
Ver figura~\ref{manual-visualizacion-plan} en la página~\pageref{manual-visualizacion-plan}.

\section{Perfil alumno}

El perfil alumno solo tiene una opción, que es la de la visualización de horarios.\\

El alumno es un tipo de usuario especial, tiene una titulación asignada, por tanto no deberá seleccionar la titulación de la que quiere ver el horario, sino que automáticamente serán visibles solo las asignaturas de su grado.\\


\begin{figure}[H] 
  \label{manual-alumno-seleccion} 
	\begin{center}
    \includegraphics[scale=0.7]{./alumno-seleccion-grupos.png}
  \end{center}
\caption{Selección de un alumno de los grupos para visualizar un horario}
\end{figure}

Para proceder a la personalización del horario, el alumno debe seleccionar 'Horarios' en el menú de la izquierda, y a continuación 'Ver horario'. Después de seleccionar el curso de la lista ofrecida, aparecerá un listado de las asignaturas disponibles, se podrán seleccionar las deseadas y el grupo de teoría que se quiere visualizar. Una vez seleccionado, en el paso siguiente se ofrecerá un listado de las asignaturas anteriormente seleccionadas junto con sus actividades y grupos, pudiendo seleccionar los que se quieran para configurar un horario a medida.\\

Una vez configurado veremos los horarios de las primeras semanas y el horario tipo de las asignaturas seleccionadas.

\chapter{Importación de asignaturas}
En esta sección se explicará el formato que deben seguir los archivos de importación de asignaturas. Existen solo un formato posible para hacer la importación de las asignaturas, éste es YAML (YAML Ain't Markup Language). Es un lenguaje creado para escribir objetos de un lenguaje orientado a objetos en un archivo de texto, por ello parece el más adecuado para hacer la importación de las asignaturas.

\section{YAML}

Es un formato que destaca por su sencillez y claridad a la hora de construir el archivo, además de ser un formato recomendado para plasmar objetos de un lenguaje de programación en un archivo de texto.
\\
La construcción del archivo se hace de la siguiente forma, se comienza con una línea con el nombre de la clase que se va a importar, en este caso, Asignatura. Es importante que la primera letra esté en mayúsculas y la palabra en singular. Se sigue la palabra del caracter ':'. A continuación después de un salto de línea el siguiente nivel debe estar indentado, el siguiente paso es especificar un objeto concreto, se escribe una palabra como identificador, seguido de ':', y a continuación el siguiente nivel, que serían los atributos de la asignatura. Estos atributos deben estar indentados y se escribirá el nombre del atributo, seguido de ':', y del valor del atributo. Es importante conocer el valor del identificador de la titulación a la que se va a asociar, éste se puede consultar en el listado de titulaciones de la aplicación.
\\
Un ejemplo de un archivo seria el siguiente:
\begin{verbatim}
Asignatura:
  Asignatura_7:
    codigo: '123'
    nombre: 'Análisis y diseño de algoritmos I'
    abreviatura: ADAI
    creditos: '6'
    materia: Algoritmia
    departamento: 'Lenguajes y sistemas'
    curso: '2'
    semestre: primero
    titulacion_id: 5
  Asignatura_8:
    codigo: '124'
    nombre: 'Estructura de Datos I'
    abreviatura: EDI
    creditos: '6'
    materia: Programación
    departamento: 'Lenguajes y sistemas'
    curso: '1'
    semestre: segundo
    titulacion_id: 5
  Asignatura_9:
    codigo: '122'
    nombre: 'Fundamentos en Informática'
    abreviatura: FI
    creditos: '6'
    materia: Porgramación
    departamento: 'Lenguajes y sistemas'
    curso: '1'
    semestre: primero
    titulacion_id: 6
\end{verbatim}

\chapter{Importación de planes docentes}
En esta sección se explicará la forma de importar masivamente planes docentes desde un archivo CSV. Para ello a continuación se explicará el formato que deben seguir estos archivos.
\\
En primer lugar hay que escribir una cabecera con los atributos necesarios para construir un plan docente. Esta cabecera seria la siguiente:

\begin{verbatim}
id_asignatura, id_actividad, horas, horas_semanales, grupos, alternas, id_curso
\end{verbatim}

Esta sería la cabecera que debe estar en la primera línea del archivo, id\_asignatura corresponde al identificador en la base de datos de la asignatura a la que corresponde este plan docente, id\_actividad es el identificador de la actividad a la que pertenece esa línea del plan docente, este es un número que corresponde al orden en el que aparece la actividad en el formulario de creación de plan docente, es decir, 1 para teoría, 2 para problemas, 3 para laboratorio, 4 para informática y 5 para prácticas de campo. El atributo horas corresponde al número de horas totales que tendrá esa asignatura en esa actividad, horas\_semanales es el número de horas semanales de esa actividad, grupos el número total de grupos que tendrá la actividad, alternas será un valor binario, que indicará si la actividad se imparte en semanas alternas o no, es decir, 1 para sí, 0 para no. Finalmente id\_curso, corresponde al identificador de la base de datos del curso al que pertenece el plan.
\\
Un ejemplo del archivo seria el siguiente:

\begin{verbatim}
id_asignatura, id_actividad, horas, horas_semanales, grupos, alternas, id_curso
2,1,40,3,3,0,1
2,2,30,2,9,0,1
3,2,30,2,9,0,1
3,1,40,3,3,0,1
\end{verbatim}
\chapter{Informes}
En este capítulo se mostrarán algunos de los informes que genera la aplicación, tanto en pdf como csv.

\section{Horarios}

Se puede exportar un horario completo a CSV, se muestran en el mismo archivo separados por saltos de línea los horarios pertenecientes a un grupo. Un ejemplo sería el siguiente:

\begin{verbatim}
Horario tipo 
,L,M,X,J,V
09:00,"ADAI C3",,,"pI A2","pI A2"
09:30,"ADAI C3",,"ADAI C4","pI A2","pI A2"
10:00,"ADAI C3",,"ADAI C4",,
10:30,"ADAI C3",,"ADAI C4","pI A2",
11:00,,,"ADAI C4",,"ADAI B4"
11:30,"pI B2","pI A2",,,"ADAI B4"
12:00,"pI B2",,,,"ADAI B4"
12:30,"pI B2",,,,"ADAI B4"
13:00,"pI B2",,,"ADAI A2","ADAI B4"
13:30,,,,,"ADAI B4"
14:00,,"ADAI A2",,"ADAI B6",
14:30,"ADAI B5",,,"ADAI B6","ADAI A2"
15:00,"ADAI B5",,,"ADAI B6","ADAI A2"
15:30,"ADAI B5",,,"ADAI B6","ADAI A2"
16:00,"ADAI B5",,,"ADAI B6",
16:30,"ADAI B5",,,"ADAI B6","ADAI A2"
17:00,"ADAI B5",,,,
17:30,,,,,
18:00,,,,,
18:30,,,,,
19:00,,,,,
19:30,,,,,
20:00,,,,,
20:30,,,,,
21:00,,,,,
21:30,,,,,
22:00,,,,,

Semana 1
,L,M,X,J,V
09:00,,,,"pI A2","pI A2"
09:30,,,,"pI A2","pI A2"
10:00,,,,,
10:30,,,,"pI A2",
11:00,,,,,
11:30,,,,,
12:00,,,,,
12:30,,,,,
13:00,,,,"ADAI A2",
13:30,,,,,
14:00,,,,"ADAI A2",
14:30,,,,,"ADAI A2"
15:00,,,,,"ADAI A2"
15:30,,,,"pI A2","ADAI A2"
16:00,,,,,
16:30,,,,,"ADAI A2"
17:00,,,,,
17:30,,,,,
18:00,,,,,
18:30,,,,,
19:00,,,,,
19:30,,,,,
20:00,,,,,
20:30,,,,,
21:00,,,,,
21:30,,,,,
22:00,,,,,

Semana 3
,L,M,X,J,V
09:00,,,,"pI A2","pI A2"
09:30,,,,"pI A2","pI A2"
10:00,,,,,
10:30,,,,"pI A2",
11:00,,,,,
11:30,,"pI A2",,,
12:00,,,,,
12:30,,,,,
13:00,,,,"ADAI A2",
13:30,,,,,
14:00,,"ADAI A2",,,
14:30,,,,,"ADAI A2"
15:00,,,,,"ADAI A2"
15:30,,,,,"ADAI A2"
16:00,,,,,
16:30,,,,,"ADAI A2"
17:00,,,,,
17:30,,,,,
18:00,,,,,
18:30,,,,,
19:00,,,,,
19:30,,,,,
20:00,,,,,
20:30,,,,,
21:00,,,,,
21:30,,,,,
22:00,,,,,

Semana 2
,L,M,X,J,V
09:00,,,,"pI A2","pI A2"
09:30,,,,"pI A2","pI A2"
10:00,,,,,
10:30,,,,"pI A2",
11:00,,,,,
11:30,,"pI A2",,,
12:00,,,,,
12:30,,,,,
13:00,,,,"ADAI A2",
13:30,,,,,
14:00,,"ADAI A2",,,
14:30,,,,,"ADAI A2"
15:00,,,,,"ADAI A2"
15:30,,,,,"ADAI A2"
16:00,,,,,
16:30,,,,,"ADAI A2"
17:00,,,,,
17:30,,,,,
18:00,,,,,
18:30,,,,,
19:00,,,,,
19:30,,,,,
20:00,,,,,
20:30,,,,,
21:00,,,,,
21:30,,,,,
22:00,,,,,
\end{verbatim}

Esto puede parecer desordenado, pero si se abre con un programa de edición de hojas de cálculo, como por ejemplo {\em LibreOffice}, lo veremos de la siguiente manera:


\begin{figure}[H] 
  \label{informes-horario} 
	\begin{center}
    \includegraphics[scale=1]{./horario-export.png}
  \end{center}
\caption{Fragmento de un horario exportado abierto con LibreOffice Calc}
\end{figure}

Se puede observar el horario de forma bastante clara, y desde aquí podremos hacer las ediciones que veamos necesarias.

\section{Ocupación de aulas}

La ocupación de aulas se visualiza en un formato muy similar a la del horario ya que en realidad es un horario en sí mismo, pero filtrado por aula en lugar de un grupo de teoría de una titulación. La única diferencia es que solo veremos la ocupación de la semana que se haya exportado y no las tres en el mismo archivo como pasa con los horarios.
\\
Un ejemplo es el siguiente:
\begin{verbatim}
,L,M,X,J,V
09:00,"ADAI C3",,,"pI A2","pI A2"
09:30,"ADAI C3",,"ADAI C4","pI A2","pI A2"
10:00,"ADAI C3",,"ADAI C4","ADAI C1",
10:30,"ADAI C3","ADAI A1","ADAI C4","pI A2|ADAI C1",
11:00,"ADAI A1","ADAI B1","ADAI C4","ADAI C1","ADAI B4"
11:30,"pI B2","pI A2|ADAI B1",,"ADAI C1","ADAI B4"
12:00,"pI B2","ADAI B1","ADAI B2",,"ADAI B4"
12:30,"pI B2","ADAI B1","ADAI B2",,"ADAI B4"
13:00,"pI B2","ADAI B1","ADAI B2","ADAI A2","ADAI B4"
13:30,,"ADAI B1","ADAI B2",,"ADAI B4"
14:00,,"ADAI A2","ADAI B2","ADAI B6",
14:30,"ADAI B5",,"ADAI B2","ADAI B6","ADAI A2"
15:00,"ADAI B5",,,"ADAI B6","ADAI A2"
15:30,"ADAI B5",,,"ADAI B6","ADAI A2"
16:00,"ADAI B5",,,"ADAI B6",
16:30,"ADAI B5",,,"ADAI B6","ADAI A2"
17:00,"ADAI B5",,,,
17:30,,,,,
18:00,,,,,
18:30,,,,,
19:00,,,,,
19:30,,,,,
20:00,,,,,
20:30,,,,,
21:00,,,,,
21:30,,,,,
22:00,,,,,
\end{verbatim}

Visualizado desde {\em LibreOffice Calc} se vería como en la siguiente imagen:

\begin{figure}[H] 
  \label{informes-ocupacion} 
	\begin{center}
    \includegraphics[scale=1]{./aula-export.png}
  \end{center}
\caption{Fragmento de la ocupación de un aula exportada abierta con LibreOffice Calc}
\end{figure}

\section{Informes de asignatura}
 
Podemos hacer informes de asignaturas en PDF con las horas que se impartirán cada semana según los horarios, el PDF se visualizará como el de la siguiente imagen:

\begin{figure}[H] 
  \label{informes-asignatura} 
	\begin{center}
    \includegraphics[scale=1]{./informe-pdf-asignatura.png}
  \end{center}
\caption{Fragmento un informe PDF de una asignatura}
\end{figure}

\section{Calendario}

En la sección de eventos podemos hacer una exportación completa del calendario del curso al formato CSV. Debido a la dificultad por encontrar un formato adecuado, se ha optado por separar cada mes, y en cada casilla del fichero CSV, incluir un día del mes, con cada semana en cada línea.
\\
Para marcar los eventos, en lugar de poner el número del día, se marca la casilla con una X, también estarán marcados con una X, al no ser lectivos, los sábados y domingos, y los días previos o posteriores a inicio y final de curso respectivamente. 
\\
Se marcan con un guión las casillas sobrantes en cada mes que no corresponden con ningún día de ese mes.
\\
A continuación se puede ver un calendario con algunos eventos de prueba marcados para mostrar el formato en el que se exporta:

\begin{verbatim}
"Semestre 1"
"Mes 9"
X,X,X,22,23,X,X
26,27,28,29,30,-,-
"Mes 10"
-,-,-,-,-,X,X
03,04,05,06,07,X,X
10,11,12,13,14,X,X
17,18,19,20,21,X,X
24,25,26,27,28,X,X
31,-,-,-,-,-,-
"Mes 11"
-,01,02,03,04,X,X
07,08,09,10,11,X,X
14,15,16,X,18,X,X
21,22,23,24,25,X,X
28,29,30,-,-,-,-
"Mes 12"
-,-,-,01,02,X,X
05,06,07,08,09,X,X
12,13,14,15,16,X,X
19,20,21,22,23,X,X
26,27,28,29,30,X,-
"Mes 1"
-,-,-,-,-,-,X
02,03,04,05,06,X,X
X,X,X
"Semestre 2"
"Mes 2"
X,14,15,16,17,X,X
20,21,22,X,24,X,X
X,X,X,-,-,-,-
"Mes 3"
-,-,-,01,02,X,X
05,06,07,08,09,X,X
12,13,14,15,16,X,X
19,20,21,22,23,X,X
26,27,28,29,30,X,-
"Mes 4"
-,-,-,-,-,-,X
02,03,04,05,06,X,X
09,10,11,12,13,X,X
16,17,18,19,20,X,X
23,24,25,26,27,X,X
30,-,-,-,-,-,-
"Mes 5"
-,01,02,03,04,X,X
07,08,09,X,11,X,X
14,15,16,17,18,X,X
21,22,23,24,25,X,X
28,29,30,31,-,-,-
"Mes 6"
-,-,-,-,01,X,X
04,05,06,07,08,X,X

\end{verbatim}

\clearpage
\addcontentsline{toc}{chapter}{Bibliografia y referencias}
% -*-portada.tex-*-
% Este fichero es parte de la plantilla LaTeX para
% la realización de Proyectos Final de Carrera, protejido
% bajo los términos de la licencia GFDL.
% Para más información, la licencia completa viene incluida en el
% fichero fdl-1.3.tex

% Fuente tomada del PFC 'libgann' de Javier Vázquez Púa

\begin{thebibliography}{99}
\bibitem{Codeigniter_userguide}Codeigniter user guide. \url{http://codeigniter.com/user_guide}.
\bibitem{jQuery_docs} Documentación jQuery. \url{http://docs.jquery.com/}.
\bibitem{fpdf_docs} Documentación librería FPDF. \url{http://www.fpdf.org/en/doc/index.php}.
\bibitem{php_docs} Documentación oficial PHP. \url{http://es.php.net/manual/es/}.
\bibitem{jquery_book} Castledine, Earle. \emph{jQuery. Novice to ninja. ISBN: 978-0-9805768-5-6}. Sitepoint, 2010.
\bibitem{Wikibooks} Wikibooks. The book of \LaTeX. \url{http://en.wikibooks.org/wiki/LaTeX/}.
\bibitem{Web_Latex}Guía para la generación de la memoria del Proyecto Fin de Carrera.\\ \url{http://osl2.uca.es/wikiformacion/index.php/LaTeX_para_Proyecto_Fin_de_Carrera}.
\bibitem{design_patterns} Gamma, Erich. \emph{Design Patterns. Elements of Reusable Object-Oriented Software. ISBN: 0-201-63361-2}. Addison-Wesley, 1995
\bibitem{TDD_book} Beck, Kent. \emph{Test-Driven Development By Example. ISBN: 0-321-14653-0}. Addison-Wesley, 2003
%\bibitem{Bib_GIMP}Peck, Akkana. \emph{Beginning GIMP : from novice to professional}, 2ª edición. Apress, 2008. 584p. ISBN:978-1-4302-1070-2
\end{thebibliography}

%\addcontentsline{toc}{chapter}{Software usado}
%\chapter*{Software utilizado}
%\input{programas.tex}

%\addcontentsline{toc}{chapter}{Instalación de \LaTeX}
%\chapter*{Instalación de \LaTeX}
%\section{Prerrequisitos}

Para poder instalar la aplicación en un servidor debemos tener previamente instalados una serie de programas, disponibles tanto en Linux como en Windows. A continuación se enumeran estos paquetes necesarios para el funcionamiento:

\begin{itemize}

\item {\bf MySQL Server}: Es el sistema gestor de base de datos de la aplicación. En Linux se puede obtener de los repositorios, también se puede descargar de la página oficial:\\
\href{http://dev.mysql.com/downloads/mysql/}{http://dev.mysql.com/downloads/mysql/}\\
Es importante conocer la contraseña de root ya que será necesaria para crear la base de datos y el usuario al que será asociada la aplicación.
\item {\bf PHP}: Debemos tener instalada una versión de PHP igual o superior a la 5.3, se puede descargar sin problemas de los repositorios, o bien de la página oficial.
\item {\bf Apache httpd server}: También disponible tanto en la página oficial como en los repositorios.
\end{itemize}

\section{Instalación de la aplicación}
La aplicación se proporciona en un fichero .zip, así que solo habrá que descomprimirlo en una carpeta del servidor web, es importante saber la ruta desde la que se accede en el servidor, ya que habrá que configurar la aplicación apropiadamente más adelante.
\paragraph{}
Es importante no cambiar ningún fichero ni carpeta en la jerarquía de directorios de la aplicación, sino el funcionamiento podría alterarse. También es importante no borrar el fichero .htaccess disponible en la raíz de la aplicación.

\section{Puesta en funcionamiento}
Para que la aplicación funcione, necesita tener una base de datos creada, además del usuario con el que se conectará desde la aplicación. Para ello se deben seguir los pasos descritos a continuación:
\begin{itemize}
\item Lo primero que hay que hacer es acceder a {\em MySQL} con el usuario root, escribiendo desde el terminal lo siguiente:
\begin{lstlisting}[style=consola]
	mysql -u root -p
\end{lstlisting}
Y a continuación se nos pedirá la contraseña.
\item El siguiente paso es crear la base de datos con el nombre ''gestiongrados'', para ello escribimos:
\begin{lstlisting}[style=consola]
	mysql > CREATE DATABASE gestiongrados;
\end{lstlisting}
\item Una vez creada la base de datos, hay que crear el usuario ''gestiongrados'', escribimos lo siguiente en la consola de {\em MySQL}:
\begin{lstlisting}[style=consola]
	mysql > GRANT CREATE, SELECT, INSERT, DELETE, UPDATE 
	ON gestiongrados.* to 'gestiongrados'@'localhost' 
	IDENTIFIED BY 'ges1234';
\end{lstlisting}
Nótese que se ha asignado el password ges1234, puede ser cambiado, pero deberá ser configurado en la aplicación más adelante.
\item Aplicamos los cambios en la base de datos:
\begin{lstlisting}[style=consola]
	mysql > FLUSH PRIVILEGES
\end{lstlisting}
\end{itemize}

Ya tenemos la base de datos creada, pero ahora hay que modificar algunos parámetros de configuración en el fichero de configuración de la aplicación. Para ello abrimos el fichero './application/config/config.php'.\\
Este fichero únicamente hace asignaciones en un array asociativo \$config, donde cada clave es un parámetro de configuración. Debemos hacer la siguiente modificación:

\begin{itemize}
\item En primer lugar debemos modificar el valor de la clave ''base\_url'', que es el que contiene la url y ruta de la carpeta del servidor donde se ubica la aplicación, por ejemplo si el servidor es gestion.uca.es, y la ruta es /gestiongrados, el valor de la clave deberá ser:
\begin{lstlisting}[style=PHP]
$config['base_url'] = 'http://gestion.uca.es/gestiongrados';
\end{lstlisting}
\item No modificar ningún otro parámetro, ya que esto podría provocar un mal funcionamiento de la aplicación.
\end{itemize}

A continuación debemos ir al fichero './application/config/database.php' y hacer las siguientes modificaciones:
\begin{itemize}
\item Modificar el valor del hostname, que corresponderá al servidor donde estará ubicada la base de datos.
\begin{lstlisting}[style=PHP]
	$db['default']['hostname'] = 'localhost';
\end{lstlisting}
\item Modificar el nombre de usuario si se ha cambiado al crear la base de datos, sino dejar el que está ('gestiongrados'):
\begin{lstlisting}[style=PHP]
	$db['default']['username'] = 'gestiongrados';
\end{lstlisting}
\item Modificar el password si se ha modificado al crear el usuario en la base de datos, sino dejar el que está ('ges1234'):
\begin{lstlisting}[style=PHP]
	$db['default']['password'] = 'ges1234';
\end{lstlisting}
\item Modificar el nombre de la base de datos si se ha modificado al crearla, sino dejar el que está ('gestiongrados'):
\begin{lstlisting}[style=PHP]
	$db['default']['database'] = 'gestiongrados';
\end{lstlisting}
\item Dejar todos los demás parámetros tal cual están, sino se podría obtener un mal funcionamiento.
\end{itemize}

Una vez hecho esto, podremos finalizar la aplicación entrando en la ruta de instalación de la aplicación, que se encargará de crear la estructura de la base de datos además de un usuario administrador, al que podremos asignar una contraseña, para ello debemos escribir en el navegador la ruta base de la aplicación, seguido de "/install", pantalla en la que se nos pedirá una contraseña para finalizar la instalación de la aplicación, además de una dirección de correo electrónico que será la que se use para entrar en la aplicación. Una vez terminada la instalación, se nos indicará con un mensaje, y podremos empezar a trabajar con ella.



\input{fdl-1.3.tex}

\end{document}
