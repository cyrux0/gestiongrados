% -*-programas.tex-*-
% Este fichero es parte de la plantilla LaTeX para
% la realización de Proyectos Final de Carrera, protejido
% bajo los términos de la licencia GFDL.
% Para más información, la licencia completa viene incluida en el
% fichero fdl-1.3.tex

% Copyright (C) 2009 Pablo Recio Quijano 
Es usual en un PFC referenciar que software has usado para la
realización del mismo. Aprovecharé este apartado para que conozcas
alguna herramienta que puede serte de ayuda para realizar tus
documentos en \LaTeX{}

\section*{Emacs + Auc\TeX}

Emacs es uno de los programas de edición más usados por
desarrolladores de software, ya que es bastante versatil admitiendo
gran cantidad de ``plugins'' o extensiones que permiten ampliar aun
más sus funcionalidades.\\

Uno de estos plugins es Auc\TeX \cite{pdf:auctex}, el cual incluye
rutas para ciertos comandos, resaltado de sintaxis, previsualización
del documento, menú matemático en el cual podemos acceder e insertar
la gran mayoria de los símbolos matemáticos, para no tener que
memorizarlos. Podemos ver un ejemplo de Emacs + Auc\TeX en la figura
\ref{auctex}

\figura{Auctex.png}{scale=0.6}{Emacs + Auc\TeX}{auctex}{h}

Por ejemplo, para cerrar un entorno $\backslash$\texttt{begin()}, con su
respectivo $\backslash$\texttt{end()}, utilizaremos el atajo
\comando{C-c M-]}, para añadir un $\backslash$\texttt{item}, tenemos
el atajo \comando{C-c C-j}, y así unos cuantos, que una vez que nos
habituamos a ellos, son bastante cómodos.\\

Además, es bastante configurable, con indentado automático, corrector
ortográfico y demás. El fichero adjunto a este documento,
\emph{conf\_emacs} incluye una configuración con varias de estas
opciones.

\section*{Doxygen}

Realmente, \programa{Doxygen} \cite{website:doxygen} no es una herramienta
que vayamos a utilizar para realizar documentos \LaTeX{}
directmaente. Sin embargo, para la documentación de código si es
bastante util.\\

Esta herramienta realiza una documentación automática de código
fuente. Es decir, para nuestro PFC, podemos utilizar para generar la
documentación de las APIs de nuestras librerias y demás. Puede generar
esta documentación en varios formatos, y entre ellos, \LaTeX, de forma
que podemos utilizar ese código generado en nuestra memoria de forma
automática.

\section*{GNU Make}

\programa{GNU Make} es el programa de recompilación y de control de
dependencias por excelencia. Se puede utilizar para compilar proyectos
software en diversos códigos, o como en el caso de este documento,
para compilar documentos \LaTeX{} con diversas opciones.\\

Para más información \cite{pdf:make}

\section*{Dia}

\programa{Dia} es un editor de gráficos vectoriales el cual incluye
distintas plantillas para distintos tipos de gráficos, como pueden ser
UML, ERe, diagramas de flujo, esquemas Cisco de red y un larguísimo
etcétera. Podemos ver el interfaz en la figura \ref{dia}

\figura{dia.png}{scale=0.6}{Interfaz de Dia}{dia}{h}

Estos diagramas podemos exportarlos a diversos formatos de imagen
(\texttt{.png}, \texttt{.eps}, ...) o a formato \texttt{.tex}, como
vimos anteriormente.