\section{Lenguaje de programación}

Una de las decisiones principales que hay que tomar a la hora de realizar un proyecto es escoger el lenguaje de programación. Al tratarse de un proyecto web, se decide utilizar PHP, al ser uno de los lenguajes más extendidos en este tipo de desarrollo y ser uno de los que más se asemeja a la sintaxis de C, que es el lenguaje del que tenemos mayor base gracias a la carrera.\\

Además otra de las razones es su extensa documentación y su gran cantidad de librerías disponibles para extender el lenguaje.\\

También es importante destacar que ya que queríamos utilizar un framework que usara la arquitectura MVC, este lenguaje era el más indicado, ya que es el que dispone de más frameworks de este tipo, entre ellos {\em Codeigniter}, {\em CakePHP}, {\em Symfony} o {\em Zend Framework}.\\

De entre ellos se decide usar Codeigniter, al ser probablemente el más ligero de todos, lo que hace que sea el más rápido comparándolo a otros frameworks, siendo la velocidad un punto crítico en esta clase de librerías. 

\section{Entorno de desarrollo}

Otra elección importante es el entorno de desarrollo de código, ya que según la elección que hagamos, puede favorecer nuestra productividad o hacernos más lentos en nuestro trabajo. Aquí existe la posibilidad de decantarse por un simple editor de código o bien utilizar un entorno de desarrollo integrado (IDE), con múltiples herramientas que nos ayudan en nuestro trabajo. Nosotros nos decantamos por la segunda opción.
\\
En un principio se comenzó usando {\em Aptana Studio}, basado en {\em Eclipse}, pero poco después descubrimos {\em NetBeans}, un entorno pensado inicialmente para el desarrollo Java, pero adaptado a otros lenguajes. NetBeans proporciona entre otras cosas autocompletado de código y un debugger muy completo que nos ayuda a encontrar errores en el código.

\section{Herramienta UML}

Para la creación de los diagramas UML se utilizó {\em BoUML}, que permite realizar todo tipo de diagrama dentro del estándar UML 2. Incluso provee una herramienta de generación de código para múltiples lenguajes.
\\
Es una herramienta multiplataforma y gratuita.\\

Además para la creación de los diagramas entidad-relación se ha utilizado la herramienta {\em DIA}, también multiplataforma y gratuita, y encuadrada en el proyecto GNOME.

\section{Redacción de la memoria y resumen}

Para la realización de la memoria se ha utilizado \LaTeX, que es un lenguaje de marcado para la composición de textos científicos. Es una herramienta realmente fácil de usar y que da como resultado documentos de una gran calidad tipográfica, bastante más difícil de obtener con un procesador de textos normal.\\

\LaTeX es libre y multiplataforma y está basado en \TeX.

\section{Ediciones rápidas de código}

A veces es necesario hacer ediciones rápidas de código para lo que no se ve necesario y productivo abrir el IDE, para ello se ha utilizado {\em Emacs}, un completo editor multiplataforma que se encuentra en el proyecto GNU y que dándole un uso adecuado puede ser tan potente como un IDE

\section{Planificación del proyecto}

Para la gestión de la planificación de recursos se ha utilizado el software gratuito {\em Planner}, la elección se realizó ya que se había utilizado en otros proyectos y la experiencia fue satisfactoria.

