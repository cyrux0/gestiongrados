
La planificación se divide en varias fases, a continuación se explicará en detalle cada una de ellas.

\section{Fase inicial}

La primera fase consistió en el planteamiento de la idea del proyecto, que en principio era desarrollar una aplicación web. El tutor finalmente decide proponer este proyecto, dejando al alumno la libre elección de las tecnologías utilizadas para su desarrollo.

\section{Fase de análisis}

Se realizan diversas reuniones con el tutor para hacer el planteamiento y una especificación informal de los requisitos. Debido a la complejidad de la aplicación fue una labor compleja que necesitó de varias reuniones y en las que hubo cambios en los requisitos debido a la forma en la que se realiza la planificación docente de las titulaciones de grado, que ha cambiado durante los últimos años.

\section{Fase de aprendizaje}

Para la realización del proyecto usaron tecnologías de las que no se tenían conocimiento, por lo tanto fue necesaria una amplia fase de aprendizaje. Esta fase se puede dividir en tres partes, el aprendizaje de PHP como lenguaje, aprendizaje del framework {\em CodeIgniter}\footnote{Framework de desarrollo en PHP, basado en el patrón MVC} y del ORM\footnote{Object Relational Mapper. Es una técnica software para convertir datos entre sistemas incompatibles en lenguajes de programación orientados a objetos. Por ejemplo datos de una base de datos en objetos del lenguaje PHP} {\em Doctrine}\footnote{Implementación para PHP de un ORM}, y finalmente de los lenguajes de la parte del cliente, es decir, HTML y JavaScript\footnote{Lenguaje de scripting del lado del cliente. Basado en objetos}.

\section{Fase de diseño}

Fase en la que se realiza el diseño de la aplicación. Es importante hacer un buen diseño aquí para que no surjan problemas más adelante y se haga necesario hacer cambios muy costosos en el diseño.

\section{Implementación}

Fase más extensa del desarrollo del proyecto. Consiste en implementar los requisitos especificados en la fase de análisis siguiendo para ello el diseño realizado en la fase anterior, procurando que la aplicación final satisfaga las necesidades.

\section{Pruebas}

Etapa importante en la que se comprueba una por una las funcionalidades del sistema verificando que no hay errores y que todo funciona como debe.

\section{Redacción de la memoria}
Esta fase se ha ido solapando con las demás ya que se ha realizado conjuntamente a las otras a medida que se iba desarrollando el proyecto. 

\section{Diagrama de Gantt}
A continuación se muestra el diagrama de Gantt\footnote{Herramienta gráfica para mostrar el tiempo previsto para la realización de tareas dentro de un proyecto.} realizado con la herramienta {\em Planner}\footnote{Herramienta software libre para realizar diagramas de Gantt}, en el que se puede comprobar los plazos utilizados para las fases del desarrollo del proyecto.
\newpage

\figura{gantt1.PNG}{scale=0.6, angle=90}{Diagrama de Gannt. Desarrollo del
  proyecto 1/2}{gannt}{H}

\figura{gantt2.PNG}{scale=0.5, angle=90}{Diagrama de Gannt. Desarrollo del
  proyecto 1/2}{gannt}{H}


