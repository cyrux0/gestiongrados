% -*-cap1.tex-*-
% Este fichero es parte de la plantilla LaTeX para
% la realización de Proyectos Final de Carrera, protejido
% bajo los términos de la licencia GFDL.
% Para más información, la licencia completa viene incluida en el
% fichero fdl-1.3.tex

% Copyright (C) 2009 Pablo Recio Quijano 

Con este Proyecto de Fin de Carrera se pretende la consecución de dos objetivos fundamentales: poner en práctica los conocimientos adquiridos en la titulación de Ingeniería Técnica en Informática de Sistemas y buscar un incremento de los conocimientos en la rama del desarrollo web, al no haber estudiado nada de este tema durante la carrera. \\

\section{Objetivos y alcance}

El proyecto consiste en la creación de un software que ayude a los coordinadores de las titulaciones de grado a programar la planificación docente. En principio está pensado únicamente para el contexto de la Escuela Superior de Ingeniería, aunque debería ser fácilmente adaptable a otras facultades. \\

Actualmente para hacer esta planificación se utilizan hojas de cálculo de {\em Microsoft Excel} o {\em OpenOffice}, haciendo que el trabajo sea algo tedioso al tener que comprobar multitud de factores manualmente. La aplicación pretende facilitar esta labor, realizando esas comprobaciones automáticamente. Por ejemplo la tarea de realizar un horario y comprobar que un aula no esté ya ocupada por otra asignatura. Para ello el objetivo es crear una aplicación web de código abierto.\\

Otro objetivo que se pretende con este proyecto es hacerlo escalable, para que en un futuro se le puedan realizar las ampliaciones necesarias sin necesidad de cambiar demasiado lo que está ya hecho.

\section{Estructura del documento}

El documento se compone de los siguientes capítulos:

\begin{itemize}
\item {\bf Introducción:} descripción del proyecto, objectivos y alcance del mismo y estructura básica del documento.
\item {\bf Planificación:} descripción del desarrollo de la planificación temporal y plazos de realización.
\item {\bf Descripción general:} descripción detallada sobre el proyecto, especificando tecnologías y herramientas usadas para su desarrollo.
\item {\bf Análisis:} fase de análisis del sistema, empleando la metodología seleccionada. Definición de requisitos funcionales del sistema, modelo conceptual y modelo de comportamiento.
\item {\bf Diseño:} fase de diseño del sistema, diseño de la base de datos y diagramas de clase aplicadas al diseño.
\item {\bf Implementación:} aspectos más relevantes de la fase de implementación del sistema y explicación de los problemas encontrados durante el desarrollo.
\item {\bf Pruebas y validaciones:} pruebas realizadas al software para verificar que todo funciona correctamente y según lo esperado.
\item {\bf Conclusiones:} valoración y conclusiones personales obtenidas tras la realización del proyecto.
\item {\bf Apéndices:}
\begin{itemize}
\item {\bf Manual de instalación:} manual para instalar correctamente la aplicación.
\item {\bf Manual de usuario:} manual para ayudar al usuario en el uso de la aplicación.
\item {\bf Manual de importación:} ejemplos de los archivos admitidos por la aplicación para importar y exportar datos.
\item {\bf Exportación:} ejemplos de los archivos exportados por la aplicación.
\end{itemize}
\item {\bf Bibliografía:} libros y referencias consultadas durante la realización del proyecto.
\item {\bf Licencia GPL 3:} texto completo sobre la licencia GPL 3, por la cual se rige el proyecto.
\end{itemize}

\section{Definiciones y acrónimos}

A continuación se detallan las abreviaturas y acrónimos utilizados a lo largo de todo el documento.

\begin{itemize}
\item {\bf PHP:} PHP: Hypertext Preprocessor. Es un lenguaje de scripting del lado del servidor.
\item {\bf XHTML:} eXtensible Hypertext Markup Language. Lenguaje de marcado para estructurar las vistas de un documento web.
\item {\bf IDE:} Entorno de desarrollo integrado. Es una aplicación con herramientas para facilitar el trabajo de un desarrollador.
\item {\bf SQL:} Lenguaje de consulta estructurado. Lenguaje para realizar operaciones sobre una base de datos.
\item {\bf MySQL:} Sistema de gestión de base de datos relacional.
\item {\bf CSS:} Cascading Style Sheets, hojas de estilo en cascada. Utilizadas para definir el estilo de un documento web.
\item {\bf ER:} Entidad-relación. Diagrama utilizado para mostrar la especificación de una base de datos.
\item {\bf ESI:} Escuela Superior de Ingenieria.
\item {\bf MVC:} Modelo Vista Controlador. Patrón de diseño arquitectónico utilizado a la hora de hacer el diseño de un sistema.
\end{itemize}
