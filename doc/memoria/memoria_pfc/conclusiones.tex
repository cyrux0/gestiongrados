En este capítulo se comentarán las conclusiones personales alcanzadas, así como las posibles ampliaciones futuras que se le podrían hacer a la aplicación.

\section{Opinión personal}

Con este proyecto se han abarcado aspectos tanto de software de gestión como de aplicación web, conocimientos que durante la carrera se dan de forma muy básica, por lo que proyectos como este pueden ayudarme a ampliar conocimientos para desarrollos y trabajos futuros.\\

Realmente es la primera vez que me enfrento a un desarrollo web relativamente grande, ya que mi conocimiento y experiencia no pasaba de la realización de algunas páginas estáticas mediante HTML. Este es, por tanto, el primer acercamiento real al lenguaje PHP, del que he descubierto su potencia en este proyecto. Es cierto que podría haber evitado usar algún framework y utilizar simplemente el lenguaje sin ninguna ayuda, pero creo que sólo hubiera hecho la labor más tediosa y aburrida, y considero que el uso de un framework debería ser considerado obligatorio por cualquier programador a la hora de realizar una aplicación web, sea en PHP o en cualquier otro lenguaje.\\

En cuanto a JavaScript, si bien es cierto que ya lo había usado alguna otra vez, realmente nunca había probado alguna librería como  {\em jQuery}, y me ha sorprendido gratamente su potencia y facilidad de uso, y lo que se puede hacer mejorando notablemente la experiencia del usuario.\\

En cuanto a los sistemas de control de versiones, ya había tenido experiencia con {\em Subversion}, pero Git sorprende por su versatilidad, mostrándose mucho más fuerte a la hora de realizar desarrollos colaborativos. Se utilizó {\em GitHub} como servidor {\em Git}, ya que disponía de una amplia comunidad de usuarios y parecía ser de los más recomendados en la web, además de que disponía de un sistema de seguimiento de tareas muy bueno y fácil de usar.\\

En cuanto a \LaTeX no es la primera vez que lo he usado, pero esta memoria me ha servido para ampliar un poco más mi conocimiento de este lenguaje.

\section{Ampliaciones futuras}

Este proyecto está sujeto a cambios, ya que es posible que cambie la forma en la que se plantea la planificación docente de una asignatura, y aunque la aplicación se ha intentado parametrizar lo máximo posible para poder personalizar muchos aspectos, es posible que sea necesario realizar cambios.\\

Entre las posibles ampliaciones que se pueden hacer está el que un alumno pueda loguearse con su usuario habitual del campus virtual, integrándolo con el LDAP de la Universidad, no se ha hecho debido a que el objetivo principal de la aplicación es ser usada por un usuario planificador para la configuración, teniendo el alumno una funcionalidad mínima.\\

Otra posible mejora sería una cierta automatización en la creación de los horarios, por ejemplo proponiendo un profesor su horario preferente y generándose el mejor horario posible, pudiendo ser modificado luego. En este aspecto también estaria bien la posibilidad de que un alumno propusiera una configuración mejor para un horario.


